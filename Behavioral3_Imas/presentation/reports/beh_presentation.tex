% -------------------------------------
% Setup 
% -------------------------------------
\documentclass[11pt, aspectratio=169]{beamer}

\usepackage{amsmath}
\usepackage{amssymb}
\usepackage[backend=biber,style=authoryear]{biblatex}
\usepackage{booktabs}
\usepackage{epsfig}
\usepackage{fontawesome}
\usepackage{graphicx}
\usepackage{hyperref}
\usepackage[utf8x]{inputenc}
\usepackage{mathtools}
\usepackage{pgfplots}
\usepackage{tikz}

\usepackage{tcolorbox}


\mathtoolsset{showonlyrefs=true}

\DeclareCiteCommand{\cite}
  {\usebibmacro{prenote}}
  {\usebibmacro{citeindex}%
   \printnames{labelname}%
   \space(\printfield{year})}
  {\multicitedelim}
  {\usebibmacro{postnote}}




% Spacing
\usepackage{parskip}
\usepackage{etoolbox}
\usepackage{setspace}
\setlength{\itemsep}{10em}
\usepackage{adjustbox}






\usepackage{physics}
% \DeclareMathOperator*{\argmax}{arg\,max}
% \DeclareMathOperator*{\argmin}{arg\,min}
\usepackage{listings}
\usepackage{booktabs}
\usepackage{caption}
\hypersetup{
    %colorlinks=true,
    %linkcolor=blue,
    %filecolor=magenta,      
    %urlcolor=cyan  
    }
\usepackage{threeparttable}
\usepackage{pdflscape}
\usepackage{float}
%\usepackage{natbib}
\usepackage{multirow}


%%%%%%%%%%%%%%%%%%%%%%%%%%%%%%%%%%%%%%%%%%%%%%%%%%%%%%%%%%%%%%%%%%%%%%%%%%%%%%%%%%%%%%%
% Shortcuts 
%%%%%%%%%%%%%%%%%%%%%%%%%%%%%%%%%%%%%%%%%%%%%%%%%%%%%%%%%%%%%%%%%%%%%%%%%%%%%%%%%%%%%%%


% % Shortcut greek
% \def\a{\alpha}
% \def\b{\beta}
% \def\g{\gamma}
% \def\D{\Delta}
% \def\d{\delta}
% \def\z{\zeta}
% \def\k{\kappa}
% \def\l{\lambda}
% \def\n{\nu}
% \def\r{\rho}
% \def\s{\sigma}
% \def\t{\tau}
% \def\x{\xi}
% \def\w{\omega}
% \def\W{\Omega}
% \def\th{\theta}
% \newcommand{\ep}{\varepsilon}

% Probability operators
% \newcommand{\V}{\mathbb{V}}
% \renewcommand{\P}{\mathbb{P}}
% \newcommand{\E}{\mathbb{E}}
% \renewcommand{\var}{\operatorname{var}}
% \newcommand{\cov}{\operatorname{cov}}
% \newcommand{\1}{\mathbbm{1}}
% \newcommand{\R}{\mathbb{R}}
% \newcommand{\supp}{\operatorname{supp}}
% \newcommand{\pcon}{\overset{p}{\to}}
% \newcommand{\dcon}{\overset{d}{\to}}
% \newcommand{\ascon}{\overset{a.s.}{\to}}
% \newcommand{\rcon}{\overset{r}{\to}}


% % Question format
% \newcommand{\question}[1]{ \begin{center} \noindent\colorbox{gray!10}{
% \parbox{0.8\textwidth}{\vspace{0.125in} #1 \vspace{0.125in} } } \end{center} }
% \newcommand{\subq}[1]{ \begin{center} \noindent\colorbox{gray!10}{
% \parbox{0.8\textwidth}{\vspace{0.125in} #1 \vspace{0.125in} } } \end{center} }


% % Other formatting shortcuts
% \newcommand{\notimplies}{\;\not\!\!\!\implies}
% %\newcommand{\ul}{\underline}
% \renewcommand{\bf}{\textbf}
% \newcommand{\fan}{\mathcal}
% \newcommand{\red}{\textcolor{red}}
% \newcommand{\green}{\textcolor{green}}
% \newcommand{\blue}{\textcolor{blue}}
% \newcommand{\gray}{\textcolor{gray}}




% ------------------------------------------------------------------------------
% Use the metropolis beamer template
% ------------------------------------------------------------------------------
\usepackage[T1]{fontenc}
\definecolor{maroon}{rgb}{0.5, 0.0, 0.0}
\mode<presentation>
{
  \usetheme[progressbar=foot,numbering=fraction,background=light]{metropolis} 
  \usecolortheme{beaver} % or try albatross, beaver, crane, ...
  \setbeamercolor{title}{fg=maroon}
  \setbeamercolor{background canvas}{bg=white}
  \setbeamercolor{subsection title}{fg=maroon}
  \usefonttheme{professionalfonts}  % or try serif, structurebold, ...
  \setbeamertemplate{navigation symbols}{}
  \setbeamertemplate{caption}[numbered]
  \setbeamercolor{section title}{fg=white, bg=maroon}
}

% \setbeamertemplate{section page}{
%   \vfill
%   \centering
%   \begin{beamercolorbox}[sep=8pt,center,shadow=false,rounded=false]{section title}
%     \usebeamerfont{title}\usebeamercolor{title}\insertsectionhead\par%
%   \end{beamercolorbox}
%   \vfill
% }

% bibliography stuff
% \setbeamertemplate{frametitle continuation}{}
% \setbeamertemplate{bibliography entry title}{}
% \setbeamertemplate{bibliography entry location}{}
% \setbeamertemplate{bibliography entry note}{}
% \renewcommand{\bibsection}{} %natbib

% \newenvironment{transitionframe}{
%   \setbeamercolor{background canvas}{bg=maroon,fg=white}
%   \color{white}
%   \centering
%   \LARGE
%   \begin{center}
%   \begin{frame}[plain, noframenumbering]}{
%   \end{frame}
%   \end{center}
% }

\setbeamertemplate{note page}[plain]
\usepackage{appendixnumberbeamer}



% Counter for notes
\newcounter{notescounter}

% Define the notes environment
\newenvironment{notes}[1][]{
  \refstepcounter{notescounter}%
  \if\relax\detokenize{#1}\relax
    % If #1 is empty, set the title to "Notes"
    \begin{tcolorbox}[
      colback=blue!10,
      colframe=blue!50,
      fonttitle=\bfseries,
      title={Notes},
      arc=5mm,
      boxrule=0.5mm,
      width=\textwidth,
      before skip=9pt,
      after skip=9pt
    ]
  \else
    % If #1 is not empty, use it as the title
    \begin{tcolorbox}[
      colback=blue!10,
      colframe=blue!50,
      fonttitle=\bfseries,
      title={#1},
      arc=5mm,
      boxrule=0.5mm,
      width=\textwidth,
      before skip=9pt,
      after skip=9pt
    ]
  \fi
}{
  \end{tcolorbox}
}

%\addbibresource{references.bib}


\newenvironment{transitionframe}{
  \setbeamercolor{background canvas}{bg=maroon,fg=white}
  \color{white}
  \centering
  \LARGE
  \begin{center}
  \begin{frame}[plain, noframenumbering]}{
  \end{frame}
  \end{center}
}

% -------------------------------------
% Begin Document 
% -------------------------------------
% -------------------------------------
% Begin Document 
% -------------------------------------
\begin{document}
\title{Behavioral Attenuation}
\subtitle{\normalsize Ben Enke, Thomas Graeber, Ryan Oprea, and Jeffrey Yang;  \textit{Working Paper.}}
\author{Shivani Pandey and Dylan Baker}
\date{April 22, 2025}

\begin{frame}[plain, noframenumbering] 
\maketitle
\end{frame}

\begin{frame}{Variation in Parameters and Pre-Registered Simple Points}

\begin{itemize}
    \item The authors 
        varied the central decision-relevant 
        parameter over a wide range, which 
        they argue moves problem complexity,
        to explore the ``demand side'' 
        of information-processing.
    \vfill
    \item In most experiments, they included 
        ``potential simple points'' at natural
        boundaries at which optimizing was 
        expected to be easy.
\end{itemize}
    
\end{frame}

%%%%%%%%%%%%%%%%%%%%%%%%%%%%%%%%%%%%%%%%%%%%%%%%%%%%%%%%%%%%%%%%%%%%%%%%%%%%%%%%%%%%%%%

\begin{frame}{Cognitive Uncertainty Elicitation}

    \begin{itemize}
        \item The authors elicited cognitive uncertainty 
            essentially by asking participants
            how certain they were that they 
            optimized.
        \vfill
        \item Example:
            \begin{itemize}
                \item Continuous decisions in subjective tasks, illustrated by Effort supply: 
                ``How certain are you that completing somewhere between [Y-1] and [Y+1] tasks is actually your best decision, given your preferences?''
            \end{itemize}
    \end{itemize}
    
\end{frame}

%%%%%%%%%%%%%%%%%%%%%%%%%%%%%%%%%%%%%%%%%%%%%%%%%%%%%%%%%%%%%%%%%%%%%%%%%%%%%%%%%%%%%%%

% % Slide 3: Main Results
\begin{transitionframe}
    Results
\end{transitionframe}

%%%%%%%%%%%%%%%%%%%%%%%%%%%%%%%%%%%%%%%%%%%%%%%%%%%%%%%%%%%%%%%%%%%%%%%%%%%%%%%%%%%%%%%

%%%%%%%%%%%%%%%%%%%%%%%%%%%%%%%%%%%%%%%%%%%%%%%%%%%%%%%%%%%%%%%%%%%%%%%%%%%%%%%%%%%%%%%

\begin{frame}{Example Experimental Results - Objective Task}

    \begin{columns}
    
        \begin{column}{0.5\textwidth}
            \begin{itemize}
                \item To the right is an example of 
                    the results from an objective task.
                \item Note that attenuation is present
                    across most of the parameter space,
                    i.e., the elasticity is smaller 
                    than expected for an optimizing agent.
                \item This is
                    especially true once off the boundary
                    and is stronger for high CU participants.
            \end{itemize}
        \end{column}
    
        \begin{column}{0.5\textwidth}
            \includegraphics[width=\linewidth]{../input/fig1a.png}
        \end{column}
    
    \end{columns}
    
\end{frame}
    

%%%%%%%%%%%%%%%%%%%%%%%%%%%%%%%%%%%%%%%%%%%%%%%%%%%%%%%%%%%%%%%%%%%%%%%%%%%%%%%%%%%%%%%

\begin{frame}{Example Experimental Results - Subjective Task}
    
    \begin{columns}
    
        \begin{column}{0.5\textwidth}
            \begin{itemize}
                \item To the right is an example of 
                    the results from a subjective task.
                \item Although there's no objective benchmark, 
                    we can see a similar pattern emerge.
            \end{itemize}
        \end{column}
    
        \begin{column}{0.5\textwidth}
            \includegraphics[width=\linewidth]{../input/fig1b.png}
        \end{column}
    
    \end{columns}
    
\end{frame}

%%%%%%%%%%%%%%%%%%%%%%%%%%%%%%%%%%%%%%%%%%%%%%%%%%%%%%%%%%%%%%%%%%%%%%%%%%%%%%%%%%%%%%%

\begin{frame}{Econometric Strategy}
    
    \begin{itemize}
        \item For each experiment, $e$, the authors 
            estimate
            \begin{align}
                a_{i, j}^e=\alpha^e+\gamma^e \theta_j^e+\beta^e \theta_j^e C U_{i, j}^e+\delta^e C U_{i, j}^e+\sum_x \chi^e d_x^e+\epsilon_{i, j}^e
            \end{align}
            where
            \begin{center}
            \begin{minipage}{0.85\linewidth}
                \begin{multicols}{2}
                    \begin{itemize}
                        \setlength\itemsep{2pt}
                        \item $i$: Individual
                        \item $j$: Parameter order, i.e., $\theta_j>\theta_{j-1}$
                        \item $a_{i, j}^e$: Decision
                        \item $d_x^e$: Controls 
                        \item $C U_{i, j}^e$: Elicited cognitive uncertainty
                        \item $\theta_j^e$: Parameter value
                        \item $\gamma^e$: Coefficient on the parameter normalized to be positive
                        \item $\beta^e$: Coefficient of interest on the interaction between CU and parameter
                    \end{itemize}
                \end{multicols}
            \end{minipage}
            \end{center}
    \end{itemize}
    
\end{frame}

%%%%%%%%%%%%%%%%%%%%%%%%%%%%%%%%%%%%%%%%%%%%%%%%%%%%%%%%%%%%%%%%%%%%%%%%%%%%%%%%%%%%%%%

\begin{frame}{T-Stat Results}


    \begin{columns}
    
        \begin{column}{0.3\textwidth}
            \begin{itemize}
                \item The attenuation hypothesis 
                    is that $\beta^e$ is negative.
                \item The authors report the 
                    t-statistic for
                    each experiment.
                \item The black-curve presents 
                    a $\mathcal{N}(0,1)$ distribution
                    for reference.
                \item 28/30 are negative; 24/30 are 
                    significant at the 5\% level.
            \end{itemize}
        \end{column}
    
        \begin{column}{0.7\textwidth}
            \includegraphics[width=\linewidth]{../input/fig2a.png}
        \end{column}
    
    \end{columns}
    
\end{frame}

%%%%%%%%%%%%%%%%%%%%%%%%%%%%%%%%%%%%%%%%%%%%%%%%%%%%%%%%%%%%%%%%%%%%%%%%%%%%%%%%%%%%%%%

\begin{frame}{Attenuation Ratio}
    
    \begin{itemize}
        \item Under the authors' specification, we have:
            \begin{align}
                \frac{\partial \mathbb{E}\left[a_{i, j}^e\right]}{\partial \theta_j^e}=\gamma^e+\beta^e \mathrm{CU}
            \end{align}
        \item From which, we can define another measure for 
            cross-experiment comparison:
            \begin{align}
                C U \text { attenuation ratio } & \equiv \frac{(\text { Sensitivity at } C U=0)-(\text { Sensitivity at } C U=0.5)}{(\text { Sensitivity at } C U=0)} \\
                & =-\frac{0.5 \hat{\beta}^e}{\hat{\gamma}^e}
            \end{align}
        \item Interpretation: How much the sensitivity
            decreases as CU increases from 0 to 0.5.
    \end{itemize}
    
\end{frame}

%%%%%%%%%%%%%%%%%%%%%%%%%%%%%%%%%%%%%%%%%%%%%%%%%%%%%%%%%%%%%%%%%%%%%%%%%%%%%%%%%%%%%%%

\begin{frame}{Attenuation Ratio Figure}
    
    \begin{columns}
    
        \begin{column}{0.3\textwidth}
            \begin{itemize}
                \item The reduction in sensitivity 
                    is 33\% on average and up to 87\%. 
            \end{itemize}
        \end{column}
    
        \begin{column}{0.7\textwidth}
            \includegraphics[width=\linewidth]{../input/fig2b.png}
        \end{column}
    
    \end{columns}

\end{frame}

%%%%%%%%%%%%%%%%%%%%%%%%%%%%%%%%%%%%%%%%%%%%%%%%%%%%%%%%%%%%%%%%%%%%%%%%%%%%%%%%%%%%%%%

\begin{frame}{Attenuation to Objective Benchmarks}

    \begin{itemize}
        \item We then consider the subset of experiments 
            with objective benchmarks.
        \item The authors estimate:
            \begin{align}
                a_{i, j}^e=v^e+\omega^e \theta_j^e+\sum_x \chi^e d_x^e+u_{i, j}^e
            \end{align}
            where the term of interest is 
            $\omega^e$.
        \item Specifically, we consider 
            the ratio between the estimated
            $\omega^e$ and the rational benchmark: 
            \begin{align}
                \frac{\hat{\omega}^e}{\omega_R^e}
            \end{align}
    \end{itemize}
    
\end{frame}

%%%%%%%%%%%%%%%%%%%%%%%%%%%%%%%%%%%%%%%%%%%%%%%%%%%%%%%%%%%%%%%%%%%%%%%%%%%%%%%%%%%%%%%

\begin{frame}{Attenuation to Objective Benchmark}
    
    \begin{columns}
    
        \begin{column}{0.3\textwidth}
            \begin{itemize}
                \item The black dots reflect $\frac{\hat{\omega}^e}{\omega_R^e}$ 
                \item The red and blue symbols plot the fitted values 
                    at 100\% CU and 0\% CU, respectively.
                \item The main takeaway is to notice that all 
                    dots are below one and the high and low CU 
                    split as expected. 
            \end{itemize}
        \end{column}
    
        \begin{column}{0.7\textwidth}
            \includegraphics[width=\linewidth]{../input/fig3.png}
        \end{column}
    
    \end{columns}
    
\end{frame}

%%%%%%%%%%%%%%%%%%%%%%%%%%%%%%%%%%%%%%%%%%%%%%%%%%%%%%%%%%%%%%%%%%%%%%%%%%%%%%%%%%%%%%%

\begin{frame}{Within- and Across-Subject Variation}

    \begin{itemize}
        \item Subject FE 
            explain 44\% of the variation in CU,
            suggesting that subject-level
            differences in cognitive ability 
            -- the so-called ``supply side''
            of information processing -- 
            may be important for attenuation
        \item The authors then estimate an adjusted version 
            of the earlier specification 
                \begin{align}
                    a_{i, j}^e=\alpha^e+\gamma^e \theta_j^e+\beta^e \theta_j^e \overline{CU}_{i}^e+\delta^e \overline{CU}_{i}^e+\sum_x \chi^e d_x^e+\epsilon_{i, j}^e
                \end{align}
            with $\overline{CU}_{i}^e$ rather than $ C U_{i, j}^e$
            reflecting the use of the subject-level average,
            rather than the individual decision's CU.
        \item This results in a large drop in the 
            average attenuation effect size (33.0 to 8.8) and
            t-statistic (-4.8 to -1.39),
            suggesting a meaningful role for within-subject variation.
    \end{itemize}
    
\end{frame}

%%%%%%%%%%%%%%%%%%%%%%%%%%%%%%%%%%%%%%%%%%%%%%%%%%%%%%%%%%%%%%%%%%%%%%%%%%%%%%%%%%%%%%%

% % Slide 3: Main Results
\begin{transitionframe}
    Problem Complexity and Diminishing Sensitivity
\end{transitionframe}

%%%%%%%%%%%%%%%%%%%%%%%%%%%%%%%%%%%%%%%%%%%%%%%%%%%%%%%%%%%%%%%%%%%%%%%%%%%%%%%%%%%%%%%


\begin{frame}{Problem Complexity and Diminishing Sensitivity}
    
    \begin{itemize}
        \item The authors now 
            leverage the fact that,
            within a given task type, 
            some configurations of the 
            task will demand more 
            information-processing 
            than others.
        \vfill
        \item Motivated by this, we 
            first return to the ``simple points''
            referenced earlier.
    \end{itemize}
    
\end{frame}

%%%%%%%%%%%%%%%%%%%%%%%%%%%%%%%%%%%%%%%%%%%%%%%%%%%%%%%%%%%%%%%%%%%%%%%%%%%%%%%%%%%%%%%

\begin{frame}{CU and Distance from Boundary}
    
    \begin{columns}
    
        \begin{column}{0.4\textwidth}
            \begin{itemize}
                \item The figure to the right 
                    shows median CU by distance from the
                    boundary.
                \item Notice that at the boundary, 
                    median CU tends to be 0.
                \item The solid line displays 
                    the overall median across 
                    experiments.
            \end{itemize}
        \end{column}
    
        \begin{column}{0.6\textwidth}
            \includegraphics[width=\linewidth]{../input/fig4.png}
        \end{column}
    
    \end{columns}
    
\end{frame}

%%%%%%%%%%%%%%%%%%%%%%%%%%%%%%%%%%%%%%%%%%%%%%%%%%%%%%%%%%%%%%%%%%%%%%%%%%%%%%%%%%%%%%%

\begin{frame}{Diminishing Sensitivity}
    
    \begin{itemize}
        \item The presented model 
            predicts that there should be diminished 
            sensitivity to the parameter value
            as the distance from the boundary
            increases due to the heightened 
            CU.
        \item To assess for diminished sensitivity, the 
            authors estimate
            \begin{align}
                a_{i, j}^e=\alpha_d^e+\gamma_d^e \theta_j^e+\beta_d^e \theta_j^e \Delta_j^e+\delta^e \Delta_j^e+\sum_x \chi^e d_x^e+v_{i, j}^e
            \end{align}
            where $\Delta_j^e$ is the distance from the
            nearest boundary
        \item Diminished sensitivity is captured by: $\hat{\beta}_d^e<0$.
    \end{itemize}

\end{frame}

%%%%%%%%%%%%%%%%%%%%%%%%%%%%%%%%%%%%%%%%%%%%%%%%%%%%%%%%%%%%%%%%%%%%%%%%%%%%%%%%%%%%%%%

\begin{frame}{Diminished Sensitivity T Statistics}


    \begin{columns}
    
        \begin{column}{0.4\textwidth}
            \begin{itemize}
                \item The figure to the 
                    right displays the 
                    t-statistics for  
                    $\hat{\beta}_d^e$.
                \item Almost are all 
                    negative and statistically 
                    significant, as expected.
            \end{itemize}
        \end{column}
    
        \begin{column}{0.7\textwidth}
            \includegraphics[width=\linewidth]{../input/fig5a.png}
        \end{column}
    
    \end{columns}
    
\end{frame}

%%%%%%%%%%%%%%%%%%%%%%%%%%%%%%%%%%%%%%%%%%%%%%%%%%%%%%%%%%%%%%%%%%%%%%%%%%%%%%%%%%%%%%%

\begin{frame}{Across-Problem Variation in Complexity and Elasticity}
    

    \begin{columns}
    
        \begin{column}{0.4\textwidth}
            \begin{itemize}
                \item We now link variation 
                    in task complexity 
                    and diminished sensitivity
                    by comparing
                    sensitivity and CU not 
                    across individuals, but 
                    across task parameters.
                \item The authors construct comparable 
                    measures of CU and sensitivity
                    across experiments by computing 
                    the local average and normalizing 
                    by the experiment average. (See 
                    figure to the right.)
            \end{itemize}
        \end{column}
    
        \begin{column}{0.7\textwidth}
            \includegraphics[width=\linewidth]{../input/fig5b.png}
        \end{column}
    
    \end{columns}

\end{frame}
%%%%%%%%%%%%%%%%%%%%%%%%%%%%%%%%%%%%%%%%%%%%%%%%%%%%%%%%%%%%%%%%%%%%%%%%%%%%%%%%%%%%%%%


    
    
\begin{frame}{Discussion}
    
    \begin{itemize}
        \item Behavioral attenuation can (at least partially)
            explain many previously studied
            anomalies, including many explored in the paper.
        \item Additionally, the authors argue that 
            behavioral attenuation offers
            a more parsimonious explanation
            than the oft-offered alternative of 
            domain-specific preferences.
        \item Finally, the authors conclude 
            with a note of optimism for the 
            power of the cognitive lens to 
            speak to a wide variety of 
            economic phenomena with 
            only a limited set of principles.
    \end{itemize}
    
\end{frame}




\end{document}