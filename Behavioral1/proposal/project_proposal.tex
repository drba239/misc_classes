\documentclass[10pt]{article}
\usepackage{amsmath}
\usepackage{amsthm}
\usepackage{amsfonts}
\usepackage{amssymb}
\usepackage{amssymb}
\usepackage{booktabs}
\setlength\parindent{0pt}
\usepackage[margin=1.2in]{geometry}
\usepackage{enumitem}
\usepackage{mathtools}
\mathtoolsset{showonlyrefs=true}
\usepackage{pdflscape}
\usepackage{xcolor}
\usepackage{hyperref}
\setcounter{tocdepth}{4}
\setcounter{secnumdepth}{4}
\usepackage[listings,skins,breakable]{tcolorbox} % package for colored boxes
\usepackage{etoolbox}
\usepackage{placeins}
\usepackage{tikz}
\usepackage{color}  % Allows for color customization
\usepackage{subcaption}
\usepackage[utf8]{inputenc}


% Make it so that the bottom page of a 
% book section doesn't have weird spacing
\raggedbottom


% Define custom colors
% You can-redo these later, they're not being used 
% for anything as of 5/12/24
\definecolor{codegreen}{rgb}{0,0.6,0}
\definecolor{codegray}{rgb}{0.5,0.5,0.5}
\definecolor{codepurple}{rgb}{0.58,0,0.82}
\definecolor{backcolour}{rgb}{0.95,0.95,0.92}

% You can-redo the lstlisting style later, it's not being used 
% for anything as of 5/12/24

% Define the lstlisting style
\lstdefinestyle{mystyle}{
    backgroundcolor=\color{backcolour},   
    commentstyle=\color{codegreen},
    keywordstyle=\color{magenta},
    numberstyle=\tiny\color{codegray},
    stringstyle=\color{codepurple},
    basicstyle=\ttfamily\footnotesize,
    breakatwhitespace=false,         
    breaklines=true,                 
    captionpos=b,                    
    keepspaces=true,                 
    numbers=left,                    
    numbersep=5pt,                  
    showspaces=false,                
    showstringspaces=false,
    showtabs=false,                  
    tabsize=2
}
\lstset{style=mystyle}


% Set the length of \parskip to add a line between paragraphs
\setlength{\parskip}{1em}


% Set the second level of itemize to use \circ as the bullet point
\setlist[itemize,2]{label={$\circ$}}

% Define symbols
\DeclareMathSymbol{\Perp}{\mathrel}{symbols}{"3F}
\newcommand\barbelow[1]{\stackunder[1.2pt]{$#1$}{\rule{.8ex}{.075ex}}}
\newcommand{\succprec}{\mathrel{\mathpalette\succ@prec{\succ\prec}}}
\newcommand{\precsucc}{\mathrel{\mathpalette\succ@prec{\prec\succ}}}


\newcounter{example}[section] % Reset example counter at each new section
\renewcommand{\theexample}{\thesection.\arabic{example}} % Format the example number as section.number

\newenvironment{example}
  {% Begin environment
   \refstepcounter{example}% Step counter and allow for labeling
   \noindent\textbf{Example \theexample.} % Display the example number
  }
  {% End environment
   \par\noindent\hfill\textit{End of Example.}\par
  }



% deeper section command
% This will let you go one level deeper than whatever section level you're on.
\makeatletter
\newcommand{\deepersection}[1]{%
  \ifnum\value{subparagraph}>0
    % Already at the deepest standard level (\subparagraph), cannot go deeper
    \subparagraph{#1}
  \else
    \ifnum\value{paragraph}>0
      \subparagraph{#1}
    \else
      \ifnum\value{subsubsection}>0
        \paragraph{#1}
      \else
        \ifnum\value{subsection}>0
          \subsubsection{#1}
        \else
          \ifnum\value{section}>0
            \subsection{#1}
          \else
            \section{#1}
          \fi
        \fi
      \fi
    \fi
  \fi
}
\makeatother


% same section command
% This will let create a section at the same level as whatever section level you're on.
\makeatletter
\newcommand{\samesection}[1]{%
  \ifnum\value{subparagraph}>0
    \subparagraph{#1}
  \else
    \ifnum\value{paragraph}>0
      \paragraph{#1}
    \else
      \ifnum\value{subsubsection}>0
        \subsubsection{#1}
      \else
        \ifnum\value{subsection}>0
          \subsection{#1}
        \else
          \ifnum\value{section}>0
            \section{#1}
          \else
            % Default to section if outside any sectioning
            \section{#1}
          \fi
        \fi
      \fi
    \fi
  \fi
}
\makeatother

\makeatletter
\newcommand{\shallowersection}[1]{%
  \ifnum\value{subparagraph}>0
    \paragraph{#1} % From subparagraph to paragraph
  \else
    \ifnum\value{paragraph}>0
      \subsubsection{#1} % From paragraph to subsubsection
    \else
      \ifnum\value{subsubsection}>0
        \subsection{#1} % From subsubsection to subsection
      \else
        \ifnum\value{subsection}>0
          \section{#1} % From subsection to section
        \else
          \ifnum\value{section}>0
            \chapter{#1} % Assuming a document class with chapters
          \else
            \section{#1} % Default to section if somehow higher than section
          \fi
        \fi
      \fi
    \fi
  \fi
}
\makeatother



\newcounter{problemcounter}
\renewcommand{\theproblemcounter}{Q.\arabic{problemcounter}}

% Define the problem environment
\newenvironment{problem}[1][]{%
  \refstepcounter{problemcounter}%
  \if\relax\detokenize{#1}\relax
    \tcolorbox[breakable, colback=red!10, colframe=red!50, fonttitle=\bfseries, title={Problem \theproblemcounter}, arc=5mm, boxrule=0.5mm]
  \else
    \tcolorbox[breakable, colback=red!10, colframe=red!50, fonttitle=\bfseries, title={Problem \theproblemcounter: #1}, arc=5mm, boxrule=0.5mm]
    \addcontentsline{toc}{subsubsection}{\theproblemcounter: #1}%
  \fi
}{
  \endtcolorbox
}

% Define a new counter for definitions
\newcounter{definitioncounter}
\renewcommand{\thedefinitioncounter}{D.\arabic{definitioncounter}}

\newenvironment{definition}[1][]{%
  \refstepcounter{definitioncounter}%
  \if\relax\detokenize{#1}\relax
    \tcolorbox[
      breakable,
      parbox=false, % Treat content normally regarding paragraphs
      before upper={\parindent0pt \parskip7pt}, % No indentation and add space between paragraphs
      colback=blue!10,
      colframe=blue!50,
      fonttitle=\bfseries,
      title={Definition \thedefinitioncounter},
      arc=5mm,
      boxrule=0.5mm,
      before skip=10pt, % Adjust vertical space before the box
      after skip=10pt % Adjust vertical space after the box
    ]
  \else
    \tcolorbox[
      breakable,
      parbox=false,
      before upper={\parindent0pt \parskip7pt},
      colback=blue!10,
      colframe=blue!50,
      fonttitle=\bfseries,
      title={Definition \thedefinitioncounter: #1},
      arc=5mm,
      boxrule=0.5mm,
      before skip=10pt,
      after skip=10pt
    ]
  \fi
}{
  \endtcolorbox
}


% Define a new counter for theorems
\newcounter{theoremcounter}
\renewcommand{\thetheoremcounter}{T.\arabic{theoremcounter}}

\newenvironment{theorem}[1][]{%
  \refstepcounter{theoremcounter}%
  \if\relax\detokenize{#1}\relax
    \tcolorbox[
      breakable,
      parbox=false, % Treat content normally regarding paragraphs
      before upper={\parindent0pt \parskip7pt}, % No indentation and add space between paragraphs
      colback=green!10,
      colframe=green!55,
      fonttitle=\bfseries,
      title={Theorem \thetheoremcounter},
      arc=5mm,
      boxrule=0.5mm,
      before skip=10pt, % Adjust vertical space before the box
      after skip=10pt % Adjust vertical space after the box
    ]
  \else
    \tcolorbox[
      breakable,
      parbox=false,
      before upper={\parindent0pt \parskip7pt},
      colback=green!10,
      colframe=green!55,
      fonttitle=\bfseries,
      title={Theorem \thetheoremcounter: #1},
      arc=5mm,
      boxrule=0.5mm,
      before skip=10pt,
      after skip=10pt
    ]
  \fi
}{
  \endtcolorbox
}


% Define a new counter for remarks
\newcounter{remarkcounter}
\renewcommand{\theremarkcounter}{R.\arabic{remarkcounter}}

\newenvironment{remark}[1][]{%
  \refstepcounter{remarkcounter}%
  \if\relax\detokenize{#1}\relax
    \tcolorbox[
      breakable,
      parbox=false, % Treat content normally regarding paragraphs
      before upper={\parindent0pt \parskip7pt}, % No indentation and add space between paragraphs
      colback=green!10,
      colframe=green!55,
      fonttitle=\bfseries,
      title={Remark \theremarkcounter},
      arc=5mm,
      boxrule=0.5mm,
      before skip=10pt, % Adjust vertical space before the box
      after skip=10pt % Adjust vertical space after the box
    ]
  \else
    \tcolorbox[
      breakable,
      parbox=false,
      before upper={\parindent0pt \parskip7pt},
      colback=green!10,
      colframe=green!55,
      fonttitle=\bfseries,
      title={Remark \theremarkcounter: #1},
      arc=5mm,
      boxrule=0.5mm,
      before skip=10pt,
      after skip=10pt
    ]
  \fi
}{
  \endtcolorbox
}

% Define a new counter for lemmas
\newcounter{lemmacounter}
\renewcommand{\thelemmacounter}{L.\arabic{lemmacounter}}

\newenvironment{lemma}[1][]{%
  \refstepcounter{lemmacounter}%
  \if\relax\detokenize{#1}\relax
    \tcolorbox[
      breakable,
      parbox=false, % Treat content normally regarding paragraphs
      before upper={\parindent0pt \parskip7pt}, % No indentation and add space between paragraphs
      colback=green!10,
      colframe=green!55,
      fonttitle=\bfseries,
      title={Lemma \thelemmacounter},
      arc=5mm,
      boxrule=0.5mm,
      before skip=10pt, % Adjust vertical space before the box
      after skip=10pt % Adjust vertical space after the box
    ]
  \else
    \tcolorbox[
      breakable,
      parbox=false,
      before upper={\parindent0pt \parskip7pt},
      colback=green!10,
      colframe=green!55,
      fonttitle=\bfseries,
      title={Lemma \thelemmacounter: #1},
      arc=5mm,
      boxrule=0.5mm,
      before skip=10pt,
      after skip=10pt
    ]
  \fi
}{
  \endtcolorbox
}

% Define a new counter for propositions
\newcounter{propositioncounter}
\renewcommand{\thepropositioncounter}{P.\arabic{propositioncounter}}

\newenvironment{proposition}[1][]{%
  \refstepcounter{propositioncounter}%
  \if\relax\detokenize{#1}\relax
    \tcolorbox[
      breakable,
      parbox=false, % Treat content normally regarding paragraphs
      before upper={\parindent0pt \parskip7pt}, % No indentation and add space between paragraphs
      colback=green!10,
      colframe=green!55,
      fonttitle=\bfseries,
      title={Proposition \thepropositioncounter},
      arc=5mm,
      boxrule=0.5mm,
      before skip=10pt, % Adjust vertical space before the box
      after skip=10pt % Adjust vertical space after the box
    ]
  \else
    \tcolorbox[
      breakable,
      parbox=false,
      before upper={\parindent0pt \parskip7pt},
      colback=green!10,
      colframe=green!55,
      fonttitle=\bfseries,
      title={Proposition \thepropositioncounter: #1},
      arc=5mm,
      boxrule=0.5mm,
      before skip=10pt,
      after skip=10pt
    ]
  \fi
}{
  \endtcolorbox
}

%\newtheorem{proposition}[theorem]{Proposition}  % Propositions share numbering with theorems


% Define a new counter for notes
\newcounter{notescounter}
\renewcommand{\thenotescounter}{D.\arabic{notescounter}}

\newenvironment{notes}[1][]{
  \refstepcounter{notescounter}%
  \if\relax\detokenize{#1}\relax
    % If #1 is empty, set the title to "Notes"
    \tcolorbox[
      breakable,
      parbox=false, % Treat content normally regarding paragraphs
      before upper={\parindent0pt \parskip7pt}, % No indentation and add space between paragraphs
      colback=blue!10,
      colframe=blue!50,
      fonttitle=\bfseries,
      title={Notes},
      arc=5mm,
      boxrule=0.5mm,
      before skip=10pt, % Adjust vertical space before the box
      after skip=10pt % Adjust vertical space after the box
    ]
  \else
    % If #1 is not empty, use it as the title
    \tcolorbox[
      breakable,
      parbox=false,
      before upper={\parindent0pt \parskip7pt},
      colback=blue!10,
      colframe=blue!50,
      fonttitle=\bfseries,
      title={#1}, % Use provided title instead of default
      arc=5mm,
      boxrule=0.5mm,
      before skip=10pt,
      after skip=10pt
    ]
  \fi
}{
  \endtcolorbox
}







% Define a new counter for questions
\newcounter{questionscounter}
\renewcommand{\thequestionscounter}{D.\arabic{questionscounter}}

\newenvironment{questions}[1][]{
  \refstepcounter{questionscounter}%
  \if\relax\detokenize{#1}\relax
    % If #1 is empty, set the title to "Questions"
    \tcolorbox[
      breakable,
      parbox=false, % Treat content normally regarding paragraphs
      before upper={\parindent0pt \parskip7pt}, % No indentation and add space between paragraphs
      colback=red!10,
      colframe=red!50,
      fonttitle=\bfseries,
      title={Questions},
      arc=5mm,
      boxrule=0.5mm,
      before skip=10pt, % Adjust vertical space before the box
      after skip=10pt % Adjust vertical space after the box
    ]
  \else
    % If #1 is not empty, use it as the title
    \tcolorbox[
      breakable,
      parbox=false,
      before upper={\parindent0pt \parskip7pt},
      colback=red!10,
      colframe=red!50,
      fonttitle=\bfseries,
      title={#1}, % Use provided title instead of default
      arc=5mm,
      boxrule=0.5mm,
      before skip=10pt,
      after skip=10pt
    ]
  \fi
}{
  \endtcolorbox
}






% Define a new counter for overview
\newcounter{overviewcounter}
\renewcommand{\theoverviewcounter}{D.\arabic{overviewcounter}}

\newenvironment{overview}[1][]{%
  \refstepcounter{overviewcounter}%
  \if\relax\detokenize{#1}\relax
    \tcolorbox[
      breakable,
      parbox=false, % Treat content normally regarding paragraphs
      before upper={\parindent0pt \parskip7pt}, % No indentation and add space between paragraphs
      colback=green!10,
      colframe=green!55,
      fonttitle=\bfseries,
      title={Overview},
      arc=5mm,
      boxrule=0.5mm,
      before skip=10pt, % Adjust vertical space before the box
      after skip=10pt % Adjust vertical space after the box
    ]
  \else
    \tcolorbox[
      breakable,
      parbox=false,
      before upper={\parindent0pt \parskip7pt},
      colback=green!10,
      colframe=green!55,
      fonttitle=\bfseries,
      title={Overview},
      arc=5mm,
      boxrule=0.5mm,
      before skip=10pt,
      after skip=10pt
    ]
  \fi
}{
  \endtcolorbox
}



% Add a line after paragraph header
\makeatletter
\renewcommand\paragraph{\@startsection{paragraph}{4}{\z@}%
            {-3.25ex \@plus -1ex \@minus -.2ex}%
            {1.5ex \@plus .2ex}%
            {\normalfont\normalsize\bfseries}}
\makeatother


% Taking away line before and after align
\BeforeBeginEnvironment{align}{\vspace{-\parskip}}
\AfterEndEnvironment{align}{\vskip0pt plus 2pt}

\usepackage{changepage}
\usepackage{titlesec}

% Adjust spacing for sections
\titlespacing*{\section}{0pt}{1.5ex plus 1ex minus .2ex}{0.5ex plus .1ex}
\titlespacing*{\subsection}{0pt}{1.5ex plus 1ex minus .2ex}{0.5ex plus .1ex}

% Draft toggle
\newif\ifdraft
%\drafttrue % Set to \drafttrue for draft mode, \draftfalse for final mode
\draftfalse

\ifdraft
\else
\usepackage[style=apa, backend=biber]{biblatex}
\addbibresource{references.bib}
\fi

\title{Project Proposal}

\begin{document}

\begin{titlepage}
    \pagestyle{empty} % Ensures no page numbers for the entire page
    \centering
    \vspace*{\fill}
    {\Huge\bfseries Project Proposal\par}
    \vspace{1.5cm}
    {\Large Dylan Baker\par}
    \vspace{1.5cm}
    {\large BUSN 38916\par}
    \vfill
    \date{\today}
\end{titlepage}

\newpage 

\pagenumbering{arabic}

\section{Background}

I'm interested in the role of ingroup observation, direct contact, 
and, most centrally, ``vicarious contact'' in informing cooperation behavior
in settings marked by intergroup conflict.
Intergroup contact theory
\ifdraft(allport1954) \else\parencite{allport1954} \fi
has been a 
prominent research topic in 
psychology for more than half a century
\ifdraft(hewstone2011) \else\parencite{hewstone2011} \fi. 
In more recent decades, 
economists have also given 
it
significant attention
\ifdraft(bazzi2019, lowe2021) \else\parencite{bazzi2019, lowe2021} \fi
. 
\ifdraft(allport1954) \else\textcite{allport1954} \fi 
introduced the ``contact hypothesis,'' which
suggested that intergroup prejudice can be mitigated if members of 
the groups engage in contact that
is characterized by certain conditions, e.g., common goals.

Decades after the introduction of the contact hypothesis,
psychologists began to explore the possibility that 
a similar reduction in prejudice could be achieved through 
``\emph{indirect} intergroup contact''. In particular, studies 
began to suggest that prejudice could be reduced via 
simply 1) knowing that a member of one's in-group has a friend
in the out-group (known as ``extended contact''\footnote{In some papers, 
``extended contact'' refers to any form of indirect contact. I am adopting the
terminology used in 
\ifdraft(mazziotta2011) \else\textcite{mazziotta2011} \fi}),
2) watching intergroup interactions between 
members of one's in-group and the out-group (``vicarious contact''), or
3) imagining intergroup interactions between 
members of one's in-group and the out-group (``imagined contact'').
In this proposal, I will focus on vicarious contact.

Vicarious contact has been the subject of a number of studies in psychology
but has seemingly not garnered the same level of attention from economists.
The seminal paper on vicarious contact is 
\ifdraft(wright1997) \else\textcite{wright1997} \fi.
\ifdraft(wright1997) \else\textcite{wright1997} \fi, in the context of 
a minimal group paradigm with undergraduate psychology student participants, 
found that witnessing intergroup interaction in a puzzle game 
led to more positive attitudes towards the out-group when 
the interaction was positive, as compared to neutral or negative.

As two motivating examples, \ifdraft(schiappa2005) \else\textcite{schiappa2005} \fi
found that having participants watch television shows
with central characters that were gay led to lower levels of prejudice
against gay people. Meanwhile, \ifdraft(paluck2009) \else\textcite{paluck2009} \fi
randomly assigned participants in Rwanda to listen to a 
radio soap-opera-style broadcast highlighting inter-ethnic 
reconciliation and cooperation 
in a fictional setting within Rwanda or a 
health-drama-oriented control broadcast.
They found that the former group was more likely to 
endorse views
and perceptions of social norms
corresponding to greater inter-ethnic cooperation, such 
as a greater willingness for their children to marry 
someone from outside of their ethnic group.

\section{Proposed Study}

The target participants would be 
individuals
from two ethnic, racial, or religious groups in a context in which
there is intergroup tension. I would congregate participants 
in a community center or rented building (or portion of building) that had rooms such 
that I could separate 
people as needed.


\subsection{Study Design}

The general structure of the experimental exercise would be as follows:

\subsubsection{Phase 1}

In the first phase of the game, two players will be paired, 
in some cases prompted to have a guided conversation, 
and asked to play a cooperative game. The cooperative game will 
share some features with the Public Goods Game and Trust Game. 
Specifically, the structure will be that each player is given an equivalent endowment.
They are then asked to decide how much of their endowment to send to the other person.
The amount sent will then be doubled upon delivery to the other player.
In the most cooperative case, both players could double their initial endowment;
in the least cooperative case, both players could keep their initial endowment;
in a one-sided full cooperation, the non-cooperative player could triple their 
endowment and the fully-cooperative player could earn nothing (from this 
exercise), and in many cases, the players would end up somewhere in between.

In all cases, the players will be observed by another two participants, though 
whether they are aware of this will vary. The general structure of Phase 1 
is as follows:

\begin{enumerate}
    \item Group-Categorized Pair Assignment: Participants would be assigned a pair, which could be centrally 
        characterized for analysis as:
        \begin{enumerate}[label=\roman*.]
            \item Same Group 1: Both participants are from Group 1.
            \item Same Group 2: Both participants are from Group 2.
            \item Mixed Group: One participant is from Group 1 and the other is from Group 2.
        \end{enumerate}
    \item Pre-Game Direct Contact: The members of the pair would either 
        engage in direct contact prior to the main game or not.
        \begin{enumerate}[label=\roman*.]
            \item Pre-Game Direct Contact: Participants would engage in a conversation with prompts before the main game.
            \item No Pre-Game Direct Contact: Participants would not speak before the main game.
        \end{enumerate}
    \item Cooperative Game with (Un)known Observers: 
        The pair would engage in a cooperative game
        where participants vary at the pair-level in their knowledge of observers watching them.
        \begin{enumerate}[label=\roman*.]
            \item No Information: Participants would play the cooperation game 
                under the presumption of no observers.
            \item Ingroup Observers: Participants would play the cooperation game 
                and be told that they are being observed by members of their own group.
            \item Generic Observers: Participants would play the cooperation game 
                and be told that they are being observed by other participants, 
                without specifying any additional information about the observers.
        \end{enumerate}
\end{enumerate}

\subsubsection{Phase 2}

In Phase 1, two observers were assigned to watch each player.
In Phase 2, the observers are paired to play a set of games themselves.


\begin{enumerate}
    \item Pair-Type Assignment: Observers would be (without their knowledge) assigned into a ``pair type.'' The specific 
        partner would be determined after the next step.
        \begin{enumerate}[label=\roman*.]
            \item Same Group 1: Both participants are from Group 1.
            \item Same Group 2: Both participants are from Group 2.
            \item Mixed Group: One participant is from Group 1 and the other is from Group 2.
        \end{enumerate}
    \item Partner Choice Sub-Game
        \begin{enumerate}[label=\roman*.]
            \item Observers would be told that they are going to play a game 
                that involves cooperation. They would be provided with a list of 
                6 potential partners:
                3 from their own group and 3 from the other group.
                They would be asked to choose 4 that they 
                would be willing to play the game with.\footnote{This description is analogous to 
                an exercise performed in 
                \ifdraft(blouin2019) \else\textcite{blouin2019}\fi} 
                Asking for 4 choices
                ensures that the participants' categorization from Step 1 above 
                can be maintained.
        \end{enumerate}
    \item Observers Play a Game
        \begin{enumerate}[label=\roman*.]
            \item Cooperation Game: A subset of observer-pairs would play the same cooperation game from Phase 1. 
            \item Dictator Game: A subset of observer-pairs would play the Dictator Game.
        \end{enumerate}
\end{enumerate}

\section{Identification Strategy}

There are many analyses that would be interesting. Given page limits, I will 
focus on what I 
perceive to be the most central comparisons. In Phase 1, as a baseline, 
we'd be interested in the main effects of each of 
the three steps, which would allow us to 
assess whether cooperation is lower in the mixed group, 
whether direct contact increases cooperation,\footnote{This could also usefully induce 
more variation in cooperation that may be helpful for varying ``treatment dosage'' 
received by Phase 2 players (Phase 1 observers).} and whether 
observation influences cooperation. We would also be interested in the 
effect of ``Mixed Group - Ingroup Observers'' compared against ``Mixed Group - No 
Information'' to see if 
this reduces cooperation amid reputational concerns. Taking a step further, 
we'd want to compare ``Mixed Group - Ingroup Observers - Pre-Game Direct Contact''
against ``Mixed Group - Ingroup Observers - No Pre-Game Direct Contact'' to see if
direct contact can mitigate the negative effect of ingroup observers, supposing it exists.
We could assess these dynamics by estimating a model with interaction terms.


In Phase 2, we would want to assess the effects of vicarious contact on behavior 
towards outgroup members among the 
observers. First, in the Partner Choice Sub-Game, we would use 
the number of outgroup members in their list of 4 choices as an outcome variable 
and look at the effect of having observed a mixed-group pair in Phase 1, in comparison
to a same-group pair of either group, controlling for performance. Note that comparing against both types of 
same-group pairs allows us to assess the impact of intergroup contact 
relative to both a control of the observer only observing their own group, as well 
as a control --that could be construed as a different variety of treatment -- 
of simply indirect \emph{exposure} to the outgroup.

After the Partner Choice Sub-Game, we want to examine the effect of 
vicarious contact on the outcome of the observer-turned-player's behavior 
in the 
cooperation game. To do this, we want to look at the performance of the 
mixed-group
observers that observed mixed-group pairs in Phase 1 compared against the 
mixed-group observers that observed each variety of same-group pairs
in Phase 1, controlling for 
performance. We would want to conduct a similar analysis for the Dictator Game
to analyze the effect on intergroup affect.
Moreover, if the intergroup contact 
is less familiar to the observer than ingroup contact, then 
we may see greater updating occur when they observe the mixed-group pair
than when they observe a same-group pair (for their ingroup). This could be 
assessed by comparing the behavior of mixed-group observers who watched 
mixed-group pairs with same-group observers who watched same-group pairs 
(of their own group)
and seeing how their behavior corresponds to the behavior they watched 
versus the average first-round behavior in the condition they watched. 

% Currently, this proposed study faces a 
% number of challenges. First, I have this set up 
% to have many cells.


%%%%%%%%%%%%%%%%%%%%%%%%%%%%%%%%%%%%%%%%%%%%%%%%%%%%%%%%%%%%%%%%%%%%%%%%%%%%%%%%%%%%%%%
%%%%%%%%%%%%%%%%%%%%%%%%%%%%%%%%%%%%%%%%%%%%%%%%%%%%%%%%%%%%%%%%%%%%%%%%%%%%%%%%%%%%%%%
%%%%%%%%%%%%%%%%%%%%%%%%%%%%%%%%%%%%%%%%%%%%%%%%%%%%%%%%%%%%%%%%%%%%%%%%%%%%%%%%%%%%%%%

\ifdraft
% Skip bibliography in draft mode
\else
\printbibliography
\fi

\end{document}