% -------------------------------------
% Setup 
% -------------------------------------
\documentclass[11pt, aspectratio=169]{beamer}

\usepackage{amsmath}
\usepackage{amssymb}
\usepackage[backend=biber,style=authoryear]{biblatex}
\usepackage{booktabs}
\usepackage{epsfig}
\usepackage{fontawesome}
\usepackage{graphicx}
\usepackage{hyperref}
\usepackage[utf8x]{inputenc}
\usepackage{mathtools}
\usepackage{pgfplots}
\usepackage{tikz}
\usepackage{multicol}

\usepackage{tcolorbox}


\mathtoolsset{showonlyrefs=true}

\DeclareCiteCommand{\cite}
  {\usebibmacro{prenote}}
  {\usebibmacro{citeindex}%
   \printnames{labelname}%
   \space(\printfield{year})}
  {\multicitedelim}
  {\usebibmacro{postnote}}




% Spacing
\usepackage{parskip}
\usepackage{etoolbox}
\usepackage{setspace}
\setlength{\itemsep}{10em}
\usepackage{adjustbox}






\usepackage{physics}
% \DeclareMathOperator*{\argmax}{arg\,max}
% \DeclareMathOperator*{\argmin}{arg\,min}
\usepackage{listings}
\usepackage{booktabs}
\usepackage{caption}
\hypersetup{
    %colorlinks=true,
    %linkcolor=blue,
    %filecolor=magenta,      
    %urlcolor=cyan  
    }
\usepackage{threeparttable}
\usepackage{pdflscape}
\usepackage{float}
%\usepackage{natbib}
\usepackage{multirow}


%%%%%%%%%%%%%%%%%%%%%%%%%%%%%%%%%%%%%%%%%%%%%%%%%%%%%%%%%%%%%%%%%%%%%%%%%%%%%%%%%%%%%%%
% Shortcuts 
%%%%%%%%%%%%%%%%%%%%%%%%%%%%%%%%%%%%%%%%%%%%%%%%%%%%%%%%%%%%%%%%%%%%%%%%%%%%%%%%%%%%%%%


% % Shortcut greek
% \def\a{\alpha}
% \def\b{\beta}
% \def\g{\gamma}
% \def\D{\Delta}
% \def\d{\delta}
% \def\z{\zeta}
% \def\k{\kappa}
% \def\l{\lambda}
% \def\n{\nu}
% \def\r{\rho}
% \def\s{\sigma}
% \def\t{\tau}
% \def\x{\xi}
% \def\w{\omega}
% \def\W{\Omega}
% \def\th{\theta}
% \newcommand{\ep}{\varepsilon}

% Probability operators
% \newcommand{\V}{\mathbb{V}}
% \renewcommand{\P}{\mathbb{P}}
% \newcommand{\E}{\mathbb{E}}
% \renewcommand{\var}{\operatorname{var}}
% \newcommand{\cov}{\operatorname{cov}}
% \newcommand{\1}{\mathbbm{1}}
% \newcommand{\R}{\mathbb{R}}
% \newcommand{\supp}{\operatorname{supp}}
% \newcommand{\pcon}{\overset{p}{\to}}
% \newcommand{\dcon}{\overset{d}{\to}}
% \newcommand{\ascon}{\overset{a.s.}{\to}}
% \newcommand{\rcon}{\overset{r}{\to}}


% % Question format
% \newcommand{\question}[1]{ \begin{center} \noindent\colorbox{gray!10}{
% \parbox{0.8\textwidth}{\vspace{0.125in} #1 \vspace{0.125in} } } \end{center} }
% \newcommand{\subq}[1]{ \begin{center} \noindent\colorbox{gray!10}{
% \parbox{0.8\textwidth}{\vspace{0.125in} #1 \vspace{0.125in} } } \end{center} }


% % Other formatting shortcuts
% \newcommand{\notimplies}{\;\not\!\!\!\implies}
% %\newcommand{\ul}{\underline}
% \renewcommand{\bf}{\textbf}
% \newcommand{\fan}{\mathcal}
% \newcommand{\red}{\textcolor{red}}
% \newcommand{\green}{\textcolor{green}}
% \newcommand{\blue}{\textcolor{blue}}
% \newcommand{\gray}{\textcolor{gray}}




% ------------------------------------------------------------------------------
% Use the metropolis beamer template
% ------------------------------------------------------------------------------
\usepackage[T1]{fontenc}
\definecolor{maroon}{rgb}{0.5, 0.0, 0.0}
\mode<presentation>
{
  \usetheme[progressbar=foot,numbering=fraction,background=light]{metropolis} 
  \usecolortheme{beaver} % or try albatross, beaver, crane, ...
  \setbeamercolor{title}{fg=maroon}
  \setbeamercolor{background canvas}{bg=white}
  \setbeamercolor{subsection title}{fg=maroon}
  \usefonttheme{professionalfonts}  % or try serif, structurebold, ...
  \setbeamertemplate{navigation symbols}{}
  \setbeamertemplate{caption}[numbered]
  \setbeamercolor{section title}{fg=white, bg=maroon}
}

% \setbeamertemplate{section page}{
%   \vfill
%   \centering
%   \begin{beamercolorbox}[sep=8pt,center,shadow=false,rounded=false]{section title}
%     \usebeamerfont{title}\usebeamercolor{title}\insertsectionhead\par%
%   \end{beamercolorbox}
%   \vfill
% }

% bibliography stuff
% \setbeamertemplate{frametitle continuation}{}
% \setbeamertemplate{bibliography entry title}{}
% \setbeamertemplate{bibliography entry location}{}
% \setbeamertemplate{bibliography entry note}{}
% \renewcommand{\bibsection}{} %natbib

% \newenvironment{transitionframe}{
%   \setbeamercolor{background canvas}{bg=maroon,fg=white}
%   \color{white}
%   \centering
%   \LARGE
%   \begin{center}
%   \begin{frame}[plain, noframenumbering]}{
%   \end{frame}
%   \end{center}
% }

\setbeamertemplate{note page}[plain]
\usepackage{appendixnumberbeamer}



% Counter for notes
\newcounter{notescounter}

% Define the notes environment
\newenvironment{notes}[1][]{
  \refstepcounter{notescounter}%
  \if\relax\detokenize{#1}\relax
    % If #1 is empty, set the title to "Notes"
    \begin{tcolorbox}[
      colback=blue!10,
      colframe=blue!50,
      fonttitle=\bfseries,
      title={Notes},
      arc=5mm,
      boxrule=0.5mm,
      width=\textwidth,
      before skip=9pt,
      after skip=9pt
    ]
  \else
    % If #1 is not empty, use it as the title
    \begin{tcolorbox}[
      colback=blue!10,
      colframe=blue!50,
      fonttitle=\bfseries,
      title={#1},
      arc=5mm,
      boxrule=0.5mm,
      width=\textwidth,
      before skip=9pt,
      after skip=9pt
    ]
  \fi
}{
  \end{tcolorbox}
}

%\addbibresource{references.bib}

% -------------------------------------
% Begin Document 
% -------------------------------------
\begin{document}
\title{From Immigrants to Americans: Race and Assimilation during the Great Migration}
\subtitle{By Vasiliki Fouka, Soumyajit Mazumder, \& Marco Tabellini}
\author{By Dylan Baker}
\date{ }

\begin{frame}[plain, noframenumbering] 
\maketitle
\end{frame}

%%%%%%%%%%%%%%%%%%%%%%%%%%%%%%%%%%%%%%%%%%%%%%%%%%%%%%%%%%%%%%%%%%%%%%%%%%%%%%%%%%%%%%%
%%%%%%%%%%%%%%%%%%%%%%%%%%%%%%%%%%%%%%%%%%%%%%%%%%%%%%%%%%%%%%%%%%%%%%%%%%%%%%%%%%%%%%%

\begin{frame}{Context}

    \begin{itemize}
        \item Historical Context
            \begin{itemize}
                \item In the early 20th century, many European 
                    immigrants were viewed as culturally distant from native born 
                    Americans and faced discrimination.
                \item The Great Migration was a period of mass 
                    migration of African Americans from the rural South 
                    to the urban North in the US from 1910-1970.
                \item Historians have argued that the Great Migration 
                    fueled the change from structuring race 
                    around ethnicity to structuring it around skin color.
            \end{itemize}
        \item Academic Context
            \begin{itemize}
                \item This paper builds on a large psychology literature 
                    on self-categorization (Turner et al. (1987)) and 
                    outgroup bias (Tajfel et al. (1971)).
                \item In economics and political science, Bordalo et al. (2016) argues 
                    for the context dependence 
                    of group stereotypes and Shayo (2009) develops 
                    a model of social identity oriented around perceived distance 
                    between groups.
            \end{itemize}
    \end{itemize}

\end{frame}

%%%%%%%%%%%%%%%%%%%%%%%%%%%%%%%%%%%%%%%%%%%%%%%%%%%%%%%%%%%%%%%%%%%%%%%%%%%%%%%%%%%%%%%
%%%%%%%%%%%%%%%%%%%%%%%%%%%%%%%%%%%%%%%%%%%%%%%%%%%%%%%%%%%%%%%%%%%%%%%%%%%%%%%%%%%%%%%

\begin{frame}{Question \& Data}

\centering
\textbf{Question}

\vspace{-0.4cm}
\hrulefill\hspace{0.5em}\dotfill\hspace{0.5em}\hrulefill

\begin{itemize}
    \item How does the introduction of a social group (Black migrants) 
    perceived to be more culturally distant to the majority (native-born 
    white Americans) affect the assimilation of a traditionally marginalized 
    group (European immigrants)?
\end{itemize}

\textbf{Data}

\vspace{-0.4cm}
\hrulefill\hspace{0.5em}\dotfill\hspace{0.5em}\hrulefill

\begin{itemize}
    \item Census Data: Decennial US Census Data from 1900-1930 from 
        the 108 non-Southern MSAs with 
        positive inflow of Southern Black migrants.
        \begin{itemize}
            \item Key variables: Naturalization status, 
                marriage status (dummy for marriage with native-born American), 
                and ethnic distinctiveness of children's names.
        \end{itemize}
    \item Newspaper Data: Articles compiled from Newspapers.com from 71 MSAs.
        \begin{itemize}
            \item Key variables: Usage of immigrant-oriented and anti-Black 
                language
        \end{itemize}
\end{itemize}

\end{frame}

%%%%%%%%%%%%%%%%%%%%%%%%%%%%%%%%%%%%%%%%%%%%%%%%%%%%%%%%%%%%%%%%%%%%%%%%%%%%%%%%%%%%%%%
%%%%%%%%%%%%%%%%%%%%%%%%%%%%%%%%%%%%%%%%%%%%%%%%%%%%%%%%%%%%%%%%%%%%%%%%%%%%%%%%%%%%%%%

\begin{frame}{Analytical Technique}

\begin{itemize}
    \item DiD: Compares change in Black population across MSAs within region
        \vspace{-0.2cm} % Adjust this value as needed
        \begin{align}
            Y_{i n r t}=\underbrace{\alpha_n}_{\parbox[t]{1.2cm}{\raggedright \footnotesize \centering MSA FE}}
            +\underbrace{(\delta_r \times \gamma_t)_{r t}}_{\parbox[t]{2cm}{\raggedright \footnotesize \centering Time $\times$ Region FE}}
            +\beta_1 \underbrace{B_{n t}}_{\parbox[t]{1.4cm}{\raggedright \footnotesize \centering Black Pop in $n$ at $t$}}
            +\beta_2 \underbrace{\operatorname{Pop}_{n t}}_{\parbox[t]{1.5cm}{\raggedright \footnotesize \centering Total Pop in $n$ at $t$}}
            +\underbrace{\mathbf{X}_{i n r t}^{\prime}}_{\parbox[t]{1.5cm}{\raggedright \footnotesize \centering Individual controls}} \Gamma
            +u_{i n r t}
        \end{align}
        \vspace{-1cm} % Adjust this value as needed
    \item Shift-Share Instrument for Black Population
        \vspace{-0.2cm} % Adjust this value as needed
        \begin{align}
            Z_{n t}=\sum_{s=1910}^t \sum_{j \in \text { South }} \underbrace{\alpha_{j n}^{1900}}_{\parbox[t]{3.3cm}{\raggedright \footnotesize \centering Share of Black migrants from state $j$ living in MSA $n$ in 1900}} \underbrace{O_{j s}}_{\parbox[t]{3cm}{\raggedright \footnotesize \centering \# of Black migrants who left state $j$ between $s-1$ and $s$}}
        \end{align}
        \vspace{-0.6cm} % Adjust this value as needed
        \begin{itemize}
            \item Migrants tended to move to the destination MSAs of past migrants from their region.
            \item Uses cross-sectional variation over $n$ in 1900 Black migrant state-of-origin mix \& time-series variation in out-migration from each Southern state post-1900.
        \end{itemize}
\end{itemize}

\end{frame}

%%%%%%%%%%%%%%%%%%%%%%%%%%%%%%%%%%%%%%%%%%%%%%%%%%%%%%%%%%%%%%%%%%%%%%%%%%%%%%%%%%%%%%%
%%%%%%%%%%%%%%%%%%%%%%%%%%%%%%%%%%%%%%%%%%%%%%%%%%%%%%%%%%%%%%%%%%%%%%%%%%%%%%%%%%%%%%%

\begin{frame}{Results}

An increase in the Black population led to:

    \begin{itemize}
        \item Heightened Assimilation Efforts by European Immigrants
            \begin{itemize}
                \item Increased Naturalization Rates: Rose 1.5 p.p. per 1 S.D. ($\sim$45,000 people) increase
                \item Less Ethnically Distinct Names: A 100,000 increase led to a name distinctiveness decline among Italians equivalent to changing Luciano to Mike
            \end{itemize}
        \item Greater Acceptance of European Immigrants Among Native Born White People
            \begin{itemize}
                \item Increased Intermarriage Rates: Rose 0.54 p.p. (7.5\%) per 1 S.D. increase
                \item Decreased Anti-immigrant News Coverage
            \end{itemize}
        \item Elevated Anti-Black Sentiment
            \begin{itemize}
                \item Increased the Frequency of Anti-Black Stereotypes in the Press
            \end{itemize}
    \end{itemize}

\end{frame}

% \begin{frame}[plain, noframenumbering]{References}
%     \printbibliography
% \end{frame}


\end{document}