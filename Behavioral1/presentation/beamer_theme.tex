
\usepackage{amsmath}
\usepackage{amssymb}
\usepackage[backend=biber,style=authoryear]{biblatex}
\usepackage{booktabs}
\usepackage{epsfig}
\usepackage{fontawesome}
\usepackage{graphicx}
\usepackage{hyperref}
\usepackage[utf8x]{inputenc}
\usepackage{mathtools}
\usepackage{pgfplots}
\usepackage{tikz}

\usepackage{tcolorbox}


\mathtoolsset{showonlyrefs=true}

\DeclareCiteCommand{\cite}
  {\usebibmacro{prenote}}
  {\usebibmacro{citeindex}%
   \printnames{labelname}%
   \space(\printfield{year})}
  {\multicitedelim}
  {\usebibmacro{postnote}}




% Spacing
\usepackage{parskip}
\usepackage{etoolbox}
\usepackage{setspace}
\setlength{\itemsep}{10em}
\usepackage{adjustbox}






\usepackage{physics}
% \DeclareMathOperator*{\argmax}{arg\,max}
% \DeclareMathOperator*{\argmin}{arg\,min}
\usepackage{listings}
\usepackage{booktabs}
\usepackage{caption}
\hypersetup{
    %colorlinks=true,
    %linkcolor=blue,
    %filecolor=magenta,      
    %urlcolor=cyan  
    }
\usepackage{threeparttable}
\usepackage{pdflscape}
\usepackage{float}
%\usepackage{natbib}
\usepackage{multirow}


%%%%%%%%%%%%%%%%%%%%%%%%%%%%%%%%%%%%%%%%%%%%%%%%%%%%%%%%%%%%%%%%%%%%%%%%%%%%%%%%%%%%%%%
% Shortcuts 
%%%%%%%%%%%%%%%%%%%%%%%%%%%%%%%%%%%%%%%%%%%%%%%%%%%%%%%%%%%%%%%%%%%%%%%%%%%%%%%%%%%%%%%


% % Shortcut greek
% \def\a{\alpha}
% \def\b{\beta}
% \def\g{\gamma}
% \def\D{\Delta}
% \def\d{\delta}
% \def\z{\zeta}
% \def\k{\kappa}
% \def\l{\lambda}
% \def\n{\nu}
% \def\r{\rho}
% \def\s{\sigma}
% \def\t{\tau}
% \def\x{\xi}
% \def\w{\omega}
% \def\W{\Omega}
% \def\th{\theta}
% \newcommand{\ep}{\varepsilon}

% Probability operators
% \newcommand{\V}{\mathbb{V}}
% \renewcommand{\P}{\mathbb{P}}
% \newcommand{\E}{\mathbb{E}}
% \renewcommand{\var}{\operatorname{var}}
% \newcommand{\cov}{\operatorname{cov}}
% \newcommand{\1}{\mathbbm{1}}
% \newcommand{\R}{\mathbb{R}}
% \newcommand{\supp}{\operatorname{supp}}
% \newcommand{\pcon}{\overset{p}{\to}}
% \newcommand{\dcon}{\overset{d}{\to}}
% \newcommand{\ascon}{\overset{a.s.}{\to}}
% \newcommand{\rcon}{\overset{r}{\to}}


% % Question format
% \newcommand{\question}[1]{ \begin{center} \noindent\colorbox{gray!10}{
% \parbox{0.8\textwidth}{\vspace{0.125in} #1 \vspace{0.125in} } } \end{center} }
% \newcommand{\subq}[1]{ \begin{center} \noindent\colorbox{gray!10}{
% \parbox{0.8\textwidth}{\vspace{0.125in} #1 \vspace{0.125in} } } \end{center} }


% % Other formatting shortcuts
% \newcommand{\notimplies}{\;\not\!\!\!\implies}
% %\newcommand{\ul}{\underline}
% \renewcommand{\bf}{\textbf}
% \newcommand{\fan}{\mathcal}
% \newcommand{\red}{\textcolor{red}}
% \newcommand{\green}{\textcolor{green}}
% \newcommand{\blue}{\textcolor{blue}}
% \newcommand{\gray}{\textcolor{gray}}




% ------------------------------------------------------------------------------
% Use the metropolis beamer template
% ------------------------------------------------------------------------------
\usepackage[T1]{fontenc}
\definecolor{maroon}{rgb}{0.5, 0.0, 0.0}
\mode<presentation>
{
  \usetheme[progressbar=foot,numbering=fraction,background=light]{metropolis} 
  \usecolortheme{beaver} % or try albatross, beaver, crane, ...
  \setbeamercolor{title}{fg=maroon}
  \setbeamercolor{background canvas}{bg=white}
  \setbeamercolor{subsection title}{fg=maroon}
  \usefonttheme{professionalfonts}  % or try serif, structurebold, ...
  \setbeamertemplate{navigation symbols}{}
  \setbeamertemplate{caption}[numbered]
  \setbeamercolor{section title}{fg=white, bg=maroon}
}

% \setbeamertemplate{section page}{
%   \vfill
%   \centering
%   \begin{beamercolorbox}[sep=8pt,center,shadow=false,rounded=false]{section title}
%     \usebeamerfont{title}\usebeamercolor{title}\insertsectionhead\par%
%   \end{beamercolorbox}
%   \vfill
% }

% bibliography stuff
% \setbeamertemplate{frametitle continuation}{}
% \setbeamertemplate{bibliography entry title}{}
% \setbeamertemplate{bibliography entry location}{}
% \setbeamertemplate{bibliography entry note}{}
% \renewcommand{\bibsection}{} %natbib

% \newenvironment{transitionframe}{
%   \setbeamercolor{background canvas}{bg=maroon,fg=white}
%   \color{white}
%   \centering
%   \LARGE
%   \begin{center}
%   \begin{frame}[plain, noframenumbering]}{
%   \end{frame}
%   \end{center}
% }

\setbeamertemplate{note page}[plain]
\usepackage{appendixnumberbeamer}



% Counter for notes
\newcounter{notescounter}

% Define the notes environment
\newenvironment{notes}[1][]{
  \refstepcounter{notescounter}%
  \if\relax\detokenize{#1}\relax
    % If #1 is empty, set the title to "Notes"
    \begin{tcolorbox}[
      colback=blue!10,
      colframe=blue!50,
      fonttitle=\bfseries,
      title={Notes},
      arc=5mm,
      boxrule=0.5mm,
      width=\textwidth,
      before skip=9pt,
      after skip=9pt
    ]
  \else
    % If #1 is not empty, use it as the title
    \begin{tcolorbox}[
      colback=blue!10,
      colframe=blue!50,
      fonttitle=\bfseries,
      title={#1},
      arc=5mm,
      boxrule=0.5mm,
      width=\textwidth,
      before skip=9pt,
      after skip=9pt
    ]
  \fi
}{
  \end{tcolorbox}
}
