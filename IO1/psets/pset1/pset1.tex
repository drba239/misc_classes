\documentclass[10pt]{article}
\usepackage{amsmath}
\usepackage{amsthm}
\usepackage{amsfonts}
\usepackage{amssymb}
\usepackage{amssymb}
\usepackage{booktabs}
\setlength\parindent{0pt}
\usepackage[margin=1.2in]{geometry}
\usepackage{enumitem}
\usepackage{mathtools}
\mathtoolsset{showonlyrefs=true}
\usepackage{pdflscape}
\usepackage{xcolor}
\usepackage{hyperref}
\setcounter{tocdepth}{4}
\setcounter{secnumdepth}{4}
\usepackage[listings,skins,breakable]{tcolorbox} % package for colored boxes
\usepackage{etoolbox}
\usepackage{placeins}
\usepackage{tikz}
\usepackage{color}  % Allows for color customization
\usepackage{subcaption}
\usepackage[utf8]{inputenc}


% Make it so that the bottom page of a 
% book section doesn't have weird spacing
\raggedbottom


% Define custom colors
% You can-redo these later, they're not being used 
% for anything as of 5/12/24
\definecolor{codegreen}{rgb}{0,0.6,0}
\definecolor{codegray}{rgb}{0.5,0.5,0.5}
\definecolor{codepurple}{rgb}{0.58,0,0.82}
\definecolor{backcolour}{rgb}{0.95,0.95,0.92}

% You can-redo the lstlisting style later, it's not being used 
% for anything as of 5/12/24

% Define the lstlisting style
\lstdefinestyle{mystyle}{
    backgroundcolor=\color{backcolour},   
    commentstyle=\color{codegreen},
    keywordstyle=\color{magenta},
    numberstyle=\tiny\color{codegray},
    stringstyle=\color{codepurple},
    basicstyle=\ttfamily\footnotesize,
    breakatwhitespace=false,         
    breaklines=true,                 
    captionpos=b,                    
    keepspaces=true,                 
    numbers=left,                    
    numbersep=5pt,                  
    showspaces=false,                
    showstringspaces=false,
    showtabs=false,                  
    tabsize=2
}
\lstset{style=mystyle}


% Set the length of \parskip to add a line between paragraphs
\setlength{\parskip}{1em}


% Set the second level of itemize to use \circ as the bullet point
\setlist[itemize,2]{label={$\circ$}}

% Define symbols
\DeclareMathSymbol{\Perp}{\mathrel}{symbols}{"3F}
\newcommand\barbelow[1]{\stackunder[1.2pt]{$#1$}{\rule{.8ex}{.075ex}}}
\newcommand{\succprec}{\mathrel{\mathpalette\succ@prec{\succ\prec}}}
\newcommand{\precsucc}{\mathrel{\mathpalette\succ@prec{\prec\succ}}}


\newcounter{example}[section] % Reset example counter at each new section
\renewcommand{\theexample}{\thesection.\arabic{example}} % Format the example number as section.number

\newenvironment{example}
  {% Begin environment
   \refstepcounter{example}% Step counter and allow for labeling
   \noindent\textbf{Example \theexample.} % Display the example number
  }
  {% End environment
   \par\noindent\hfill\textit{End of Example.}\par
  }



% deeper section command
% This will let you go one level deeper than whatever section level you're on.
\makeatletter
\newcommand{\deepersection}[1]{%
  \ifnum\value{subparagraph}>0
    % Already at the deepest standard level (\subparagraph), cannot go deeper
    \subparagraph{#1}
  \else
    \ifnum\value{paragraph}>0
      \subparagraph{#1}
    \else
      \ifnum\value{subsubsection}>0
        \paragraph{#1}
      \else
        \ifnum\value{subsection}>0
          \subsubsection{#1}
        \else
          \ifnum\value{section}>0
            \subsection{#1}
          \else
            \section{#1}
          \fi
        \fi
      \fi
    \fi
  \fi
}
\makeatother


% same section command
% This will let create a section at the same level as whatever section level you're on.
\makeatletter
\newcommand{\samesection}[1]{%
  \ifnum\value{subparagraph}>0
    \subparagraph{#1}
  \else
    \ifnum\value{paragraph}>0
      \paragraph{#1}
    \else
      \ifnum\value{subsubsection}>0
        \subsubsection{#1}
      \else
        \ifnum\value{subsection}>0
          \subsection{#1}
        \else
          \ifnum\value{section}>0
            \section{#1}
          \else
            % Default to section if outside any sectioning
            \section{#1}
          \fi
        \fi
      \fi
    \fi
  \fi
}
\makeatother

\makeatletter
\newcommand{\shallowersection}[1]{%
  \ifnum\value{subparagraph}>0
    \paragraph{#1} % From subparagraph to paragraph
  \else
    \ifnum\value{paragraph}>0
      \subsubsection{#1} % From paragraph to subsubsection
    \else
      \ifnum\value{subsubsection}>0
        \subsection{#1} % From subsubsection to subsection
      \else
        \ifnum\value{subsection}>0
          \section{#1} % From subsection to section
        \else
          \ifnum\value{section}>0
            \chapter{#1} % Assuming a document class with chapters
          \else
            \section{#1} % Default to section if somehow higher than section
          \fi
        \fi
      \fi
    \fi
  \fi
}
\makeatother



\newcounter{problemcounter}
\renewcommand{\theproblemcounter}{Q.\arabic{problemcounter}}

% Define the problem environment
\newenvironment{problem}[1][]{%
  \refstepcounter{problemcounter}%
  \if\relax\detokenize{#1}\relax
    \tcolorbox[breakable, colback=red!10, colframe=red!50, fonttitle=\bfseries, title={Problem \theproblemcounter}, arc=5mm, boxrule=0.5mm]
  \else
    \tcolorbox[breakable, colback=red!10, colframe=red!50, fonttitle=\bfseries, title={Problem \theproblemcounter: #1}, arc=5mm, boxrule=0.5mm]
    \addcontentsline{toc}{subsubsection}{\theproblemcounter: #1}%
  \fi
}{
  \endtcolorbox
}

% Define a new counter for definitions
\newcounter{definitioncounter}
\renewcommand{\thedefinitioncounter}{D.\arabic{definitioncounter}}

\newenvironment{definition}[1][]{%
  \refstepcounter{definitioncounter}%
  \if\relax\detokenize{#1}\relax
    \tcolorbox[
      breakable,
      parbox=false, % Treat content normally regarding paragraphs
      before upper={\parindent0pt \parskip7pt}, % No indentation and add space between paragraphs
      colback=blue!10,
      colframe=blue!50,
      fonttitle=\bfseries,
      title={Definition \thedefinitioncounter},
      arc=5mm,
      boxrule=0.5mm,
      before skip=10pt, % Adjust vertical space before the box
      after skip=10pt % Adjust vertical space after the box
    ]
  \else
    \tcolorbox[
      breakable,
      parbox=false,
      before upper={\parindent0pt \parskip7pt},
      colback=blue!10,
      colframe=blue!50,
      fonttitle=\bfseries,
      title={Definition \thedefinitioncounter: #1},
      arc=5mm,
      boxrule=0.5mm,
      before skip=10pt,
      after skip=10pt
    ]
  \fi
}{
  \endtcolorbox
}


% Define a new counter for theorems
\newcounter{theoremcounter}
\renewcommand{\thetheoremcounter}{T.\arabic{theoremcounter}}

\newenvironment{theorem}[1][]{%
  \refstepcounter{theoremcounter}%
  \if\relax\detokenize{#1}\relax
    \tcolorbox[
      breakable,
      parbox=false, % Treat content normally regarding paragraphs
      before upper={\parindent0pt \parskip7pt}, % No indentation and add space between paragraphs
      colback=green!10,
      colframe=green!55,
      fonttitle=\bfseries,
      title={Theorem \thetheoremcounter},
      arc=5mm,
      boxrule=0.5mm,
      before skip=10pt, % Adjust vertical space before the box
      after skip=10pt % Adjust vertical space after the box
    ]
  \else
    \tcolorbox[
      breakable,
      parbox=false,
      before upper={\parindent0pt \parskip7pt},
      colback=green!10,
      colframe=green!55,
      fonttitle=\bfseries,
      title={Theorem \thetheoremcounter: #1},
      arc=5mm,
      boxrule=0.5mm,
      before skip=10pt,
      after skip=10pt
    ]
  \fi
}{
  \endtcolorbox
}


% Define a new counter for remarks
\newcounter{remarkcounter}
\renewcommand{\theremarkcounter}{R.\arabic{remarkcounter}}

\newenvironment{remark}[1][]{%
  \refstepcounter{remarkcounter}%
  \if\relax\detokenize{#1}\relax
    \tcolorbox[
      breakable,
      parbox=false, % Treat content normally regarding paragraphs
      before upper={\parindent0pt \parskip7pt}, % No indentation and add space between paragraphs
      colback=green!10,
      colframe=green!55,
      fonttitle=\bfseries,
      title={Remark \theremarkcounter},
      arc=5mm,
      boxrule=0.5mm,
      before skip=10pt, % Adjust vertical space before the box
      after skip=10pt % Adjust vertical space after the box
    ]
  \else
    \tcolorbox[
      breakable,
      parbox=false,
      before upper={\parindent0pt \parskip7pt},
      colback=green!10,
      colframe=green!55,
      fonttitle=\bfseries,
      title={Remark \theremarkcounter: #1},
      arc=5mm,
      boxrule=0.5mm,
      before skip=10pt,
      after skip=10pt
    ]
  \fi
}{
  \endtcolorbox
}

% Define a new counter for lemmas
\newcounter{lemmacounter}
\renewcommand{\thelemmacounter}{L.\arabic{lemmacounter}}

\newenvironment{lemma}[1][]{%
  \refstepcounter{lemmacounter}%
  \if\relax\detokenize{#1}\relax
    \tcolorbox[
      breakable,
      parbox=false, % Treat content normally regarding paragraphs
      before upper={\parindent0pt \parskip7pt}, % No indentation and add space between paragraphs
      colback=green!10,
      colframe=green!55,
      fonttitle=\bfseries,
      title={Lemma \thelemmacounter},
      arc=5mm,
      boxrule=0.5mm,
      before skip=10pt, % Adjust vertical space before the box
      after skip=10pt % Adjust vertical space after the box
    ]
  \else
    \tcolorbox[
      breakable,
      parbox=false,
      before upper={\parindent0pt \parskip7pt},
      colback=green!10,
      colframe=green!55,
      fonttitle=\bfseries,
      title={Lemma \thelemmacounter: #1},
      arc=5mm,
      boxrule=0.5mm,
      before skip=10pt,
      after skip=10pt
    ]
  \fi
}{
  \endtcolorbox
}

% Define a new counter for propositions
\newcounter{propositioncounter}
\renewcommand{\thepropositioncounter}{P.\arabic{propositioncounter}}

\newenvironment{proposition}[1][]{%
  \refstepcounter{propositioncounter}%
  \if\relax\detokenize{#1}\relax
    \tcolorbox[
      breakable,
      parbox=false, % Treat content normally regarding paragraphs
      before upper={\parindent0pt \parskip7pt}, % No indentation and add space between paragraphs
      colback=green!10,
      colframe=green!55,
      fonttitle=\bfseries,
      title={Proposition \thepropositioncounter},
      arc=5mm,
      boxrule=0.5mm,
      before skip=10pt, % Adjust vertical space before the box
      after skip=10pt % Adjust vertical space after the box
    ]
  \else
    \tcolorbox[
      breakable,
      parbox=false,
      before upper={\parindent0pt \parskip7pt},
      colback=green!10,
      colframe=green!55,
      fonttitle=\bfseries,
      title={Proposition \thepropositioncounter: #1},
      arc=5mm,
      boxrule=0.5mm,
      before skip=10pt,
      after skip=10pt
    ]
  \fi
}{
  \endtcolorbox
}

%\newtheorem{proposition}[theorem]{Proposition}  % Propositions share numbering with theorems


% Define a new counter for notes
\newcounter{notescounter}
\renewcommand{\thenotescounter}{D.\arabic{notescounter}}

\newenvironment{notes}[1][]{
  \refstepcounter{notescounter}%
  \if\relax\detokenize{#1}\relax
    % If #1 is empty, set the title to "Notes"
    \tcolorbox[
      breakable,
      parbox=false, % Treat content normally regarding paragraphs
      before upper={\parindent0pt \parskip7pt}, % No indentation and add space between paragraphs
      colback=blue!10,
      colframe=blue!50,
      fonttitle=\bfseries,
      title={Notes},
      arc=5mm,
      boxrule=0.5mm,
      before skip=10pt, % Adjust vertical space before the box
      after skip=10pt % Adjust vertical space after the box
    ]
  \else
    % If #1 is not empty, use it as the title
    \tcolorbox[
      breakable,
      parbox=false,
      before upper={\parindent0pt \parskip7pt},
      colback=blue!10,
      colframe=blue!50,
      fonttitle=\bfseries,
      title={#1}, % Use provided title instead of default
      arc=5mm,
      boxrule=0.5mm,
      before skip=10pt,
      after skip=10pt
    ]
  \fi
}{
  \endtcolorbox
}







% Define a new counter for questions
\newcounter{questionscounter}
\renewcommand{\thequestionscounter}{D.\arabic{questionscounter}}

\newenvironment{questions}[1][]{
  \refstepcounter{questionscounter}%
  \if\relax\detokenize{#1}\relax
    % If #1 is empty, set the title to "Questions"
    \tcolorbox[
      breakable,
      parbox=false, % Treat content normally regarding paragraphs
      before upper={\parindent0pt \parskip7pt}, % No indentation and add space between paragraphs
      colback=red!10,
      colframe=red!50,
      fonttitle=\bfseries,
      title={Questions},
      arc=5mm,
      boxrule=0.5mm,
      before skip=10pt, % Adjust vertical space before the box
      after skip=10pt % Adjust vertical space after the box
    ]
  \else
    % If #1 is not empty, use it as the title
    \tcolorbox[
      breakable,
      parbox=false,
      before upper={\parindent0pt \parskip7pt},
      colback=red!10,
      colframe=red!50,
      fonttitle=\bfseries,
      title={#1}, % Use provided title instead of default
      arc=5mm,
      boxrule=0.5mm,
      before skip=10pt,
      after skip=10pt
    ]
  \fi
}{
  \endtcolorbox
}






% Define a new counter for overview
\newcounter{overviewcounter}
\renewcommand{\theoverviewcounter}{D.\arabic{overviewcounter}}

\newenvironment{overview}[1][]{%
  \refstepcounter{overviewcounter}%
  \if\relax\detokenize{#1}\relax
    \tcolorbox[
      breakable,
      parbox=false, % Treat content normally regarding paragraphs
      before upper={\parindent0pt \parskip7pt}, % No indentation and add space between paragraphs
      colback=green!10,
      colframe=green!55,
      fonttitle=\bfseries,
      title={Overview},
      arc=5mm,
      boxrule=0.5mm,
      before skip=10pt, % Adjust vertical space before the box
      after skip=10pt % Adjust vertical space after the box
    ]
  \else
    \tcolorbox[
      breakable,
      parbox=false,
      before upper={\parindent0pt \parskip7pt},
      colback=green!10,
      colframe=green!55,
      fonttitle=\bfseries,
      title={Overview},
      arc=5mm,
      boxrule=0.5mm,
      before skip=10pt,
      after skip=10pt
    ]
  \fi
}{
  \endtcolorbox
}



% Add a line after paragraph header
\makeatletter
\renewcommand\paragraph{\@startsection{paragraph}{4}{\z@}%
            {-3.25ex \@plus -1ex \@minus -.2ex}%
            {1.5ex \@plus .2ex}%
            {\normalfont\normalsize\bfseries}}
\makeatother


% Taking away line before and after align
\BeforeBeginEnvironment{align}{\vspace{-\parskip}}
\AfterEndEnvironment{align}{\vskip0pt plus 2pt}

\usepackage{changepage}

\title{IO1 Pset 1}

\author{Dylan Baker}
\date{October 2024}

\begin{document}
\maketitle

\section{Problem 1}

Suppose value added is a general function of TFP, labor, and capital,i.e., 
$Y=f(A, L, K)$. Assuming perfect competition and constant returns to scale, 
derive the following expression for logged TFP: $a=y-\alpha l-(1-\alpha) k$, 
where lower-case letters denote logs of their level counterparts and $\alpha$ the 
wage bill as a share of total revenues.

\hrulefill\hspace{0.5em}\dotfill\hspace{0.5em}\hrulefill

\begin{align}
    Y &= f(A, L, K) \\
    \Rightarrow d Y &= \frac{\partial f}{\partial A} d A + \frac{\partial f}{\partial L} d L + \frac{\partial f}{\partial K} d K && \text{Total differential} \\
    \Rightarrow \frac{d Y}{Y} &= \frac{\partial f}{\partial A} \frac{d A}{Y} + \frac{\partial f}{\partial L} \frac{d L}{Y} + \frac{\partial f}{\partial K} \frac{d K}{Y} && \text{Divide by $Y$} \\
    &= \frac{A}{Y} \frac{\partial f}{\partial A} \frac{d A}{A} + \frac{L}{Y} \frac{\partial f}{\partial L} \frac{d L}{L} + \frac{K}{Y} \frac{\partial f}{\partial K} \frac{d K}{K} \label{eq:io1_pset1_q1_dyy}
\end{align}

Note that a cost maximizing firm will set the 
marginal product equal to the ratio of the real input prices.
That is,

\begin{align}
    \frac{\partial F}{\partial L}=\frac{w}{P} \\
    \frac{\partial F}{\partial K}=\frac{r}{P}
\end{align}

Additionally, 

\begin{align}
    d \ln Y = \frac{d Y}{Y}
\end{align}

and the analogous equalities hold for $A$, $L$, and $K$.

Thus, we can return to \eqref{eq:io1_pset1_q1_dyy} to get:

\begin{align}
    \frac{d Y}{Y} = &\frac{A}{Y} \frac{\partial f}{\partial A} \frac{d A}{A} + \frac{L}{Y} \frac{\partial f}{\partial L} \frac{d L}{L} + \frac{K}{Y} \frac{\partial f}{\partial K} \frac{d K}{K} \\
    = &\frac{A}{Y} \frac{\partial f}{\partial A} \frac{d A}{A} + \frac{L}{Y} \frac{w}{P} \frac{d L}{L} + \frac{K}{Y} \frac{r}{P} \frac{d K}{K} \\
    \Rightarrow d \ln Y = &\frac{A}{Y} \frac{\partial f}{\partial A} d \ln A + \frac{L}{Y} \frac{w}{P} d \ln L + \frac{K}{Y} \frac{r}{P} d \ln K \\
    = & d \ln A + \frac{L}{Y} \frac{w}{P} d \ln L + \frac{K}{Y} \frac{r}{P} d \ln K && \text{since $\frac{\partial f}{\partial A} = \frac{Y}{A}$} \\
    \Rightarrow d y = & d a + \alpha d l + (1-\alpha) d k \\
    \Rightarrow d a = & d y - \alpha d l - (1-\alpha) d k && \text{rearrange} \\ 
    \Rightarrow a = & y - \alpha l - (1-\alpha) k
\end{align}


\section{Problem 2}

Do the following Monte Carlo exercise to convince yourself of the possible 
endogeneity problems in production function estimation. (Matlab, Stata's matrix 
programming language Mata, or even Excel can be used for this, though eventually 
you're going to have to move the data into a program that runs regressions.)

Construct a balanced panel dataset of 100 firms, each operating in 50 periods 
(for a total of 5000 observations). You will have to create data for firms' value 
added (output), TFP, labor and capital inputs, and real wage rates.

The firm's production function is, $y_{i t}=a_{i t}+0.7 l_{i t}+0.3 k_{i t}$, 
where all variables are expressed in logs.

Logged TFP is described by the following statistical process: 
$a_{i t}=\gamma_i+\omega_{i t}$. Here $\gamma_i$ is a firm fixed effect and 
$\omega_{i t}$ follows an $\operatorname{AR}(1)$ process: 
$\omega_{i t}=\rho \omega_{i t-1}+\varepsilon_{i t}$. Assume 
$\gamma_i \sim$ $\mathrm{N}(0,0.25)$-that is, has a standard deviation of 0.5 -across 
firms, $\rho=0.8$, and $\varepsilon_{i t}$ is i.i.d. $\mathrm{N}(0,0.01)$ across 
firms and time. The firm observes $\gamma_i$ and $\omega_{i t-1}$ at the beginning 
of period t and then chooses its inputs. After that, $\varepsilon_{i t}$ is revealed.

Assume capital is exogenous to the firm and randomly determined (obviously 
nonsensical, but it makes things easy). Specifically, $\mathrm{k}_{i t}$ is 
i.i.d. $\mathrm{N}(0,0.01)$ across firms and time. Likewise, the log real wage 
rate facing the firm is also i.i.d., with $\mathrm{w}_{i t} \sim \mathrm{~N}(0,0.25)$.

\begin{enumerate}[label=\alph*.]
    \item After constructing these exogenous variables, figure out what the firm's labor inputs are. To do so, assume the firm is a price taker in its output and labor markets. Use the expression for the profit-maximizing labor level to derive the labor demand equation, and construct the firms' labor input values according to this equation.
    \item Construct the firms' output levels from their TFP, capital, and labor values.
    \item Estimate the production function by regressing $y$ on $k$ and $l$ using OLS. Are the coefficients biased? In which direction, if so? Explain why.
    \item Now estimate the production function using firm fixed effects. Is there a bias, and how does this compare to (c)? Explain. (Just for fun, try this using random effects too.)
    \item Is there anything in the data you created that would be a suitable instrument? Explain why, if so, and then estimate the production function using IV. How do the results change?
\end{enumerate}

\subsection{Part A}

After constructing these exogenous variables, figure out what the firm's labor inputs are. To do so, assume the firm is a price taker in its output and labor markets. Use the expression for the profit-maximizing labor level to derive the labor demand equation, and construct the firms' labor input values according to this equation.

\hrulefill\hspace{0.5em}\dotfill\hspace{0.5em}\hrulefill

A given firm, in a given period, is solving:

\begin{align}
    \underset{L_{i,t}}{\text{max }} P \mathbb{E}[A_{i,t} \mid \omega_{i, t-1}] L_{i,t}^\alpha K_{i,t}^{1-\alpha}-W L_{i,t}
\end{align}

which gives us:

\begin{align}
    \frac{\partial \pi}{\partial L} &= P \mathbb{E}[A_{i,t} \mid \omega_{i, t-1}] \alpha L_{i,t}^{\alpha-1} K_{i,t}^{1-\alpha} - W = 0 \\
    \Rightarrow L_{i,t}^* &= \left(\frac{\alpha \mathbb{E}[A_{i,t} \mid \omega_{i, t-1}]}{W / P}\right)^{\frac{1}{1-\alpha}} K_{i,t}
\end{align}

Switching to logs and plugging in $\alpha = 0.7$ we get:

\begin{align}
    l_{i t}^*=k_{i t}+\frac{10}{3}\left(\ln (0.7)+\gamma_i+\rho \omega_{i t-1}-w_{i t}\right)
\end{align}


\subsection{Part B}

Construct the firms' output levels from their TFP, capital, and labor values.

\hrulefill\hspace{0.5em}\dotfill\hspace{0.5em}\hrulefill

See code.

\subsection{Part C \& D}

Part C: Estimate the production function by regressing $y$ on $k$ and $l$ using OLS. Are the coefficients biased? In which direction, if so? Explain why.

Part D: Now estimate the production function using firm fixed effects. Is there a bias, and how does this compare to (c)? Explain. (Just for fun, try this using random effects too.)

\hrulefill\hspace{0.5em}\dotfill\hspace{0.5em}\hrulefill

\FloatBarrier

\begingroup
\centering
\begin{tabular}{lcc}
   \tabularnewline \midrule \midrule
   Dependent Variable: & \multicolumn{2}{c}{y}\\
   Model:       & (1)            & (2)\\  
   \midrule
   \emph{Variables}\\
   Constant     & 0.1929$^{***}$ &   \\   
                & (0.0056)       &   \\   
   l            & 0.8408$^{***}$ & 0.7125$^{***}$\\   
                & (0.0022)       & (0.0014)\\   
   k            & 0.1710$^{***}$ & 0.2992$^{***}$\\   
                & (0.0503)       & (0.0187)\\   
   \midrule
   \emph{Fixed-effects}\\
   firm\_id     &                & Yes\\  
   \midrule
   \emph{Fit statistics}\\
   Observations & 5,000          & 5,000\\  
   R$^2$        & 0.96675        & 0.99472\\  
   Within R$^2$ &                & 0.98627\\  
   \midrule \midrule
   \multicolumn{3}{l}{\emph{Signif. Codes: ***: 0.01, **: 0.05, *: 0.1}}\\
\end{tabular}
\par\endgroup


The estimate for l is biased up in each case, though most so for 
the baseline OLS estimate. The estimate for k is biased down in each case,
though, again, most so for the baseline OLS estimate. The random effects 
version is between the two regressions presented in the table here.

Here, we are suffering from omitted variable bias, since $l_{it}$
is correlated with the error term, since it's a function of $\omega_{it-1}$
and $\gamma_i$. By including fixed effects, we are able to partially address this
by accounting for the permanent piece of firm productivity.


\FloatBarrier

\subsection{Part E}

Is there anything in the data you created that would be a suitable instrument? Explain why, if so, and then estimate the production function using IV. How do the results change?

\hrulefill\hspace{0.5em}\dotfill\hspace{0.5em}\hrulefill

We can use log real wage as an instrument. If we do so, 
we get the below results:

\begingroup
\centering
\begin{tabular}{lc}
   \tabularnewline \midrule \midrule
   Dependent Variable: & y\\  
   Model:              & (1)\\  
   \midrule
   \emph{Variables}\\
   Constant            & 0.0439$^{***}$\\   
                       & (0.0081)\\   
   l                   & 0.6987$^{***}$\\   
                       & (0.0041)\\   
   k                   & 0.3078$^{***}$\\   
                       & (0.0681)\\   
   \midrule
   \emph{Fit statistics}\\
   Observations        & 5,000\\  
   R$^2$               & 0.93924\\  
   Adjusted R$^2$      & 0.93921\\  
   \midrule \midrule
   \multicolumn{2}{l}{\emph{IID standard-errors in parentheses}}\\
   \multicolumn{2}{l}{\emph{Signif. Codes: ***: 0.01, **: 0.05, *: 0.1}}\\
\end{tabular}
\par\endgroup


The estimates are still slightly off, but they are quite close.


%%%%%%%%%%%%%%%%%%%%%%%%%%%%%%%%%%%%%%%%%%%%%%%%%%%%%%%%%%%%%%%%%%%%%%%%%%%%%%%%%%%%%%%
%%%%%%%%%%%%%%%%%%%%%%%%%%%%%%%%%%%%%%%%%%%%%%%%%%%%%%%%%%%%%%%%%%%%%%%%%%%%%%%%%%%%%%%
%%%%%%%%%%%%%%%%%%%%%%%%%%%%%%%%%%%%%%%%%%%%%%%%%%%%%%%%%%%%%%%%%%%%%%%%%%%%%%%%%%%%%%%

\section{Problem 3}

A simple model of equilibrium industry productivity and output heterogeneity.

Suppose that firms in an industry have a variable profit function $\pi\left(\phi_i, \mu, \sigma\right)$, where $\phi_i$ is the firm's productivity level, and $\sigma$ is the elasticity of substitution between the output of industry firms. $\mu$ is a value that indexes the intensity of industry competition (like $s$ in the Ericson-Pakes model); assume that it is an increasing function of both the average productivity level and mass of industry producers. Assume that $\pi_\phi(\cdot)>0$ and $\pi_\mu(\cdot)<0$; i.e., being more productive increases profits, and stiffer competition decreases profits. There is also a fixed cost of production, $f$, so total operating profits are $\Pi\left(\phi_i, \mu, \sigma, f\right)=\pi\left(\phi_i, \mu, \sigma\right)-f$.

Entry is a simple two stage game. In the first stage, a very large number of ex-ante identical potential entrants decide whether to enter into the industry. If they choose to do so, they pay a sunk entry cost, $s$, and learn their own productivity draw (taken from a common distribution $G(\phi)$ over $\left.\left[\phi, \phi^{\prime}\right]\right)$ as well as those of all other firms who paid the entry cost. Upon learning these productivity draws, they simultaneously decide whether or not to produce in the second stage, and earn operating profits as specified above.

Two conditions hold in equilibrium. No firm produces at a loss (they can always choose not to produce and earn zero profits). Second, there is free entry: in equilibrium, the expected value of entry is equal to $s$.

\begin{enumerate}[label=\alph*.]
    \item Use the first equilibrium condition and the firm's operating profit function to argue that there is a unique cutoff productivity level, $\phi^*$, where firms with productivity draws $\phi_i<\phi^*$ do not produce. Assume that $G(\phi)$ has a large enough domain so that $\pi\left(\varphi^l, \mu, \sigma\right)<f$ and $\pi\left(\varphi^u, \mu, \sigma\right)>f$ for all $\mu$ and $\sigma$. Use the unique cutoff result to write a zero-marginal-profit equation that embodies the first equilibrium condition.
    \item Derive, using this equilibrium equation, the condition under which $d \phi^* / d f>0$; that is, higher fixed production costs raise the efficiency bar for profitable operation. Give the intuition behind the condition and this result. (Note that the average productivity level in the industry is a function of $\phi^*$.)
    \item Write down an equation that embodies the second (free-entry) 
        equilibrium condition. (Remember that the expected value of entry is equal 
        to the probability of successful entry multiplied by average profits 
        conditional upon successful entry.) Use it to show that $d \phi^* / d s<0$; 
        i.e., high entry costs imply it is easier for inefficient firms to operate 
        profitably. (Hint: use the zero-cutoff-profit equation to substitute in for 
        the fixed cost in the free-entry equation before doing the comparative statics.) Why does an increase in the lump-sum sunk cost have an effect opposite that of an increase in the lump-sum fixed production cost?
    \item Show the condition on the profit function required so that $d \phi^* / d \sigma>0$, i.e., an increase in substitutability makes it more difficult for inefficient producers to survive. Give the intuition for the derived condition. Do you find it intuitively plausible?
    \item How would you empirically test these implications?
\end{enumerate}

\subsection{Part A}

Use the first equilibrium condition and the firm's operating profit function to argue that there is a unique cutoff productivity level, $\phi^*$, where firms with productivity draws $\phi_i<\phi^*$ do not produce. Assume that $G(\phi)$ has a large enough domain so that $\pi\left(\varphi^l, \mu, \sigma\right)<f$ and $\pi\left(\varphi^u, \mu, \sigma\right)>f$ for all $\mu$ and $\sigma$. Use the unique cutoff result to write a zero-marginal-profit equation that embodies the first equilibrium condition.

\hrulefill\hspace{0.5em}\dotfill\hspace{0.5em}\hrulefill

A firm will produce if

\begin{align}
    \pi\left(\phi_i, \mu, \sigma\right) > f
\end{align}

Given that for any fixed $\mu$ and $\sigma$, there is some 
$\underline{\phi}$ such that $\pi\left(\underline{\phi}, \mu, \sigma\right) < f$ and
some $\overline{\phi}$ such that $\pi\left(\overline{\phi}, \mu, \sigma\right) > f$.
With this established, the desired result follows 
from the fact that $\pi_{\phi}(\cdot)>0$.\footnote{I'm 
taking $\pi$ to be continuous in $\phi$.}

\subsection{Part B}

Derive, using this equilibrium equation, the condition under which $d \phi^* / d f>0$; that is, higher fixed production costs raise the efficiency bar for profitable operation. Give the intuition behind the condition and this result. (Note that the average productivity level in the industry is a function of $\phi^*$.)

\hrulefill\hspace{0.5em}\dotfill\hspace{0.5em}\hrulefill

Use total differentiation:

\begin{align}
    &\left(\frac{d \pi\left(\phi^*, \mu\left(\phi^*\right)\right)}{d \phi^*}+\frac{d \pi\left(\phi^*, \mu\left(\phi^*\right)\right)}{d \mu\left(\phi^*\right)} \frac{d \mu\left(\phi^*\right)}{d \phi^*}\right) d \phi^*-d f=0 \\
    \Leftrightarrow &\frac{d \phi^*}{d f}=\frac{1}{\frac{d \pi\left(\phi^*, \mu\left(\phi^*\right)\right)}{d \phi^*}+\frac{d \pi\left(\phi^*, \mu\left(\phi^*\right)\right)}{d \mu\left(\phi^*\right)} \frac{d \mu\left(\phi^*\right)}{d \phi^*}}
\end{align}

Given 

\begin{align}
    \frac{d \pi\left(\phi^*, \mu\left(\phi^*\right)\right)}{d \mu\left(\phi^*\right)}<0 \\
    \frac{d \pi\left(\phi^*, \mu\left(\phi^*\right)\right)}{d \phi^*}>0
\end{align}

we know that

\begin{align}
    \frac{d \phi^*}{d f}>0 \Rightarrow \frac{d \mu\left(\phi^*\right)}{d \phi^*}>0
\end{align}

and 

\begin{align}
    \frac{d \pi\left(\phi^*, \mu\left(\phi^*\right)\right)}{d \phi^*} \frac{d \mu\left(\phi^*\right)}{d \phi^*}>-\frac{d \pi\left(\phi^*, \mu\left(\phi^*\right)\right)}{d \mu\left(\phi^*\right)}
\end{align}


The intuition behind this condition is that we are 
looking at the difference in the rising profit 
of a productive firm against the falling profit
of a firm in a more competitive industry. 


\subsection{Part C}

Write down an equation that embodies the second (free-entry) 
equilibrium condition. (Remember that the expected value of entry is equal 
to the probability of successful entry multiplied by average profits 
conditional upon successful entry.) Use it to show that $d \phi^* / d s<0$; 
i.e., high entry costs imply it is easier for inefficient firms to operate 
profitably. (Hint: use the zero-cutoff-profit equation to substitute in for 
the fixed cost in the free-entry equation before doing the comparative statics.) Why does an increase in the lump-sum sunk cost have an effect opposite that of an increase in the lump-sum fixed production cost?

\hrulefill\hspace{0.5em}\dotfill\hspace{0.5em}\hrulefill

Using $g$ to be the density of $G$,
the free entry condition is given by:

\begin{align}
    \int_{\phi^*}^{\phi^u}(\pi(\phi, \mu, \sigma)-f) g(\phi) d \phi-s=0
\end{align}

Then we can write:

\begin{align}
    &\left[-\left[\pi\left(\phi^*, \mu, \sigma\right)-f\right] g\left(\phi^*\right)+\int_{\phi^*}^{\phi^*} \pi_\mu(\phi) \frac{\partial \mu}{\partial \phi^*} g(\phi) d \phi\right] d \phi^*-d s=0 && \parbox[t]{4cm}{\raggedright total differentiation} \\
    \Rightarrow & \left\{\int_{\phi^*}^{\phi^u} \pi_\mu(\phi) \frac{\partial \mu}{\partial \phi^*} g(\phi) d \phi\right\} d \phi^*=d s && \parbox[t]{4cm}{\raggedright since $\pi\left(\phi^*, \mu, \sigma\right)-f=0$} \\
    \Rightarrow &\frac{d \phi^*}{d s}=\frac{1}{\left\{\int_{\phi^*}^{\phi^u} \pi_\mu(\phi) \frac{\partial \mu}{\partial \phi^*} g(\phi) d \phi\right\}}
\end{align}

Noting $\pi_\mu(\cdot)<0$ and $\frac{\partial \mu}{\partial \phi^*}>0, \frac{d \phi^*}{d s}<0$.


An increase in $f$ corresponds to a relative increase in the 
difficulty of production for less productive firms. If $s$ 
is increase, meanwhile, it is relatively easier for 
less productive firms to produce, since the fixed cost
of discovering productivity is raised. 

\subsection{Part D}

Show the condition on the profit function required so that $d \phi^* / d \sigma>0$, i.e., an increase in substitutability makes it more difficult for inefficient producers to survive. Give the intuition for the derived condition. Do you find it intuitively plausible?

\hrulefill\hspace{0.5em}\dotfill\hspace{0.5em}\hrulefill

We have

\begin{align}
    \int_{\phi^*}^{\phi^u}\left(\pi(\phi, \mu, \sigma)-\pi\left(\phi^*, \mu, \sigma\right)\right) g(\phi) d \phi-s=0
\end{align}

from 

\begin{align}
    \pi\left(\phi^*, \mu, \sigma\right)=f
\end{align}

Using total differentiation and the zero-marginal-profit equation, we get:

\begin{align}
    \left\{\int_{\phi^*}^{\phi^\mu}\left[\pi_\mu(\phi) \frac{\partial \mu}{\partial \phi^*}-\pi_\phi\left(\phi^*\right)-\pi_\mu\left(\phi^*\right) \frac{\partial \mu}{\partial \phi^*}\right] g(\phi) d \phi\right\} d \phi^*+\left\{\int_{\phi^*}^{\phi^u}\left[\pi_\sigma(\phi)-\pi_\sigma\left(\phi^*\right)\right] g(\phi) d \phi\right\} d \sigma=0
\end{align}

Thus, 

\begin{align}
    \frac{d \phi^*}{d \sigma}=-\frac{\int_{\phi^*}^{\phi^u}\left[\pi_\sigma(\phi)-\pi_\sigma\left(\phi^*\right)\right] g(\phi) d \phi}{\int_{\phi^*}^{\phi^\mu}\left[\pi_\mu(\phi) \frac{\partial \mu}{\partial \phi^*}-\pi_\phi\left(\phi^*\right)-\pi_\mu\left(\phi^*\right) \frac{\partial \mu}{\partial \phi^*}\right] g(\phi) d \phi}
\end{align}

The denominator is positive by earlier parts 
and $\frac{d \phi^*}{d \sigma}>0$. Thus, 

\begin{align}
    \int_{\phi^*}^{\phi^u}\left[\pi_\sigma(\phi)-\pi_\sigma\left(\phi^*\right)\right] g(\phi) d \phi< 0
\end{align}

If $\sigma$ changes and elicits a positive effect on $\phi$ such that $\phi > \phi^*$, then 
it becomes increasingly challenging for 
less productive firms to survive, moving the cutoff. 



\subsection{Part E}

How would you empirically test these implications?

\hrulefill\hspace{0.5em}\dotfill\hspace{0.5em}\hrulefill

We are looking for exogenous variation in 
$f, s$, and $\sigma$. These are challenging given the clear 
endogeneity issues accompanying them. In principle, 
we'd be interested in something like valid instruments or 
natural experiments lending themselves 
to design-based identification strategies. 
However, this is challenging in its own right.




\end{document}