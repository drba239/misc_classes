\documentclass[12pt]{article}
\usepackage{amsmath}
\usepackage{amsthm}
\usepackage{amsfonts}
\usepackage{amssymb}
\usepackage{amssymb}
\usepackage{booktabs}
\setlength\parindent{0pt}
\usepackage[margin=1.2in]{geometry}
\usepackage{enumitem}
\usepackage{mathtools}
\mathtoolsset{showonlyrefs=true}
\usepackage{pdflscape}
\usepackage{xcolor}
\usepackage{hyperref}
\setcounter{tocdepth}{4}
\setcounter{secnumdepth}{4}
\usepackage[listings,skins,breakable]{tcolorbox} % package for colored boxes
\usepackage{etoolbox}
\usepackage{placeins}
\usepackage{tikz}
\usepackage{color}  % Allows for color customization
\usepackage{subcaption}
\usepackage[utf8]{inputenc}


% Make it so that the bottom page of a 
% book section doesn't have weird spacing
\raggedbottom


% Define custom colors
% You can-redo these later, they're not being used 
% for anything as of 5/12/24
\definecolor{codegreen}{rgb}{0,0.6,0}
\definecolor{codegray}{rgb}{0.5,0.5,0.5}
\definecolor{codepurple}{rgb}{0.58,0,0.82}
\definecolor{backcolour}{rgb}{0.95,0.95,0.92}

% You can-redo the lstlisting style later, it's not being used 
% for anything as of 5/12/24

% Define the lstlisting style
\lstdefinestyle{mystyle}{
    backgroundcolor=\color{backcolour},   
    commentstyle=\color{codegreen},
    keywordstyle=\color{magenta},
    numberstyle=\tiny\color{codegray},
    stringstyle=\color{codepurple},
    basicstyle=\ttfamily\footnotesize,
    breakatwhitespace=false,         
    breaklines=true,                 
    captionpos=b,                    
    keepspaces=true,                 
    numbers=left,                    
    numbersep=5pt,                  
    showspaces=false,                
    showstringspaces=false,
    showtabs=false,                  
    tabsize=2
}
\lstset{style=mystyle}


% Set the length of \parskip to add a line between paragraphs
\setlength{\parskip}{1em}


% Set the second level of itemize to use \circ as the bullet point
\setlist[itemize,2]{label={$\circ$}}

% Define symbols
\DeclareMathSymbol{\Perp}{\mathrel}{symbols}{"3F}
\newcommand\barbelow[1]{\stackunder[1.2pt]{$#1$}{\rule{.8ex}{.075ex}}}
\newcommand{\succprec}{\mathrel{\mathpalette\succ@prec{\succ\prec}}}
\newcommand{\precsucc}{\mathrel{\mathpalette\succ@prec{\prec\succ}}}


\newcounter{example}[section] % Reset example counter at each new section
\renewcommand{\theexample}{\thesection.\arabic{example}} % Format the example number as section.number

\newenvironment{example}
  {% Begin environment
   \refstepcounter{example}% Step counter and allow for labeling
   \noindent\textbf{Example \theexample.} % Display the example number
  }
  {% End environment
   \par\noindent\hfill\textit{End of Example.}\par
  }



% deeper section command
% This will let you go one level deeper than whatever section level you're on.
\makeatletter
\newcommand{\deepersection}[1]{%
  \ifnum\value{subparagraph}>0
    % Already at the deepest standard level (\subparagraph), cannot go deeper
    \subparagraph{#1}
  \else
    \ifnum\value{paragraph}>0
      \subparagraph{#1}
    \else
      \ifnum\value{subsubsection}>0
        \paragraph{#1}
      \else
        \ifnum\value{subsection}>0
          \subsubsection{#1}
        \else
          \ifnum\value{section}>0
            \subsection{#1}
          \else
            \section{#1}
          \fi
        \fi
      \fi
    \fi
  \fi
}
\makeatother


% same section command
% This will let create a section at the same level as whatever section level you're on.
\makeatletter
\newcommand{\samesection}[1]{%
  \ifnum\value{subparagraph}>0
    \subparagraph{#1}
  \else
    \ifnum\value{paragraph}>0
      \paragraph{#1}
    \else
      \ifnum\value{subsubsection}>0
        \subsubsection{#1}
      \else
        \ifnum\value{subsection}>0
          \subsection{#1}
        \else
          \ifnum\value{section}>0
            \section{#1}
          \else
            % Default to section if outside any sectioning
            \section{#1}
          \fi
        \fi
      \fi
    \fi
  \fi
}
\makeatother

\makeatletter
\newcommand{\shallowersection}[1]{%
  \ifnum\value{subparagraph}>0
    \paragraph{#1} % From subparagraph to paragraph
  \else
    \ifnum\value{paragraph}>0
      \subsubsection{#1} % From paragraph to subsubsection
    \else
      \ifnum\value{subsubsection}>0
        \subsection{#1} % From subsubsection to subsection
      \else
        \ifnum\value{subsection}>0
          \section{#1} % From subsection to section
        \else
          \ifnum\value{section}>0
            \chapter{#1} % Assuming a document class with chapters
          \else
            \section{#1} % Default to section if somehow higher than section
          \fi
        \fi
      \fi
    \fi
  \fi
}
\makeatother



\newcounter{problemcounter}
\renewcommand{\theproblemcounter}{Q.\arabic{problemcounter}}

% Define the problem environment
\newenvironment{problem}[1][]{%
  \refstepcounter{problemcounter}%
  \if\relax\detokenize{#1}\relax
    \tcolorbox[breakable, colback=red!10, colframe=red!50, fonttitle=\bfseries, title={Problem \theproblemcounter}, arc=5mm, boxrule=0.5mm]
  \else
    \tcolorbox[breakable, colback=red!10, colframe=red!50, fonttitle=\bfseries, title={Problem \theproblemcounter: #1}, arc=5mm, boxrule=0.5mm]
    \addcontentsline{toc}{subsubsection}{\theproblemcounter: #1}%
  \fi
}{
  \endtcolorbox
}

% Define a new counter for definitions
\newcounter{definitioncounter}
\renewcommand{\thedefinitioncounter}{D.\arabic{definitioncounter}}

\newenvironment{definition}[1][]{%
  \refstepcounter{definitioncounter}%
  \if\relax\detokenize{#1}\relax
    \tcolorbox[
      breakable,
      parbox=false, % Treat content normally regarding paragraphs
      before upper={\parindent0pt \parskip7pt}, % No indentation and add space between paragraphs
      colback=blue!10,
      colframe=blue!50,
      fonttitle=\bfseries,
      title={Definition \thedefinitioncounter},
      arc=5mm,
      boxrule=0.5mm,
      before skip=10pt, % Adjust vertical space before the box
      after skip=10pt % Adjust vertical space after the box
    ]
  \else
    \tcolorbox[
      breakable,
      parbox=false,
      before upper={\parindent0pt \parskip7pt},
      colback=blue!10,
      colframe=blue!50,
      fonttitle=\bfseries,
      title={Definition \thedefinitioncounter: #1},
      arc=5mm,
      boxrule=0.5mm,
      before skip=10pt,
      after skip=10pt
    ]
  \fi
}{
  \endtcolorbox
}


% Define a new counter for theorems
\newcounter{theoremcounter}
\renewcommand{\thetheoremcounter}{T.\arabic{theoremcounter}}

\newenvironment{theorem}[1][]{%
  \refstepcounter{theoremcounter}%
  \if\relax\detokenize{#1}\relax
    \tcolorbox[
      breakable,
      parbox=false, % Treat content normally regarding paragraphs
      before upper={\parindent0pt \parskip7pt}, % No indentation and add space between paragraphs
      colback=green!10,
      colframe=green!55,
      fonttitle=\bfseries,
      title={Theorem \thetheoremcounter},
      arc=5mm,
      boxrule=0.5mm,
      before skip=10pt, % Adjust vertical space before the box
      after skip=10pt % Adjust vertical space after the box
    ]
  \else
    \tcolorbox[
      breakable,
      parbox=false,
      before upper={\parindent0pt \parskip7pt},
      colback=green!10,
      colframe=green!55,
      fonttitle=\bfseries,
      title={Theorem \thetheoremcounter: #1},
      arc=5mm,
      boxrule=0.5mm,
      before skip=10pt,
      after skip=10pt
    ]
  \fi
}{
  \endtcolorbox
}


% Define a new counter for remarks
\newcounter{remarkcounter}
\renewcommand{\theremarkcounter}{R.\arabic{remarkcounter}}

\newenvironment{remark}[1][]{%
  \refstepcounter{remarkcounter}%
  \if\relax\detokenize{#1}\relax
    \tcolorbox[
      breakable,
      parbox=false, % Treat content normally regarding paragraphs
      before upper={\parindent0pt \parskip7pt}, % No indentation and add space between paragraphs
      colback=green!10,
      colframe=green!55,
      fonttitle=\bfseries,
      title={Remark \theremarkcounter},
      arc=5mm,
      boxrule=0.5mm,
      before skip=10pt, % Adjust vertical space before the box
      after skip=10pt % Adjust vertical space after the box
    ]
  \else
    \tcolorbox[
      breakable,
      parbox=false,
      before upper={\parindent0pt \parskip7pt},
      colback=green!10,
      colframe=green!55,
      fonttitle=\bfseries,
      title={Remark \theremarkcounter: #1},
      arc=5mm,
      boxrule=0.5mm,
      before skip=10pt,
      after skip=10pt
    ]
  \fi
}{
  \endtcolorbox
}

% Define a new counter for lemmas
\newcounter{lemmacounter}
\renewcommand{\thelemmacounter}{L.\arabic{lemmacounter}}

\newenvironment{lemma}[1][]{%
  \refstepcounter{lemmacounter}%
  \if\relax\detokenize{#1}\relax
    \tcolorbox[
      breakable,
      parbox=false, % Treat content normally regarding paragraphs
      before upper={\parindent0pt \parskip7pt}, % No indentation and add space between paragraphs
      colback=green!10,
      colframe=green!55,
      fonttitle=\bfseries,
      title={Lemma \thelemmacounter},
      arc=5mm,
      boxrule=0.5mm,
      before skip=10pt, % Adjust vertical space before the box
      after skip=10pt % Adjust vertical space after the box
    ]
  \else
    \tcolorbox[
      breakable,
      parbox=false,
      before upper={\parindent0pt \parskip7pt},
      colback=green!10,
      colframe=green!55,
      fonttitle=\bfseries,
      title={Lemma \thelemmacounter: #1},
      arc=5mm,
      boxrule=0.5mm,
      before skip=10pt,
      after skip=10pt
    ]
  \fi
}{
  \endtcolorbox
}

% Define a new counter for propositions
\newcounter{propositioncounter}
\renewcommand{\thepropositioncounter}{P.\arabic{propositioncounter}}

\newenvironment{proposition}[1][]{%
  \refstepcounter{propositioncounter}%
  \if\relax\detokenize{#1}\relax
    \tcolorbox[
      breakable,
      parbox=false, % Treat content normally regarding paragraphs
      before upper={\parindent0pt \parskip7pt}, % No indentation and add space between paragraphs
      colback=green!10,
      colframe=green!55,
      fonttitle=\bfseries,
      title={Proposition \thepropositioncounter},
      arc=5mm,
      boxrule=0.5mm,
      before skip=10pt, % Adjust vertical space before the box
      after skip=10pt % Adjust vertical space after the box
    ]
  \else
    \tcolorbox[
      breakable,
      parbox=false,
      before upper={\parindent0pt \parskip7pt},
      colback=green!10,
      colframe=green!55,
      fonttitle=\bfseries,
      title={Proposition \thepropositioncounter: #1},
      arc=5mm,
      boxrule=0.5mm,
      before skip=10pt,
      after skip=10pt
    ]
  \fi
}{
  \endtcolorbox
}

%\newtheorem{proposition}[theorem]{Proposition}  % Propositions share numbering with theorems


% Define a new counter for notes
\newcounter{notescounter}
\renewcommand{\thenotescounter}{D.\arabic{notescounter}}

\newenvironment{notes}[1][]{
  \refstepcounter{notescounter}%
  \if\relax\detokenize{#1}\relax
    % If #1 is empty, set the title to "Notes"
    \tcolorbox[
      breakable,
      parbox=false, % Treat content normally regarding paragraphs
      before upper={\parindent0pt \parskip7pt}, % No indentation and add space between paragraphs
      colback=blue!10,
      colframe=blue!50,
      fonttitle=\bfseries,
      title={Notes},
      arc=5mm,
      boxrule=0.5mm,
      before skip=10pt, % Adjust vertical space before the box
      after skip=10pt % Adjust vertical space after the box
    ]
  \else
    % If #1 is not empty, use it as the title
    \tcolorbox[
      breakable,
      parbox=false,
      before upper={\parindent0pt \parskip7pt},
      colback=blue!10,
      colframe=blue!50,
      fonttitle=\bfseries,
      title={#1}, % Use provided title instead of default
      arc=5mm,
      boxrule=0.5mm,
      before skip=10pt,
      after skip=10pt
    ]
  \fi
}{
  \endtcolorbox
}







% Define a new counter for questions
\newcounter{questionscounter}
\renewcommand{\thequestionscounter}{D.\arabic{questionscounter}}

\newenvironment{questions}[1][]{
  \refstepcounter{questionscounter}%
  \if\relax\detokenize{#1}\relax
    % If #1 is empty, set the title to "Questions"
    \tcolorbox[
      breakable,
      parbox=false, % Treat content normally regarding paragraphs
      before upper={\parindent0pt \parskip7pt}, % No indentation and add space between paragraphs
      colback=red!10,
      colframe=red!50,
      fonttitle=\bfseries,
      title={Questions},
      arc=5mm,
      boxrule=0.5mm,
      before skip=10pt, % Adjust vertical space before the box
      after skip=10pt % Adjust vertical space after the box
    ]
  \else
    % If #1 is not empty, use it as the title
    \tcolorbox[
      breakable,
      parbox=false,
      before upper={\parindent0pt \parskip7pt},
      colback=red!10,
      colframe=red!50,
      fonttitle=\bfseries,
      title={#1}, % Use provided title instead of default
      arc=5mm,
      boxrule=0.5mm,
      before skip=10pt,
      after skip=10pt
    ]
  \fi
}{
  \endtcolorbox
}






% Define a new counter for overview
\newcounter{overviewcounter}
\renewcommand{\theoverviewcounter}{D.\arabic{overviewcounter}}

\newenvironment{overview}[1][]{%
  \refstepcounter{overviewcounter}%
  \if\relax\detokenize{#1}\relax
    \tcolorbox[
      breakable,
      parbox=false, % Treat content normally regarding paragraphs
      before upper={\parindent0pt \parskip7pt}, % No indentation and add space between paragraphs
      colback=green!10,
      colframe=green!55,
      fonttitle=\bfseries,
      title={Overview},
      arc=5mm,
      boxrule=0.5mm,
      before skip=10pt, % Adjust vertical space before the box
      after skip=10pt % Adjust vertical space after the box
    ]
  \else
    \tcolorbox[
      breakable,
      parbox=false,
      before upper={\parindent0pt \parskip7pt},
      colback=green!10,
      colframe=green!55,
      fonttitle=\bfseries,
      title={Overview},
      arc=5mm,
      boxrule=0.5mm,
      before skip=10pt,
      after skip=10pt
    ]
  \fi
}{
  \endtcolorbox
}



% Add a line after paragraph header
\makeatletter
\renewcommand\paragraph{\@startsection{paragraph}{4}{\z@}%
            {-3.25ex \@plus -1ex \@minus -.2ex}%
            {1.5ex \@plus .2ex}%
            {\normalfont\normalsize\bfseries}}
\makeatother


% Taking away line before and after align
\BeforeBeginEnvironment{align}{\vspace{-\parskip}}
\AfterEndEnvironment{align}{\vskip0pt plus 2pt}

\usepackage{changepage}
\usepackage{titlesec}

% Adjust spacing for sections
\titlespacing*{\section}{0pt}{1.5ex plus 1ex minus .2ex}{0.5ex plus .1ex}
\titlespacing*{\subsection}{0pt}{1.5ex plus 1ex minus .2ex}{0.5ex plus .1ex}

% Draft toggle
\newif\ifdraft
%\drafttrue % Set to \drafttrue for draft mode, \draftfalse for final mode
\draftfalse

\ifdraft
\else
\usepackage[style=apa, backend=biber]{biblatex}
\addbibresource{references.bib}
\fi

\title{Project Proposal}

\begin{document}

\begin{titlepage}
    \pagestyle{empty} % Ensures no page numbers for the entire page
    \centering
    \vspace*{\fill}
    {\Huge\bfseries Referee Report\par}
    \vspace{1.5cm}
    {\Large Dylan Baker\par}
    \vspace{1.5cm}
    {\large Econ 401\par}
    \vfill
    \date{\today}
\end{titlepage}

\newpage 

\pagenumbering{arabic}

\section{Summary}

This paper examines the effect of investors owning 
shares in multiple horizontal competitors, i.e., ``common ownership,''
on firm productivity. Towards answering this question, the authors constructed a novel
dataset on ownership and control structure
among public US firms. The baseline analysis in the paper consisted 
of a panel regression of firm productivity on common ownership 
(operationalized in the main analysis using profit weights), as well as 
a number of controls (most notably industry-by-year fixed effects); the paper 
also includes a second version adding a squared term for common ownership
to accommodate non-linearities:
\vspace{-1em}

\begin{align}
    \widetilde{\omega}_{f t}&=\beta_1 * \kappa_{f t}+\Phi * X_{f t}+\alpha_{s t} \\
    \widetilde{\omega}_{f t}&=\beta_1 * \kappa_{f t}+\beta_2 * \kappa_{f t}^2+\Phi * X_{f t s}+\alpha_{s t}
\end{align}
\vspace{-1em}

where $\widetilde{\omega}_{f t}$ is estimated ``revenue productivity,''
$\kappa_{f t}$ is the common-ownership measure, $X_{f t}$ is a vector of controls,
and $\alpha_{s t}$ is industry-by-year FEs.
The authors find a negative relationship between common ownership and firm productivity.
The authors also re-estimate the model after adding 
in a variable for managerial incentives (operationalized with 
wealth-performance sensitivity) and find that managerial incentives
account for some, but not all, of the relationship 
between common ownership and productivity.

To further address endogeneity concerns, the authors then turn to 
a diff-in-diff strategy. They use the addition of a firm to the S\&P 500 index
as a plausibly exogenous shock to the common 
ownership incentives of index incumbent competitors and estimate the below model:
\vspace{-1em}

\begin{align}
    \widetilde{\omega}_{f t}&=\text { Post }_{f t} * \text { Treat }_{f t}+\text { Post }_{f t}+\text { Treat }_{f t}+\Phi_1 X_{f t}+\Phi_2 \text { Post }_{f t} * X_{f t}+\alpha_t+\alpha_s
\end{align}
\vspace{-1.5em}

where the terms are either clear or follow from the previous model.
They find a negative, but statistically insignificant, effect of 
index additions on firm productivity. 

Finally, they employ the strategy of using index additions as an instrument 
for variation in managerial incentives with the outcome of interest 
remaining firm (revenue) productivity. They find a strong first-stage 
in which 
the index addition of a competitor leads to a 17\% decline in 
managerial wealth-performance sensitivity. They estimate the following:

\begin{align}
    \begin{gathered}
        \text { wps }_{f t}=\text { Post }_{f t} * \text { Treat }_{f t}+\text { Post }_{f t}+\text { Treat}_{f t}+\Phi X_{f t}+\text { Treat }_{ft} * X_{f t}+\alpha_t+\alpha_s \\
        \widetilde{\omega}_{f t}=\widehat{w p s}_{f t}+\Phi X_{f t}+\alpha_t+\alpha_s
    \end{gathered}
\end{align}

They find a significant positive effect of the instrumented variation in 
WPS on firm productivity, 
which corresponds to a negative effect of common ownership on firm productivity.


\section{Commentary}

This paper makes several contributions to the literature on 
the effects of common ownership. First, the creation of the novel dataset 
on ownership and control structure among public US firms 
is a significant contribution in and of itself. In contrast to past datasets, 
this dataset includes not only institutional investors' holdings, but 
also large individual shareholders and corporate insiders. Moreover, the paper 
aims to bring empirical evidence to bear on theoretical debates.
Although there has certainly been work on the effects of 
common ownership 
\ifdraft(bas2023, anton2024, lopez2019) \else\parencite{bas2023, anton2024, lopez2019}\fi,
there has not been work on assessing the effect of common ownership on 
firm productivity, despite theory-based arguments that common ownership
may influence firm productivity via changes to managerial incentives
\ifdraft(anton2023) \else\parencite{anton2023}\fi.

Beginning with the panel data analysis, I would be interesting in 
seeing some discussion of contending with the challenges 
stemming from unbalanced panel data, which the authors seem to have 
\ifdraft(baltagi2021) \else\parencite{baltagi2021}\fi.

It appears that the authors ran a traditional 
staggered diff-in-diff. It seems it would be sensible to 
employ -- at least as robustness checks -- 
some of the more recent advances in the diff-in-diff literature
discussed in \ifdraft(baker2022) \else\textcite{baker2022}\fi.




The final analysis of the paper employing the index additions as an instrument
for managerial incentives currently reads as under-justified. The authors 
dedicate compelling evidence to the first-stage relationship between
index additions and managerial incentives (further evidencing 
results found in \ifdraft(anton2023) \else\textcite{anton2023}\fi).
However, sentences like ``Anton et al. (2022) and our validation of their results 
show that index addition shock is a valid instrument, as it causes significant 
variation in managerial incentives.'' are jarring to read, as they seem to imply
that the validity of an instrument relies only on instrument relevance. (If there 
is a terminology norm that I am not familiar with, I apologize, but this is 
not the norm that I am familiar with.) 

The point of confusion referenced in the last paragraph is 
indicative of my broader complaint, which is that the authors dedicate 
very little time to convincing the reader that the 
exclusion restriction is satisfied. The authors' argument
on this front is primarily limited to
the sentence:
``Furthermore, the exclusion condition, that the shock to common 
ownership only affects productivity via its effect on managerial 
incentives, is fulfilled, assuming that our prior argument for 
insignificant countervailing effects via competitive pressures holds''
and the use of the Sargan-Hansen test.
Though it is good that the authors' cite the evidence from 
\ifdraft(afego2017) \else\textcite{afego2017} \fi to 
argue against the role of competitive pressures, I would have anticipated 
more discussion of the alternative avenues by which index additions (even in 
the diff-in-diff design) could 
influence firm productivity outside of common ownership (partially provided), 
as well as 
the avenues by which common ownership could influence firm productivity
outside of managerial incentives. It may well be that the exclusion 
restriction is satisfied, but given its importance for the claims the authors 
are making, I would want more evidence/argument. For example, it doesn't seem that 
they spend much time -- in the context of the instrument -- 
discussing possible effects of common ownership on productivity 
via other mechanisms
put forward in 
\ifdraft(rotemberg1984) \else\textcite{rotemberg1984}\fi.

The discussion of ``revenue'' productivity versus more conventional 
productivity through the lens of analyzing the effects on markups 
is helpful. If there is any way to provide cleaner bounds on the effect 
on conventional productivity, that would be helpful. (Perhaps there are additional assumptions or
partial 
identification methods that could be applicable to this exercise.)

I am somewhat perplexed by the intensely negative $R^2$ values in some of 
the models, e.g., Table 30 with a -1.8. This may come from my own ignorance, 
but it may be beneficial to clarify why that's normal in this setting. 

It seems that a number of robustness checks fail to reproduce the 
main results (at least at significant levels; e.g., the dynamic panel). These results, 
combined with the insignificant diff-in-diff, provide a 
number of results that don't seem to cleanly fit the proposed 
narrative.
I think it is to the authors' credit that they included all of these 
results despite their lack of significance, rather than attempt to 
justify not including them or massage them into something cleaner. 
That said, I think that the authors could spend a bit more 
time convincing the reader of why their chosen iterations
of answering this question are the most compelling.



%%%%%%%%%%%%%%%%%%%%%%%%%%%%%%%%%%%%%%%%%%%%%%%%%%%%%%%%%%%%%%%%%%%%%%%%%%%%%%%%%%%%%%%
%%%%%%%%%%%%%%%%%%%%%%%%%%%%%%%%%%%%%%%%%%%%%%%%%%%%%%%%%%%%%%%%%%%%%%%%%%%%%%%%%%%%%%%
%%%%%%%%%%%%%%%%%%%%%%%%%%%%%%%%%%%%%%%%%%%%%%%%%%%%%%%%%%%%%%%%%%%%%%%%%%%%%%%%%%%%%%%
\ifdraft
% Skip bibliography in draft mode
\else
\printbibliography
\fi

\end{document}