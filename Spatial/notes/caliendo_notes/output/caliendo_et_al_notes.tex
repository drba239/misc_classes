\documentclass[10pt]{article}
\usepackage{amsmath}
\usepackage{amsthm}
\usepackage{amsfonts}
\usepackage{amssymb}
\usepackage{amssymb}
\usepackage{booktabs}
\setlength\parindent{0pt}
\usepackage[margin=1.2in]{geometry}
\usepackage{enumitem}
\usepackage{mathtools}
\mathtoolsset{showonlyrefs=true}
\usepackage{pdflscape}
\usepackage{xcolor}
\usepackage{hyperref}
\setcounter{tocdepth}{4}
\setcounter{secnumdepth}{4}
\usepackage[listings,skins,breakable]{tcolorbox} % package for colored boxes
\usepackage{etoolbox}
\usepackage{placeins}
\usepackage{tikz}
\usepackage{color}  % Allows for color customization
\usepackage{subcaption}
\usepackage[utf8]{inputenc}


% Make it so that the bottom page of a 
% book section doesn't have weird spacing
\raggedbottom


% Define custom colors
% You can-redo these later, they're not being used 
% for anything as of 5/12/24
\definecolor{codegreen}{rgb}{0,0.6,0}
\definecolor{codegray}{rgb}{0.5,0.5,0.5}
\definecolor{codepurple}{rgb}{0.58,0,0.82}
\definecolor{backcolour}{rgb}{0.95,0.95,0.92}

% You can-redo the lstlisting style later, it's not being used 
% for anything as of 5/12/24

% Define the lstlisting style
\lstdefinestyle{mystyle}{
    backgroundcolor=\color{backcolour},   
    commentstyle=\color{codegreen},
    keywordstyle=\color{magenta},
    numberstyle=\tiny\color{codegray},
    stringstyle=\color{codepurple},
    basicstyle=\ttfamily\footnotesize,
    breakatwhitespace=false,         
    breaklines=true,                 
    captionpos=b,                    
    keepspaces=true,                 
    numbers=left,                    
    numbersep=5pt,                  
    showspaces=false,                
    showstringspaces=false,
    showtabs=false,                  
    tabsize=2
}
\lstset{style=mystyle}


% Set the length of \parskip to add a line between paragraphs
\setlength{\parskip}{1em}


% Set the second level of itemize to use \circ as the bullet point
\setlist[itemize,2]{label={$\circ$}}

% Define symbols
\DeclareMathSymbol{\Perp}{\mathrel}{symbols}{"3F}
\newcommand\barbelow[1]{\stackunder[1.2pt]{$#1$}{\rule{.8ex}{.075ex}}}
\newcommand{\succprec}{\mathrel{\mathpalette\succ@prec{\succ\prec}}}
\newcommand{\precsucc}{\mathrel{\mathpalette\succ@prec{\prec\succ}}}


\newcounter{example}[section] % Reset example counter at each new section
\renewcommand{\theexample}{\thesection.\arabic{example}} % Format the example number as section.number

\newenvironment{example}
  {% Begin environment
   \refstepcounter{example}% Step counter and allow for labeling
   \noindent\textbf{Example \theexample.} % Display the example number
  }
  {% End environment
   \par\noindent\hfill\textit{End of Example.}\par
  }



% deeper section command
% This will let you go one level deeper than whatever section level you're on.
\makeatletter
\newcommand{\deepersection}[1]{%
  \ifnum\value{subparagraph}>0
    % Already at the deepest standard level (\subparagraph), cannot go deeper
    \subparagraph{#1}
  \else
    \ifnum\value{paragraph}>0
      \subparagraph{#1}
    \else
      \ifnum\value{subsubsection}>0
        \paragraph{#1}
      \else
        \ifnum\value{subsection}>0
          \subsubsection{#1}
        \else
          \ifnum\value{section}>0
            \subsection{#1}
          \else
            \section{#1}
          \fi
        \fi
      \fi
    \fi
  \fi
}
\makeatother


% same section command
% This will let create a section at the same level as whatever section level you're on.
\makeatletter
\newcommand{\samesection}[1]{%
  \ifnum\value{subparagraph}>0
    \subparagraph{#1}
  \else
    \ifnum\value{paragraph}>0
      \paragraph{#1}
    \else
      \ifnum\value{subsubsection}>0
        \subsubsection{#1}
      \else
        \ifnum\value{subsection}>0
          \subsection{#1}
        \else
          \ifnum\value{section}>0
            \section{#1}
          \else
            % Default to section if outside any sectioning
            \section{#1}
          \fi
        \fi
      \fi
    \fi
  \fi
}
\makeatother

\makeatletter
\newcommand{\shallowersection}[1]{%
  \ifnum\value{subparagraph}>0
    \paragraph{#1} % From subparagraph to paragraph
  \else
    \ifnum\value{paragraph}>0
      \subsubsection{#1} % From paragraph to subsubsection
    \else
      \ifnum\value{subsubsection}>0
        \subsection{#1} % From subsubsection to subsection
      \else
        \ifnum\value{subsection}>0
          \section{#1} % From subsection to section
        \else
          \ifnum\value{section}>0
            \chapter{#1} % Assuming a document class with chapters
          \else
            \section{#1} % Default to section if somehow higher than section
          \fi
        \fi
      \fi
    \fi
  \fi
}
\makeatother



\newcounter{problemcounter}
\renewcommand{\theproblemcounter}{Q.\arabic{problemcounter}}

% Define the problem environment
\newenvironment{problem}[1][]{%
  \refstepcounter{problemcounter}%
  \if\relax\detokenize{#1}\relax
    \tcolorbox[breakable, colback=red!10, colframe=red!50, fonttitle=\bfseries, title={Problem \theproblemcounter}, arc=5mm, boxrule=0.5mm]
  \else
    \tcolorbox[breakable, colback=red!10, colframe=red!50, fonttitle=\bfseries, title={Problem \theproblemcounter: #1}, arc=5mm, boxrule=0.5mm]
    \addcontentsline{toc}{subsubsection}{\theproblemcounter: #1}%
  \fi
}{
  \endtcolorbox
}

% Define a new counter for definitions
\newcounter{definitioncounter}
\renewcommand{\thedefinitioncounter}{D.\arabic{definitioncounter}}

\newenvironment{definition}[1][]{%
  \refstepcounter{definitioncounter}%
  \if\relax\detokenize{#1}\relax
    \tcolorbox[
      breakable,
      parbox=false, % Treat content normally regarding paragraphs
      before upper={\parindent0pt \parskip7pt}, % No indentation and add space between paragraphs
      colback=blue!10,
      colframe=blue!50,
      fonttitle=\bfseries,
      title={Definition \thedefinitioncounter},
      arc=5mm,
      boxrule=0.5mm,
      before skip=10pt, % Adjust vertical space before the box
      after skip=10pt % Adjust vertical space after the box
    ]
  \else
    \tcolorbox[
      breakable,
      parbox=false,
      before upper={\parindent0pt \parskip7pt},
      colback=blue!10,
      colframe=blue!50,
      fonttitle=\bfseries,
      title={Definition \thedefinitioncounter: #1},
      arc=5mm,
      boxrule=0.5mm,
      before skip=10pt,
      after skip=10pt
    ]
  \fi
}{
  \endtcolorbox
}


% Define a new counter for theorems
\newcounter{theoremcounter}
\renewcommand{\thetheoremcounter}{T.\arabic{theoremcounter}}

\newenvironment{theorem}[1][]{%
  \refstepcounter{theoremcounter}%
  \if\relax\detokenize{#1}\relax
    \tcolorbox[
      breakable,
      parbox=false, % Treat content normally regarding paragraphs
      before upper={\parindent0pt \parskip7pt}, % No indentation and add space between paragraphs
      colback=green!10,
      colframe=green!55,
      fonttitle=\bfseries,
      title={Theorem \thetheoremcounter},
      arc=5mm,
      boxrule=0.5mm,
      before skip=10pt, % Adjust vertical space before the box
      after skip=10pt % Adjust vertical space after the box
    ]
  \else
    \tcolorbox[
      breakable,
      parbox=false,
      before upper={\parindent0pt \parskip7pt},
      colback=green!10,
      colframe=green!55,
      fonttitle=\bfseries,
      title={Theorem \thetheoremcounter: #1},
      arc=5mm,
      boxrule=0.5mm,
      before skip=10pt,
      after skip=10pt
    ]
  \fi
}{
  \endtcolorbox
}


% Define a new counter for remarks
\newcounter{remarkcounter}
\renewcommand{\theremarkcounter}{R.\arabic{remarkcounter}}

\newenvironment{remark}[1][]{%
  \refstepcounter{remarkcounter}%
  \if\relax\detokenize{#1}\relax
    \tcolorbox[
      breakable,
      parbox=false, % Treat content normally regarding paragraphs
      before upper={\parindent0pt \parskip7pt}, % No indentation and add space between paragraphs
      colback=green!10,
      colframe=green!55,
      fonttitle=\bfseries,
      title={Remark \theremarkcounter},
      arc=5mm,
      boxrule=0.5mm,
      before skip=10pt, % Adjust vertical space before the box
      after skip=10pt % Adjust vertical space after the box
    ]
  \else
    \tcolorbox[
      breakable,
      parbox=false,
      before upper={\parindent0pt \parskip7pt},
      colback=green!10,
      colframe=green!55,
      fonttitle=\bfseries,
      title={Remark \theremarkcounter: #1},
      arc=5mm,
      boxrule=0.5mm,
      before skip=10pt,
      after skip=10pt
    ]
  \fi
}{
  \endtcolorbox
}

% Define a new counter for lemmas
\newcounter{lemmacounter}
\renewcommand{\thelemmacounter}{L.\arabic{lemmacounter}}

\newenvironment{lemma}[1][]{%
  \refstepcounter{lemmacounter}%
  \if\relax\detokenize{#1}\relax
    \tcolorbox[
      breakable,
      parbox=false, % Treat content normally regarding paragraphs
      before upper={\parindent0pt \parskip7pt}, % No indentation and add space between paragraphs
      colback=green!10,
      colframe=green!55,
      fonttitle=\bfseries,
      title={Lemma \thelemmacounter},
      arc=5mm,
      boxrule=0.5mm,
      before skip=10pt, % Adjust vertical space before the box
      after skip=10pt % Adjust vertical space after the box
    ]
  \else
    \tcolorbox[
      breakable,
      parbox=false,
      before upper={\parindent0pt \parskip7pt},
      colback=green!10,
      colframe=green!55,
      fonttitle=\bfseries,
      title={Lemma \thelemmacounter: #1},
      arc=5mm,
      boxrule=0.5mm,
      before skip=10pt,
      after skip=10pt
    ]
  \fi
}{
  \endtcolorbox
}

% Define a new counter for propositions
\newcounter{propositioncounter}
\renewcommand{\thepropositioncounter}{P.\arabic{propositioncounter}}

\newenvironment{proposition}[1][]{%
  \refstepcounter{propositioncounter}%
  \if\relax\detokenize{#1}\relax
    \tcolorbox[
      breakable,
      parbox=false, % Treat content normally regarding paragraphs
      before upper={\parindent0pt \parskip7pt}, % No indentation and add space between paragraphs
      colback=green!10,
      colframe=green!55,
      fonttitle=\bfseries,
      title={Proposition \thepropositioncounter},
      arc=5mm,
      boxrule=0.5mm,
      before skip=10pt, % Adjust vertical space before the box
      after skip=10pt % Adjust vertical space after the box
    ]
  \else
    \tcolorbox[
      breakable,
      parbox=false,
      before upper={\parindent0pt \parskip7pt},
      colback=green!10,
      colframe=green!55,
      fonttitle=\bfseries,
      title={Proposition \thepropositioncounter: #1},
      arc=5mm,
      boxrule=0.5mm,
      before skip=10pt,
      after skip=10pt
    ]
  \fi
}{
  \endtcolorbox
}

%\newtheorem{proposition}[theorem]{Proposition}  % Propositions share numbering with theorems


% Define a new counter for notes
\newcounter{notescounter}
\renewcommand{\thenotescounter}{D.\arabic{notescounter}}

\newenvironment{notes}[1][]{
  \refstepcounter{notescounter}%
  \if\relax\detokenize{#1}\relax
    % If #1 is empty, set the title to "Notes"
    \tcolorbox[
      breakable,
      parbox=false, % Treat content normally regarding paragraphs
      before upper={\parindent0pt \parskip7pt}, % No indentation and add space between paragraphs
      colback=blue!10,
      colframe=blue!50,
      fonttitle=\bfseries,
      title={Notes},
      arc=5mm,
      boxrule=0.5mm,
      before skip=10pt, % Adjust vertical space before the box
      after skip=10pt % Adjust vertical space after the box
    ]
  \else
    % If #1 is not empty, use it as the title
    \tcolorbox[
      breakable,
      parbox=false,
      before upper={\parindent0pt \parskip7pt},
      colback=blue!10,
      colframe=blue!50,
      fonttitle=\bfseries,
      title={#1}, % Use provided title instead of default
      arc=5mm,
      boxrule=0.5mm,
      before skip=10pt,
      after skip=10pt
    ]
  \fi
}{
  \endtcolorbox
}







% Define a new counter for questions
\newcounter{questionscounter}
\renewcommand{\thequestionscounter}{D.\arabic{questionscounter}}

\newenvironment{questions}[1][]{
  \refstepcounter{questionscounter}%
  \if\relax\detokenize{#1}\relax
    % If #1 is empty, set the title to "Questions"
    \tcolorbox[
      breakable,
      parbox=false, % Treat content normally regarding paragraphs
      before upper={\parindent0pt \parskip7pt}, % No indentation and add space between paragraphs
      colback=red!10,
      colframe=red!50,
      fonttitle=\bfseries,
      title={Questions},
      arc=5mm,
      boxrule=0.5mm,
      before skip=10pt, % Adjust vertical space before the box
      after skip=10pt % Adjust vertical space after the box
    ]
  \else
    % If #1 is not empty, use it as the title
    \tcolorbox[
      breakable,
      parbox=false,
      before upper={\parindent0pt \parskip7pt},
      colback=red!10,
      colframe=red!50,
      fonttitle=\bfseries,
      title={#1}, % Use provided title instead of default
      arc=5mm,
      boxrule=0.5mm,
      before skip=10pt,
      after skip=10pt
    ]
  \fi
}{
  \endtcolorbox
}






% Define a new counter for overview
\newcounter{overviewcounter}
\renewcommand{\theoverviewcounter}{D.\arabic{overviewcounter}}

\newenvironment{overview}[1][]{%
  \refstepcounter{overviewcounter}%
  \if\relax\detokenize{#1}\relax
    \tcolorbox[
      breakable,
      parbox=false, % Treat content normally regarding paragraphs
      before upper={\parindent0pt \parskip7pt}, % No indentation and add space between paragraphs
      colback=green!10,
      colframe=green!55,
      fonttitle=\bfseries,
      title={Overview},
      arc=5mm,
      boxrule=0.5mm,
      before skip=10pt, % Adjust vertical space before the box
      after skip=10pt % Adjust vertical space after the box
    ]
  \else
    \tcolorbox[
      breakable,
      parbox=false,
      before upper={\parindent0pt \parskip7pt},
      colback=green!10,
      colframe=green!55,
      fonttitle=\bfseries,
      title={Overview},
      arc=5mm,
      boxrule=0.5mm,
      before skip=10pt,
      after skip=10pt
    ]
  \fi
}{
  \endtcolorbox
}



% Add a line after paragraph header
\makeatletter
\renewcommand\paragraph{\@startsection{paragraph}{4}{\z@}%
            {-3.25ex \@plus -1ex \@minus -.2ex}%
            {1.5ex \@plus .2ex}%
            {\normalfont\normalsize\bfseries}}
\makeatother


% Taking away line before and after align
\BeforeBeginEnvironment{align}{\vspace{-\parskip}}
\AfterEndEnvironment{align}{\vskip0pt plus 2pt}

\usepackage{changepage}

\title{Caliendo Paper Notes}

\author{Dylan Baker}
\date{March 2025}

\begin{document}
\maketitle

\tableofcontents

\section{Terms}

\input{../input/caliendo_et_al_terms.tex}

\section{The Model}

\subsection{Initial Setup}


``In each region-sector combination,
there is a competitive labor market. 
In each market, there is a continuum of perfectly
competitive firms producing intermediate goods.''

Firms have a Cobb-Douglas Constant-Returns-to-Scale (CRS)
production function, which utilizes labor, 
``composite local factor that we refer to as structures, 
and materials from all sectors.''

We assume that productivities 
are 
``distributed Fréchet with a sector-specific
productivity dispersion parameter $\theta^j$.''

Time is discrete, $t=0,1,2, \ldots$.

``Households are forward looking,
have perfect foresight, and optimally decide where to 
move given some initial distribution of labor 
across locations and sectors.
Households face costs to move across markets
and experience an idiosyncratic shock that affects their moving decision.''

\subsection{Household Problem}

At $t=0$, there is a mass of households 
in location $n$ and sector $j$, denoted by $L_0^{n j}$.
Households are either \emph{employed} or 
\emph{non-employed}.

If employed in location $n$ and sector $j$ at time $t$, 
workers inelastically supply a unit of labor 
and receive wage $w_t^{n j}$.

Given income, the household decides how to allocate 
their consumption over final goods across sectors 
with a Cobb-Douglas aggregator. Preferences,
$U\left(C_t^{n j}\right)$, are over 
baskets of final local goods:

\begin{align}
    C_t^{n j}=\prod_{k=1}^J\left(c_t^{n j, k}\right)^{\alpha^k}
\end{align}

``Households are forward-looking 
and discount the future at rate $\beta \geq 0$.
Migration decisions are subject to sectoral and spatial 
mobility costs.''

\begin{notes}[Assumption 1]
    Labor relocation costs $\tau^{n j, i k} \geq 0$ depend on the origin ($n j$) and destination ($ik$) and are time invariant, additive, and measured in terms of utility.
\end{notes}

Households have idiosyncratic shocks $\epsilon_t^{i k}$ for each choice of market.


The timing for the household's problem is as follows:

\begin{enumerate}
    \item Households observe the economic conditions in each market, as well 
        as their own idiosyncratic shocks.
    \item Returns
        \begin{itemize}
            \item If they begin the period in the labor market, they work 
            and receive the market wage.
            \item  If they are non-employed in a region, they receive 
                home production.
        \end{itemize}
    \item Households choose whether to relocate.
\end{enumerate}

Formally:

\begin{align}
    \mathrm{v}_t^{n j}=&U\left(C_t^{n j}\right)+\max _{\{i, k\}_{i=1, k=0}^{N, J}}\left\{\beta E\left[\mathrm{v}_{t+1}^{i k}\right]-\tau^{n j, i k}+\nu \epsilon_t^{i k}\right\} \\
    & \text { s.t. } C_t^{n j} \equiv \begin{cases}b^n & \text { if } j=0 \\
    w_t^{n j} / P_t^n & \text { otherwise }\end{cases}
\end{align}

$\mathrm{v}_t^{n j}$ is the lifetime utility of a 
household currently in location $n$ and sector $j$ at time $t$, with 
the expectation taken over future realizations of the idiosyncratic shock. 
$\nu$ is a parameter that scales the variance of the idiosyncratic shock.

Households choose to move to the labor market 
with the highest utility net of costs.

\begin{notes}[Assumption 2]
    The idiosyncratic shock $\epsilon$ is i.i.d. over time 
    and distributed Type-I Extreme Value with zero mean.
\end{notes}

Note that Assumption 2 is common in the literature\footnote{Caliendo et al. (2019)
refers readers to 

Aguirregabiria, V., \& Mira, P. (2010). Dynamic discrete choice structural models: A survey. \emph{Journal of Econometrics, 156}(1), 38-67.} 
and is useful to allow ``for 
simple aggregation of idiosyncratic decisions 
made by households.''

Let

\begin{align}
    V_t^{n j} \equiv \mathbb{E}\left[\mathrm{v}_t^{n j}\right] \label{eq:V_t_nj_init}
\end{align}

Then 

\begin{align}
    V_t^{n j}=U\left(C_t^{n j}\right)+\nu \log \left(\sum_{i=1}^N \sum_{k=0}^J \exp \left(\beta V_{t+1}^{i k}-\tau^{n j, i k}\right)^{1 / \nu}\right) \label{eq:V_t_nj_exp}
\end{align}

The derivation is given in \autoref{sec:V_t_nj_exp}.

``$V_t^{n j}$ can be interpreted as the expected lifetime 
utility of a household before the realization of the 
household preference shocks or, alternatively, as the 
average utility of households in that market.''

From there we can derive the fraction of households 
that relocate from market $nj$ to $ik$:

\begin{align}
    \mu_t^{n j, i k}=\frac{\exp \left(\beta V_{t+1}^{i k}-\tau^{n j, i k}\right)^{1 / \nu}}{\sum_{m=1}^N \sum_{h=0}^J \exp \left(\beta V_{t+1}^{m h}-\tau^{n j, m h}\right)^{1 / \nu}} \label{eq:mu_t_nj_ik}
\end{align}

\begin{questions}[To-do]
    Add derivation if desired.
\end{questions}

All else equal \eqref{eq:mu_t_nj_ik}
implies that markets with a higher lifetime 
utility (net of migration costs) will be the ones to attract more migrants.
Moreover, it allows $\frac{1}{\nu}$ to be interpreted as a migration 
elasticity with respect to the expected utility of the destination market,
i.e., $\beta V_{t+1}^{i k}-\tau^{n j, i k}$.

You can see the elasticity interpretation intuitively if you consider:

\begin{align}
    &\mu_t^{n j, i k}=\frac{\exp \left(\beta V_{t+1}^{i k}-\tau^{n j, i k}\right)^{1 / \nu}}{\sum_{m=1}^N \sum_{h=0}^J \exp \left(\beta V_{t+1}^{m h}-\tau^{n j, m h}\right)^{1 / \nu}} \\
    \Rightarrow &\ln \mu_t^{n j, i k} = \frac{1}{\nu} \left( \beta V_{t+1}^{i k}-\tau^{n j, i k} - \ln \sum_{m=1}^N \sum_{h=0}^J \exp \left(\beta V_{t+1}^{m h}-\tau^{n j, m h}\right) \right) \\
    \Rightarrow &\frac{\partial \ln \mu_t^{n j, i k}}{\partial (\beta V_{t+1}^{i k} - \tau^{n j, i k})} \approx \frac{1}{\nu} \\
\end{align}

\eqref{eq:mu_t_nj_ik} ``conveys all the 
information needed to determine how the distribution of 
labor across markets evolves over time. In particular,''

\begin{align}
    L_{t+1}^{n j}=\sum_{i=1}^N \sum_{k=0}^J \mu_t^{i k, n j} L_t^{i k} \label{eq:L_t_nj}
\end{align}

\eqref{eq:L_t_nj} characterizes the distribution of 
employment and non-employment across markets. 

Note that, under the given timing assumptions, 
labor in period $t$ is ``fully determined by 
forward-looking decisions at period $t-1$.''
%%%%%%%%%%%%%%%%%%%%%%%%%%%%%%%%%%%%%%%%%%%%%%%%%%%%%%%%%%%%%%%%%%%%%%%%%%%%%%%%%%%%%%%
%%%%%%%%%%%%%%%%%%%%%%%%%%%%%%%%%%%%%%%%%%%%%%%%%%%%%%%%%%%%%%%%%%%%%%%%%%%%%%%%%%%%%%%

\subsection{Production}

``Firms in each sector and region 
are able to produce many varieties
of intermediate goods. ''

Production requires 
labor, structures, and materials from all sectors.

``TFP of an intermediate good is composed of two terms, 
a time-varying sectoral-regional component $\left(A_t^{n j}\right)$, 
which is common to all varieties in a region and sector, and a 
variety-specific component $\left(z^{n j}\right)$.''

%%%%%%%%%%%%%%%%%%%%%%%%%%%%%%%%%%%%%%%%%%%%%%%%%%%%%%%%%%%%%%%%%%%%%%%%%%%%%%%%%%%%%%%

\subsubsection{Intermediate Good Production}

\paragraph{Production Function}

We assume that the production function is CRS and Cobb-Douglas.

The output for a producer of an intermediate good is given by:

\begin{align}
    q_t^{n j}=z^{n j}\left(A_t^{n j}\left(h_t^{n j}\right)^{\xi^n}\left(l_t^{n j}\right)^{1-\xi^n}\right)^{\gamma^{n j}} \prod_{k=1}^J\left(M_t^{n j, n k}\right)^{\gamma^{n j, n k}}
\end{align}

where

\begin{itemize}
    \item $l_t^{n j}$: Labor inputs in location $n$ and sector $j$ at time $t$
    \item $h_t^{n j}$: Structure inputs in location $n$ and sector $j$ at time $t$
    \item $M_t^{n j, n k}$: Material 
        inputs from sector $k$ demanded by a firm in sector $j$ and region $n$ at time $t$
    \item $\gamma^{n j} \geq 0$: Share of value-added in 
        the production of sector $j$ and region $n$
        \begin{itemize}
            \item This comes from $\gamma^{n j} + \sum_{k=1}^J \gamma^{n j, n k}=1$ and the Cobb-Douglas form, 
                so the value-add share is the share that comes from 
                the non-material inputs.
        \end{itemize}
    \item $\gamma^{n, n k} \geq 0$: The share of materials from sector $k$ in 
        the production of sector $j$ and region $n$.
    \item $\xi^n$: The share of structures in value-add.
\end{itemize}

``Material inputs are goods from sector $k$
produced in the same region $n$.''

\paragraph{Unit Price of an Input Bundle}

The unit price of an input bundle is given by:

\begin{align}
    x_t^{n j}=B^{n j}\left(\left(r_t^{n j}\right)^{\xi^n}\left(w_t^{n j}\right)^{1-\xi^n}\right)^{\gamma^{n j}} \prod_{k=1}^J\left(P_t^{n k}\right)^{\gamma^{n j, n k}}
\end{align}

where $B^{n j}$ is a constant.

Then the unit cost of a given intermediate good is given by:

\begin{align}
    \frac{x_t^{n j}}{z^{n j}\left(A_t^{n j}\right)^{\gamma^{n j}}}
\end{align}

since the cost for one unit of the inputs is $x_t^{n j}$ and
the productivity of the variety, $z^{n j}\left(A_t^{n j}\right)^{\gamma^{n j}}$,
scales the amount you can produce with the relevant input.

\paragraph{Price of Good $j$ in Region $n$}

We take

\begin{itemize}
    \item $\boldsymbol{\kappa}_t^{n j, i j}$: Iceberg trade costs between market location $n$ and $i$ in 
    sector $j$.
    \begin{itemize}
        \item Delivering one unit of good from market $n$ to $i$ 
            requires producing $\boldsymbol{\kappa}_t^{n j, i j} \geq  1$ units of the good.
        \item If a good is nontradable, then $\kappa=\infty$.
    \end{itemize}
\end{itemize}

``Competition implies that the 
price paid for a particular variety of good $j$ in region $n$
is given by the minimum cost across regions, 
taking into account trade costs, 
and where the vector of productivity 
draws received by the different regions is ''

\begin{align}
    z^j=\left(z^{1 j}, z^{2 j}, \ldots, z^{N j}\right)
\end{align}

``That is, using $z^j$ to index varieties:''

\begin{align}
    p_t^{n j}\left(z^j\right)=\min _i\left\{\frac{\kappa_t^{n j, i j} x_t^{i j}}{z^{i j}\left(A_t^{i j}\right)^{\gamma^{i j}}}\right\}
\end{align}

\subsubsection{Local Sectoral Aggregate Goods}

\paragraph{Quantity of Aggregate Sectoral Goods}

``Intermediate goods demanded 
from sector $j$ 
and from all regions are aggregated 
into a local sectoral good denoted by $Q$
and that can be thought of as a bundle of 
goods purchased from different regions.''

Take 

\begin{itemize}
    \item $Q_t^{n j}$: The quantity produced of aggregate sectoral 
        goods $j$ in region $n$ 
        \begin{align}
            Q_t^{n j}=\left(\int\left(\tilde{q}_t^{n j}\left(z^j\right)\right)^{1-1 / \eta^{n j}} d \phi^j\left(z^j\right)\right)^{\eta^{n j} /\left(\eta^{n j}-1\right)}
        \end{align}
    \item $\phi^j\left(z^j\right)$: The joint distribution over the vector $z^j$.
        \begin{align}
            \phi^j\left(z^j\right)=\exp \left\{-\sum_{n=1}^N\left(z^{n j}\right)^{-\theta^j}\right\}
        \end{align}
        \begin{itemize}
            \item $\phi^{n j}\left(z^{n j}\right)$: The marginal distribution of $z^{n j}$.
                \begin{align}
                    \phi^{n j}\left(z^{n j}\right)=\exp \left\{-\left(z^{n j}\right)^{-\theta^j}\right\}
                \end{align}
        \end{itemize}
    \item $\tilde{q}_t^{n j}\left(z^j\right)$: The 
        quantity demanded of an intermediate good
        of a given variety from the lowest-cost supplier.
\end{itemize}

Note that 
``there are no fixed costs 
or barriers to entry and exit 
in the production of intermediate and sectoral goods. ''
Moreover, ``[c]ompetitive behavior 
implies zero profit at all times.''

\paragraph{Price of Aggregate Sectoral Goods}

Given properties of the Fréchet distribution,
the price of the aggregate sectoral good is given by:

\begin{align}
    P_t^{n j}=\Gamma^{n j}\left(\sum_{i=1}^N\left(x_t^{i j} \kappa_t^{n j, i j}\right)^{-\theta^j}\left(A_t^{i j}\right)^{\theta^j \gamma^{i j}}\right)^{-1 / \theta^j}
\end{align}

where $\Gamma^{n j}$ is a constant
and we assume that $1+\theta^j>\eta^{n j}$.

\paragraph{Share of Total Expenditure in Market $nj$ on Goods $j$ from Market $i$}

For the share of total expenditure in market $nj$ on goods $j$ from market $i$, we 
get the expression:

\begin{align}
    \pi_t^{n j, i j}=\frac{\left(x_t^{i j} \kappa_t^{n j, i j}\right)^{-\theta^j}\left(A_t^{i j}\right)^{\theta^j \gamma^{i j}}}{\sum_{m=1}^N\left(x_t^{m j} \kappa_t^{n j, m j}\right)^{-\theta^j}\left(A_t^{m j}\right)^{\theta^j \gamma^{m j}}}
\end{align}

``This equilibrium condition resembles a gravity equation.''

%%%%%%%%%%%%%%%%%%%%%%%%%%%%%%%%%%%%%%%%%%%%%%%%%%%%%%%%%%%%%%%%%%%%%%%%%%%%%%%%%%%%%%%
\subsubsection{Market Clearing and Unbalanced Trade}

\paragraph{Global Portfolio and Rent}

We assume a mass 1 of rentiers in each region who 
own the local structures, rent them to firms, send their 
rents to a global portfolio, receive a constant
share $\iota^n$ from the global portfolio, 
with $\sum_{n=1}^N \iota^n=1$,
and cannot re-locate.

The differences between the 
remittances and the income 
rentiers receive from the global portfolio
generate imbalances, ``which changes in magnitude 
as the rental prices change.''

The imbalance is given by:

\begin{align}
    \sum_{k=1}^J r_t^{i k} H^{i k}-\iota^n \chi_t
\end{align}

where 

\begin{align}
    \chi_t=\sum_{i=1}^N \sum_{k=1}^J r_t^{i k} H^{i k}
\end{align}

is the total revenue in the global portfolio.

\paragraph{Total Expenditure on Goods $j$ in Region $n$}

Let $X_t^{n j}$ be the total expenditure on sector $j$ good in region $n$. Then, goods market clearing implies

\begin{align}
    X_t^{n j}=\underbrace{\sum_{k=1}^J \gamma^{n k, n j} \sum_{i=1}^N \pi_t^{i k, n k} X_t^{i k}}_{\parbox[t]{4cm}{\raggedright \footnotesize \centering Value of total demand for $nj$ market goods used as materials}}+
        \underbrace{\alpha^j\left(\sum_{k=1}^J w_t^{n k} L_t^{n k}+\iota^n \chi_t\right)}_{\parbox[t]{2cm}{\raggedright \footnotesize \centering Value of final demand in region $n$}}
\end{align}

\paragraph{Labor Expression}

``Labor market clearing in region $n$ and sector $j$ 
is''

\begin{align}
    L_t^{n j}=\underbrace{\frac{1}{w_t^{n j}} \underbrace{\underbrace{\gamma^{n j}}_{\parbox[t]{1cm}{\raggedright \footnotesize \centering Value-added share}}\underbrace{\left(1-\xi^n\right)}_{\parbox[t]{1.8cm}{\raggedright \footnotesize \centering Share of value-add paid to labor}} \underbrace{\sum_{i=1}^N \pi_t^{i j, n j} X_t^{i j}}_{\parbox[t]{2cm}{\raggedright \footnotesize \centering Total expenditure from $ij$ market on $nj$ goods}}}_{\parbox[t]{2cm}{\raggedright \footnotesize \centering Total wage bill in $nj$}}}_{\parbox[t]{2cm}{\raggedright \footnotesize \centering Labor quantity demanded in $nj$}}
\end{align}

``while the labor market clearing for structures in region $n$ and sector $j$ must satisfy''


\begin{align}
    H^{n j}=\frac{\gamma^{n j} \xi^n}{r_t^{n j}} \sum_{i=1}^N \pi_t^{i j, n j} X_t^{i j}
\end{align}

%%%%%%%%%%%%%%%%%%%%%%%%%%%%%%%%%%%%%%%%%%%%%%%%%%%%%%%%%%%%%%%%%%%%%%%%%%%%%%%%%%%%%%%
%%%%%%%%%%%%%%%%%%%%%%%%%%%%%%%%%%%%%%%%%%%%%%%%%%%%%%%%%%%%%%%%%%%%%%%%%%%%%%%%%%%%%%%
\subsection{Equilibrium}

\subsubsection{Fundamentals and Parameters}

``The endogenous state of the economy 
at any moment in time is given by the distribution of labor 
across all markets $L_t$.''

The fundamentals of the economy are deterministic and include 

\begin{itemize}
    \item $A_t=\left\{A_t^{n j}\right\}_{n=1, j=1}^{N, J}$: Aggregate sectoral regional productivities
    \item $\boldsymbol{\kappa}_t=\left\{\boldsymbol{\kappa}_t^{n j, i j}\right\}_{n=1, i=1, j=1}^{N, N, J}$: Bilateral trade costs 
    \item $Y=\left\{\tau^{n j, i k}\right\}_{n=1, j=0, i=1, k=0}^{N, J, J, N}$: Labor re-location costs
    \item $H=\left\{H^{n j}\right\}_{n=1, j=1}^{N, J}$: Stock of land and structures across markets
    \item $b=\left\{b^n\right\}_{n=1}^N$: Home production across regions
\end{itemize}

Time-varying fundamentals are denoted by

\begin{align}
    \Theta_t \equiv\left(A_t, \kappa_t\right)
\end{align}

Constant fundamentals are denoted by

\begin{align}
    \bar{\Theta} \equiv(\Upsilon, H, b)
\end{align}

The parameters in the economy, assumed constant throughout the paper, 
are given by:

\begin{itemize}
    \item $\gamma^{n j}$: Value-added shares
    \item $\left(1-\xi^n\right)$: Labor-share in value-added
    \item $\left(\gamma^{n k, n j}\right)$: Input-output coefficients
    \item $\iota^n$: Portfolio shares
    \item $\alpha^j$: Final consumption expenditure shares
    \item $\beta$: The discount factor
    \item $\theta$: Trade elasticities
    \item $\nu$: Migration elasticity
\end{itemize}

%%%%%%%%%%%%%%%%%%%%%%%%%%%%%%%%%%%%%%%%%%%%%%%%%%%%%%%%%%%%%%%%%%%%%%%%%%%%%%%%%%%%%%%

\subsubsection{Equilibrium Definition}

\paragraph{Temporary Equilibrium}


\begin{definition}[Temporary Equilibrium] 
    
    Given $\left(L_t, \boldsymbol{\Theta}_t, \overline{\boldsymbol{\Theta}}\right)$, a temporary equilibrium is a vector of wages $w\left(L_t, \boldsymbol{\Theta}_t\right.$, $\bar{\Theta})$ that 
    satisfies the equilibrium conditions of the static subproblem
    characterized by:

    \begin{align}
        x_t^{n j}=&B^{n j}\left(\left(r_t^{n j}\right)^{\xi^n}\left(w_t^{n j}\right)^{1-\xi^n}\right)^{\gamma^{n j}} \prod_{k=1}^J\left(P_t^{n k}\right)^{\gamma^{n j, n k}} \\
        P_t^{n j}=&\Gamma^{n j}\left(\sum_{i=1}^N\left(x_t^{i j} \kappa_t^{n j, i j}\right)^{-\theta^j}\left(A_t^{i j}\right)^{\theta^j \gamma^{i j}}\right)^{-1 / \theta^j} \\
        \pi_t^{n j, i j}=&\frac{\left(x_t^{i j} \kappa_t^{n j, i j}\right)^{-\theta^j}\left(A_t^{i j}\right)^{\theta^j \gamma^{i j}}}{\sum_{m=1}^N\left(x_t^{m j} \kappa_t^{n j, m j}\right)^{-\theta^j}\left(A_t^{m j}\right)^{\theta^j \gamma^{m j}}} \\
        X_t^{n j}=&\sum_{k=1}^J \gamma^{n k, n j} \sum_{i=1}^N \pi_t^{i k, n k} X_t^{i k}+\alpha^j\left(\sum_{k=1}^J w_t^{n k} L_t^{n k}+\iota^n \chi_t\right) \\
        L_t^{n j}=&\frac{\gamma^{n j}\left(1-\xi^n\right)}{w_t^{n j}} \sum_{i=1}^N \pi_t^{i j, n j} X_t^{i j} \\
        H^{n j}=&\frac{\gamma^{n j} \xi^n}{r_t^{n j}} \sum_{i=1}^N \pi_t^{i j, n j} X_t^{i j}
    \end{align}

\end{definition}

``The temporary equilibrium of our model is the solution 
to a static multicountry interregional trade model.''

We can then express real wages as:

\begin{align}
    \omega^{n j}\left(L_t, \Theta_t, \bar{\Theta}\right)=w_t^{n j} / P_t^n
\end{align}

\paragraph{Sequential Competitive Equilibrium}

Consider

\begin{itemize}
    \item $\mu_t=\left\{\mu_t^{n j, i k}\right\}_{n=1, j=0, i=1, k=0}^{N, J, N, J}$: The migration shares
    \item $V_t=\left\{V_t^{n j}\right\}_{n=1, j=0}^{N, J}$: Lifetime utilities
\end{itemize}

\begin{definition}[Sequential Competitive Equilibrium]
    
    Given $\left(L_0,\left\{\Theta_t\right\}_{t=0}^{\infty}, \bar{\Theta}\right)$, 
    a sequential competitive equilibrium of the model is a sequence of 
    $\left\{L_t, \mu_t, V_t, w\left(L_t, \Theta_t, \bar{\Theta}\right)\right\}_{t=0}^{\infty}$ 
    that solves equilibrium conditions 
    
    \begin{align}
        V_t^{n j}=&U\left(C_t^{n j}\right)+\nu \log \left(\sum_{i=1}^N \sum_{k=0}^J \exp \left(\beta V_{t+1}^{i k}-\tau^{n j, i k}\right)^{1 / \nu}\right) \\
        \mu_t^{n j, i k}=&\frac{\exp \left(\beta V_{t+1}^{i k}-\tau^{n j, i k}\right)^{1 / \nu}}{\sum_{m=1}^N \sum_{h=0}^J \exp \left(\beta V_{t+1}^{m h}-\tau^{n j, m h}\right)^{1 / \nu}} \\
        L_{t+1}^{n j}=&\sum_{i=1}^N \sum_{k=0}^J \mu_t^{i k, n j} L_t^{i k}
    \end{align}

    and the 
    temporary equilibrium at each $t$.
    
\end{definition}

\paragraph{Stationary Equilibrium}

\begin{definition}[Stationary Equilibrium] 
    
    A stationary equilibrium of the model is a sequential competitive 
    equilibrium such that 
    
    \begin{align}
        \left\{L_t, \mu_t, V_t, w\left(L_t, \Theta_t, \bar{\Theta}\right)\right\}_{t-0}^{\infty}
    \end{align}
    
    are constant for all $t$.

\end{definition}

%%%%%%%%%%%%%%%%%%%%%%%%%%%%%%%%%%%%%%%%%%%%%%%%%%%%%%%%%%%%%%%%%%%%%%%%%%%%%%%%%%%%%%%
%%%%%%%%%%%%%%%%%%%%%%%%%%%%%%%%%%%%%%%%%%%%%%%%%%%%%%%%%%%%%%%%%%%%%%%%%%%%%%%%%%%%%%%
%%%%%%%%%%%%%%%%%%%%%%%%%%%%%%%%%%%%%%%%%%%%%%%%%%%%%%%%%%%%%%%%%%%%%%%%%%%%%%%%%%%%%%%
\section{Dynamic Hat Algebra}

\subsection{Solving the Model}

We aim to study the general equilibrium implications 
of a change in fundamentals relative to a baseline economy.

\begin{definition}[Baseline Economy] 
    The baseline economy is the allocation 
    
    \begin{align}
        \left\{L_t, \mu_{t-1}, \pi_t, X_t\right\}_{t=0}^{\infty}
    \end{align}

    corresponding to the sequence of fundamentals $\left\{\Theta_t\right\}_{t=0}^{\infty}$ and to $\bar{\Theta}$.
\end{definition}

We now consider how to solve for the baseline economy in 
time differences.

We use the notation:

\begin{itemize}
    \item $\dot{y}_{t+1}$: The proportional change in any scalar or 
        vector between periods $t$ and $t+1$.
        \begin{align}
            \dot{y}_{t+1} \equiv\left(y_{t+1}^1 / y_t^1, y_{t+1}^2 / y_t^2, \ldots\right)
        \end{align}
\end{itemize}

%%%%%%%%%%%%%%%%%%%%%%%%%%%%%%%%%%%%%%%%%%%%%%%%%%%%%%%%%%%%%%%%%%%%%%%%%%%%%%%%%%%%%%%
\subsubsection{Temporary Equilibrium of the Baseline Economy After Changing Employment and Fundamentals}

First, we'll consider how to solve for a temporary equilibrium
of the baseline economy at $t+1$ after changing employment, $\dot{L}_{t+1}$, 
and fundamentals, $\dot{\Theta}_{t+1}$ -- which we can do without 
estimates of $\Theta_t$ or $\bar{\Theta}$.


\begin{proposition} 
    Given the allocation of the temporary equilibrium at 
    $t,\left\{L_t, \pi_t, X_t\right\}$, the solution to the temporary 
    equilibrium at $t+1$ for a given change in $\dot{L}_{t+1}$ and 
    $\dot{\Theta}_{t+1}$ does not require information on the level 
    of fundamentals at $t, \Theta_t$, or $\bar{\Theta}$. In particular, 
    it is obtained as the solution to the following system 
    of nonlinear equations: 

    \begin{align}
        \dot{x}_{t+1}^{n j}=&\left(\dot{L}_{t+1}^{n j}\right)^{\gamma^{n j} \xi^n}\left(\dot{w}_{t+1}^{n j}\right)^{\gamma^{n j}} \prod_{k=1}^J\left(\dot{P}_{t+1}^{n k}\right)^{\gamma^{n j, n k}} \\
        \dot{P}_{t+1}^{n j}=&\left(\sum_{i=1}^N \pi_t^{n j, i j}\left(\dot{x}_{t+1}^{i j} \dot{\kappa}_{t+1}^{n j, i j}\right)^{-\theta^j}\left(\dot{A}_{t+1}^{i j}\right)^{\theta^j \gamma^{i j}}\right)^{-1 / \theta^j} \\
        \pi_{t+1}^{n j, i j}=&\pi_t^{n j, i j}\left(\frac{\dot{x}_{t+1}^{i j} \dot{\kappa}_{t+1}^{n j, i j}}{\dot{P}_{t+1}^{n j}}\right)^{-\theta^j}\left(\dot{A}_{t+1}^{i j}\right)^{\theta^j \gamma^{i j}} \\
        X_{t+1}^{n j}=&\sum_{k=1}^J \gamma^{n k, n j} \sum_{i=1}^N \pi_{t+1}^{i k, n k} X_{t+1}^{i k}+\alpha^j\left(\sum_{k=1}^J \dot{w}_{t+1}^{n k} \dot{L}_{t+1}^{n k} w_t^{n k} L_t^{n k}+\iota^n \chi_{t+1}\right) \\
        \dot{w}_{t+1}^{n j} \dot{L}_{t+1}^{n j} w_t^{n j} L_t^{n j}=&\gamma^{n j}\left(1-\xi^n\right) \sum_{i=1}^N \pi_{t+1}^{i j, n j} X_{t+1}^{i j} \\
        \text { where } \quad \chi_{t+1}=&\sum_{i=1}^N \sum_{k=1}^J \frac{\xi^i}{1-\xi^i} \dot{w}_{t+1}^{i k} \dot{L}_{t+1}^{i k} w_t^{i k} L_t^{i k}
    \end{align}

\end{proposition}

This proposition highlights that 
we don't need to know the level of fundamentals
to solve for the temporary equilibrium at $t+1$; we can solve 
with only the changes in employment and fundamentals.

%%%%%%%%%%%%%%%%%%%%%%%%%%%%%%%%%%%%%%%%%%%%%%%%%%%%%%%%%%%%%%%%%%%%%%%%%%%%%%%%%%%%%%%
\subsubsection{Solving for the Sequential Equilibrium}

Now, we'll show that ``given 
an allocation at $t=0$, $\left\{L_0, \pi_0, X_0\right\}$, 
the matrix of gross migration flows at $t = -1$, $\mu_{-1}$,
and a sequence of changes in fundamentals, one can solve for the 
sequential equilibrium in time differences 
without needing to estimate the levels of fundamentals.''

\begin{definition}[Converging Sequence of Changes in Fundamentals] 
    A converging sequence of changes in fundamentals is such that $\lim _{t \rightarrow \infty} \dot{\Theta}_t=1$
\end{definition}

\begin{notes}[Assumption]
    Agents have logarithmic preferences, $U\left(C_t^{n j}\right) \equiv \log \left(C_t^{n j}\right)$.
\end{notes}

We denote

\begin{align}
    u_t^{n j} \equiv \exp \left(V_t^{n j}\right)
\end{align}

and 

\begin{itemize}
    \item $\dot{\omega}^{n j}\left(\dot{L}_{t+1}, \dot{\Theta}_{t+1}\right)$: (For all 
    $n$ and $j$) the equilibrium real wages in time differences 
    as functions of the change
    in labor $\dot{L}_{t+1}$ and time-varying fundamentals $\dot{\Theta}_{t+1}$.
\end{itemize}


``Namely, $\dot{\omega}^{n j}\left(\dot{L}_{t+1}, \dot{\Theta}_{t+1}\right)$ is the solution
to the system in Proposition 1.''

\begin{proposition} 
    Conditional on an initial allocation of the economy, 
    $\left(L_0, \pi_0, X_0, \mu_{-1}\right)$, given an anticipated 
    convergent sequence of changes in fundamentals, 
    $\left\{\dot{\Theta}_t\right\}_{t=1}^{\infty}$, the solution to the 
    sequential equilibrium in time differences does not require 
    information on the level of the fundamentals 
    $\left\{\Theta_t\right\}_{t=0}^{\infty}$ or $\bar{\Theta}$ and solves 
    the following system of nonlinear equations: 

    \begin{align}
        \mu_{t+1}^{n j, i k}=&\frac{\mu_t^{n j, i k}\left(\dot{u}_{t+2}^{i k}\right)^{\beta / \nu}}{\sum_{m=1}^N \sum_{h=0}^J \mu_t^{n j, m h}\left(\dot{u}_{t+2}^{m h}\right)^{\beta / \nu}}, \\
        \dot{u}_{t+1}^{n j}=&\dot{\omega}^{n j}\left(\dot{L}_{t+1}, \dot{\Theta}_{t+1}\right)\left(\sum_{i=1}^N \sum_{k=0}^J \mu_t^{n j, i k}\left(\dot{u}_{t+2}^{i k}\right)^{\beta / \nu}\right)^\nu \\
        L_{t+1}^{n j}=&\sum_{i=1}^N \sum_{k=0}^J \mu_t^{i k, n j} L_t^{i k}
    \end{align}

    for all $j, n, i$, and $k$ at each $t$, where $\left\{\dot{\omega}^{n j}\left(\dot{L}_t, \dot{\Theta}_t\right)\right\}_{n=1, j=0, t=1}^{N, J, \infty}$ is the solution to the temporary equilibrium given $\left\{\dot{L}_t, \dot{\Theta}_t\right\}_{t=1}^{\infty}$.

\end{proposition}

%%%%%%%%%%%%%%%%%%%%%%%%%%%%%%%%%%%%%%%%%%%%%%%%%%%%%%%%%%%%%%%%%%%%%%%%%%%%%%%%%%%%%%%
%%%%%%%%%%%%%%%%%%%%%%%%%%%%%%%%%%%%%%%%%%%%%%%%%%%%%%%%%%%%%%%%%%%%%%%%%%%%%%%%%%%%%%%
\subsection{Solving for Counterfactuals}

In addition to solving for the sequential equilibrium, we are 
interested in computing counterfactuals, 
by which we mean the 
``study of how allocations change across 
space and time, relative to a baseline 
economy, given a new sequence of fundamentals, 
which we denote by $\left\{\Theta_t^{\prime}\right\}_{t=1}^{\infty}$.''



%%%%%%%%%%%%%%%%%%%%%%%%%%%%%%%%%%%%%%%%%%%%%%%%%%%%%%%%%%%%%%%%%%%%%%%%%%%%%%%%%%%%%%%
%%%%%%%%%%%%%%%%%%%%%%%%%%%%%%%%%%%%%%%%%%%%%%%%%%%%%%%%%%%%%%%%%%%%%%%%%%%%%%%%%%%%%%%
%%%%%%%%%%%%%%%%%%%%%%%%%%%%%%%%%%%%%%%%%%%%%%%%%%%%%%%%%%%%%%%%%%%%%%%%%%%%%%%%%%%%%%%
\section{Derivations}

%%%%%%%%%%%%%%%%%%%%%%%%%%%%%%%%%%%%%%%%%%%%%%%%%%%%%%%%%%%%%%%%%%%%%%%%%%%%%%%%%%%%%%%
%%%%%%%%%%%%%%%%%%%%%%%%%%%%%%%%%%%%%%%%%%%%%%%%%%%%%%%%%%%%%%%%%%%%%%%%%%%%%%%%%%%%%%%

\subsection{Derivation of $V_t^{n j}$}
\label{sec:V_t_nj_exp}

This is a derivation of \autoref{eq:V_t_nj_exp}.

As a quick starting point, note that we know the 
value of $U\left(C_t^{n j}\right)$, so 
the uncertainty is over subsequent periods:

\begin{align}
    V_t^{n j} \equiv &\mathbb{E}\left[\mathrm{v}_t^{n j}\right] \\
    = &\mathbb{E}\left[U\left(C_t^{n j}\right)+\max _{\{i, k\}_{i=1, k=0}^{N, J}}\left\{\beta E\left[\mathrm{v}_{t+1}^{i k}\right]-\tau^{n j, i k}+\nu \epsilon_t^{i k}\right\}\right] \\
    = &U\left(C_t^{n j}\right)+ \mathbb{E}\left[\max _{\{i, k\}_{i=1, k=0}^{N, J}}\left\{\beta E\left[\mathrm{v}_{t+1}^{i k}\right]-\tau^{n j, i k}+\nu \epsilon_t^{i k}\right\}\right]
\end{align}

Denote the uncertain term as:

\begin{align}
    \Phi_t^{n j} \equiv \mathbb{E}\left[\max _{\{i, k\}_{i=1, k=0}^{N, J}}\left\{\beta E\left[\mathrm{v}_{t+1}^{i k}\right]-\tau^{n j, i k}+\nu \epsilon_t^{i k}\right\}\right]
\end{align}

Recall that we assumed that the idiosyncratic shock $\epsilon$ is i.i.d. over time
and distributed Type-I Extreme Value with zero mean. Then, the 
CDF of $\epsilon$ is:

\begin{align}
    F(\epsilon)=\exp (-\exp (-\epsilon-\bar{\gamma}))
\end{align}

where 

\begin{align}
    \bar{\gamma} \equiv \int_{-\infty}^{\infty} x \exp (-x-\exp (-x)) d x
\end{align}

is Euler's constant and 

\begin{align}
    f(\epsilon)=\partial F / \partial \epsilon
\end{align}

Denote

\begin{align}
    \bar{\epsilon}_t^{i k, m h} \equiv \frac{\beta\left(V_{t+1}^{i k}-V_{t+1}^{m h}\right)-\left(\tau^{n j, i k}-\tau^{n j, m h}\right)}{\nu} \label{eq:bar_epsilon_exp}
\end{align}

Then notice that:

\begin{align}
    &\Phi_t^{n j} \\
    \equiv &\mathbb{E}\left[\max _{\{i, k\}_{i=1, k=0}^{N, J}}\left\{\beta E\left[\mathrm{v}_{t+1}^{i k}\right]-\tau^{n j, i k}+\nu \epsilon_t^{i k}\right\}\right] \\
    = &\mathbb{E}\left[\max _{\{i, k\}_{i=1, k=0}^{N, J}} \left\{ \beta V_{t+1}^{i k}-\tau^{n j, i k}+\nu \epsilon_t^{i k} \right\}\right] && \text{by \eqref{eq:V_t_nj_init}} \\
    = &\mathbb{E}\left[ \sum_{i = 1}^N \sum_{k = 0}^J \left(\beta V_{t+1}^{i k}-\tau^{n j, i k}+\nu \epsilon_t^{i k}\right) \mathbf{1}\left\{ \underset{{\{m, h\}_{m=1, h=0}^{N, J}}}{\text{argmax}} \left\{ \beta V_{t+1}^{m h}-\tau^{n j, m h}+\nu \epsilon_t^{m h} \right\} = i k \right\} \right] \\
    = &\mathbb{E}\left[ \sum_{i = 1}^N \sum_{k = 0}^J \left(\beta V_{t+1}^{i k}-\tau^{n j, i k}+\nu \epsilon_t^{i k}\right) \prod_{m h \neq i k} \mathbf{1}\left\{ \beta V_{t+1}^{m h}-\tau^{n j, m h}+\nu \epsilon_t^{m h} < \beta V_{t+1}^{i k}-\tau^{n j, i k}+\nu \epsilon_t^{i k} \right\} \right] \\
    = &\sum_{i=1}^N \sum_{k=0}^J \mathbb{E}\left[ \left(\beta V_{t+1}^{i k}-\tau^{n j, i k}+\nu \epsilon_t^{i k}\right) \prod_{m h \neq i k} \mathbf{1}\left\{ \beta V_{t+1}^{m h}-\tau^{n j, m h}+\nu \epsilon_t^{m h} < \beta V_{t+1}^{i k}-\tau^{n j, i k}+\nu \epsilon_t^{i k} \right\} \right] \\
    = &\sum_{i=1}^N \sum_{k=0}^J \mathbb{E}\left[ \mathbb{E} \left[ \left(\beta V_{t+1}^{i k}-\tau^{n j, i k}+\nu \epsilon_t^{i k}\right) \prod_{m h \neq i k} \mathbf{1}\left\{ \beta V_{t+1}^{m h}-\tau^{n j, m h}+\nu \epsilon_t^{m h} < \beta V_{t+1}^{i k}-\tau^{n j, i k}+\nu \epsilon_t^{i k} \right\} \mid \epsilon_t^{i k} \right] \right] && \text{by LIE} \\
    = &\sum_{i=1}^N \sum_{k=0}^J \mathbb{E}\left[ \left(\beta V_{t+1}^{i k}-\tau^{n j, i k}+\nu \epsilon_t^{i k}\right) \prod_{m h \neq i k} \mathbb{E} \left[ \mathbf{1}\left\{ \beta V_{t+1}^{m h}-\tau^{n j, m h}+\nu \epsilon_t^{m h} < \beta V_{t+1}^{i k}-\tau^{n j, i k}+\nu \epsilon_t^{i k} \right\} \mid \epsilon_t^{i k} \right] \right]  && \text{by iid}\\
    = &\sum_{i=1}^N \sum_{k=0}^J \mathbb{E}\left[ \left(\beta V_{t+1}^{i k}-\tau^{n j, i k}+\nu \epsilon_t^{i k}\right) \prod_{m h \neq i k} \Pr \left(\beta V_{t+1}^{m h}-\tau^{n j, m h}+\nu \epsilon_t^{m h} < \beta V_{t+1}^{i k}-\tau^{n j, i k}+\nu \epsilon_t^{i k} \mid \epsilon_t^{i k} \right) \right] && \parbox[t]{1.5cm}{\raggedright binary variable property}
\end{align}

Now, notice that 

\begin{align}
    &\Pr\left(\beta V_{t+1}^{m h}-\tau^{n j, m h}+\nu \epsilon_t^{m h}<\beta V_{t+1}^{i k}-\tau^{n j, i k}+\nu \epsilon_t^{i k} \mid \epsilon_t^{i k}\right) \\
    = &\Pr\left(\epsilon_t^{m h} < \epsilon_t^{i k}+\frac{\beta V_{t+1}^{i k}-\tau^{n j, i k}-\beta V_{t+1}^{m h}+\tau^{n j, m h}}{\nu} \mid \epsilon_t^{i k}\right) \\
    = &\Pr\left(\epsilon_t^{m h} < \epsilon_t^{i k}+\bar{\epsilon}_t^{i k, m h} \mid \epsilon_t^{i k}\right) && \text{by \eqref{eq:bar_epsilon_exp}} \\
    = &F\left(\epsilon_t^{i k}+\bar{\epsilon}_t^{i k, m h}\right) \\
\end{align}

Thus,

\begin{align}
    \Phi_t^{n j} = &\sum_{i=1}^N \sum_{k=0}^J \mathbb{E}\left[ \left(\beta V_{t+1}^{i k}-\tau^{n j, i k}+\nu \epsilon_t^{i k}\right) \prod_{m h \neq i k} F\left(\epsilon_t^{i k}+\bar{\epsilon}_t^{i k, m h}\right) \right] \\
     = &\sum_{i=1}^N \sum_{k=0}^J \int_{-\infty}^{\infty}\left(\beta V_{t+1}^{i k}-\tau^{n j, i k}+\nu \epsilon_t^{i k}\right) f\left(\epsilon_t^{i k}\right) \prod_{m h \neq i k} F\left(\bar{\epsilon}_t^{i k, m h}+\epsilon_t^{i k}\right) d \epsilon_t^{i k}
\end{align}

From there, we can note that

\begin{align}
    f\left(\epsilon_t^{i k}\right) = &\exp ( - \epsilon_t^{i k} - \bar{\gamma}) \exp \left( -\exp \left( - \epsilon_t^{i k} - \bar{\gamma} \right) \right) \\
    F\left(\bar{\epsilon}_t^{i k, m h}+\epsilon_t^{i k}\right)= &\exp \left[-\exp \left(-\left[\bar{\epsilon}_t^{i k, m h}+\epsilon_t^{i k}+\bar{\gamma}\right]\right)\right]
\end{align}

Thus,

\begin{align}
    &f\left(\epsilon_t^{i k}\right) \prod_{m h \neq i k} F\left(\bar{\epsilon}_t^{i k, m h}+\epsilon_t^{i k}\right) \\
    = &\exp ( - \epsilon_t^{i k} - \bar{\gamma}) \exp \left( -\exp \left( - \epsilon_t^{i k} - \bar{\gamma} \right) \right)
        \exp \left[-\sum_{(m, h) \neq(i, k)} \exp \left(-\left(\bar{\epsilon}_t^{i k, m h}+\epsilon_t^{i k}+\bar{\gamma}\right)\right)\right] \\
    = & \exp ( - \epsilon_t^{i k} - \bar{\gamma}) \exp \left[ -\exp \left( - \epsilon_t^{i k} - \bar{\gamma} \right) - \sum_{(m, h) \neq(i, k)} \exp \left(-\left(\bar{\epsilon}_t^{i k, m h}+\epsilon_t^{i k}+\bar{\gamma}\right)\right) \right] \\
    = & \exp ( - \epsilon_t^{i k} - \bar{\gamma}) \exp \left[  - \sum_{m=1}^N \sum_{h=0}^J \exp \left( -\bar{\epsilon}_t^{i k, m h}-\epsilon_t^{i k}-\bar{\gamma} \right) \right] && \parbox[t]{4cm}{\raggedright since $\bar{\epsilon}_t^{i k, i k} = 0$} \\
    = & \exp ( - \epsilon_t^{i k} - \bar{\gamma}) \exp \left[ - \exp(-\epsilon_t^{i k} - \bar{\gamma}) \sum_{m=1}^N \sum_{h=0}^J \exp \left( -\bar{\epsilon}_t^{i k, m h} \right) \right] 
\end{align}

Thus,

\begin{align}
    &\Phi_t^{n j} \\
    =&\sum_{i=1}^N \sum_{k=0}^J \int_{-\infty}^{\infty}\left(\beta V_{t+1}^{i k}-\tau^{n j, i k}+\nu \epsilon_t^{i k}\right) f\left(\epsilon_t^{i k}\right) \prod_{m h \neq i k} F\left(\bar{\epsilon}_t^{i k, m h}+\epsilon_t^{i k}\right) d \epsilon_t^{i k} \\
    = &\sum_{i=1}^N \sum_{k=0}^J \int_{-\infty}^{\infty}\left(\beta V_{t+1}^{i k}-\tau^{n j, i k}+\nu \epsilon_t^{i k}\right) \exp ( - \epsilon_t^{i k} - \bar{\gamma}) \exp \left[ - \exp(-\epsilon_t^{i k} - \bar{\gamma}) \sum_{m=1}^N \sum_{h=0}^J \exp \left( -\bar{\epsilon}_t^{i k, m h} \right) \right] d \epsilon_t^{i k}
\end{align}

Denote:

\begin{align}
    \lambda_t^{i k} &\equiv \log \sum_{m=1}^N \sum_{h=0}^J \exp \left(-\bar{\epsilon}_t^{i k, m h}\right) \\
    \zeta_t^{i k} &\equiv \epsilon_t^{i k}+\bar{\gamma}
\end{align}

\begin{questions}[Break]
    Putting this derivation on pause here. (The below lines 
    are directly from the paper.)
\end{questions}

Then through a change of variables, we have:

\begin{align}
    \Phi_t^{n j}=\sum_{i=1}^N \sum_{k=0}^J \int_{-\infty}^{\infty}\left(\beta V_{t+1}^{i k}-\tau^{n j i k}+\nu\left(\zeta_t^{i k}-\bar{\gamma}\right)\right) \exp \left(-\zeta_t^{i k}-\exp \left(-\left(\zeta_t^{i k}-\lambda_t^{i k}\right)\right)\right) d \zeta_t^{i k}
\end{align}

Consider an additional change of variables using:

\begin{align}
    \tilde{y}_t^{i k}=\zeta_t^{i k}-\lambda_t^{i k}
\end{align}

Then we obtain:

\begin{align}
    \Phi_t^{n j}=\sum_{i=1}^N \sum_{k=0}^J \exp \left(-\lambda_t^{i k}\right) \left(\left(\beta V_{t+1}^{i k}-\tau^{n j, i k}+\nu\left(\lambda_t^{i k}-\bar{\gamma}\right)\right)+\nu \int_{-\infty}^{\infty} \tilde{y}_t^{i k} \exp \left(-\tilde{y}_t^{i k}-\exp \left(-\tilde{y}_t^{i k}\right)\right) d \tilde{y}_t^{i k}\right)
\end{align}


\end{document}
