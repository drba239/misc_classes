\documentclass[10pt]{article}
\usepackage{amsmath}
\usepackage{amsthm}
\usepackage{amsfonts}
\usepackage{amssymb}
\usepackage{amssymb}
\usepackage{booktabs}
\setlength\parindent{0pt}
\usepackage[margin=1.2in]{geometry}
\usepackage{enumitem}
\usepackage{mathtools}
\mathtoolsset{showonlyrefs=true}
\usepackage{pdflscape}
\usepackage{xcolor}
\usepackage{hyperref}
\setcounter{tocdepth}{4}
\setcounter{secnumdepth}{4}
\usepackage[listings,skins,breakable]{tcolorbox} % package for colored boxes
\usepackage{etoolbox}
\usepackage{placeins}
\usepackage{tikz}
\usepackage{color}  % Allows for color customization
\usepackage{subcaption}
\usepackage[utf8]{inputenc}


% Make it so that the bottom page of a 
% book section doesn't have weird spacing
\raggedbottom


% Define custom colors
% You can-redo these later, they're not being used 
% for anything as of 5/12/24
\definecolor{codegreen}{rgb}{0,0.6,0}
\definecolor{codegray}{rgb}{0.5,0.5,0.5}
\definecolor{codepurple}{rgb}{0.58,0,0.82}
\definecolor{backcolour}{rgb}{0.95,0.95,0.92}

% You can-redo the lstlisting style later, it's not being used 
% for anything as of 5/12/24

% Define the lstlisting style
\lstdefinestyle{mystyle}{
    backgroundcolor=\color{backcolour},   
    commentstyle=\color{codegreen},
    keywordstyle=\color{magenta},
    numberstyle=\tiny\color{codegray},
    stringstyle=\color{codepurple},
    basicstyle=\ttfamily\footnotesize,
    breakatwhitespace=false,         
    breaklines=true,                 
    captionpos=b,                    
    keepspaces=true,                 
    numbers=left,                    
    numbersep=5pt,                  
    showspaces=false,                
    showstringspaces=false,
    showtabs=false,                  
    tabsize=2
}
\lstset{style=mystyle}


% Set the length of \parskip to add a line between paragraphs
\setlength{\parskip}{1em}


% Set the second level of itemize to use \circ as the bullet point
\setlist[itemize,2]{label={$\circ$}}

% Define symbols
\DeclareMathSymbol{\Perp}{\mathrel}{symbols}{"3F}
\newcommand\barbelow[1]{\stackunder[1.2pt]{$#1$}{\rule{.8ex}{.075ex}}}
\newcommand{\succprec}{\mathrel{\mathpalette\succ@prec{\succ\prec}}}
\newcommand{\precsucc}{\mathrel{\mathpalette\succ@prec{\prec\succ}}}


\newcounter{example}[section] % Reset example counter at each new section
\renewcommand{\theexample}{\thesection.\arabic{example}} % Format the example number as section.number

\newenvironment{example}
  {% Begin environment
   \refstepcounter{example}% Step counter and allow for labeling
   \noindent\textbf{Example \theexample.} % Display the example number
  }
  {% End environment
   \par\noindent\hfill\textit{End of Example.}\par
  }



% deeper section command
% This will let you go one level deeper than whatever section level you're on.
\makeatletter
\newcommand{\deepersection}[1]{%
  \ifnum\value{subparagraph}>0
    % Already at the deepest standard level (\subparagraph), cannot go deeper
    \subparagraph{#1}
  \else
    \ifnum\value{paragraph}>0
      \subparagraph{#1}
    \else
      \ifnum\value{subsubsection}>0
        \paragraph{#1}
      \else
        \ifnum\value{subsection}>0
          \subsubsection{#1}
        \else
          \ifnum\value{section}>0
            \subsection{#1}
          \else
            \section{#1}
          \fi
        \fi
      \fi
    \fi
  \fi
}
\makeatother


% same section command
% This will let create a section at the same level as whatever section level you're on.
\makeatletter
\newcommand{\samesection}[1]{%
  \ifnum\value{subparagraph}>0
    \subparagraph{#1}
  \else
    \ifnum\value{paragraph}>0
      \paragraph{#1}
    \else
      \ifnum\value{subsubsection}>0
        \subsubsection{#1}
      \else
        \ifnum\value{subsection}>0
          \subsection{#1}
        \else
          \ifnum\value{section}>0
            \section{#1}
          \else
            % Default to section if outside any sectioning
            \section{#1}
          \fi
        \fi
      \fi
    \fi
  \fi
}
\makeatother

\makeatletter
\newcommand{\shallowersection}[1]{%
  \ifnum\value{subparagraph}>0
    \paragraph{#1} % From subparagraph to paragraph
  \else
    \ifnum\value{paragraph}>0
      \subsubsection{#1} % From paragraph to subsubsection
    \else
      \ifnum\value{subsubsection}>0
        \subsection{#1} % From subsubsection to subsection
      \else
        \ifnum\value{subsection}>0
          \section{#1} % From subsection to section
        \else
          \ifnum\value{section}>0
            \chapter{#1} % Assuming a document class with chapters
          \else
            \section{#1} % Default to section if somehow higher than section
          \fi
        \fi
      \fi
    \fi
  \fi
}
\makeatother



\newcounter{problemcounter}
\renewcommand{\theproblemcounter}{Q.\arabic{problemcounter}}

% Define the problem environment
\newenvironment{problem}[1][]{%
  \refstepcounter{problemcounter}%
  \if\relax\detokenize{#1}\relax
    \tcolorbox[breakable, colback=red!10, colframe=red!50, fonttitle=\bfseries, title={Problem \theproblemcounter}, arc=5mm, boxrule=0.5mm]
  \else
    \tcolorbox[breakable, colback=red!10, colframe=red!50, fonttitle=\bfseries, title={Problem \theproblemcounter: #1}, arc=5mm, boxrule=0.5mm]
    \addcontentsline{toc}{subsubsection}{\theproblemcounter: #1}%
  \fi
}{
  \endtcolorbox
}

% Define a new counter for definitions
\newcounter{definitioncounter}
\renewcommand{\thedefinitioncounter}{D.\arabic{definitioncounter}}

\newenvironment{definition}[1][]{%
  \refstepcounter{definitioncounter}%
  \if\relax\detokenize{#1}\relax
    \tcolorbox[
      breakable,
      parbox=false, % Treat content normally regarding paragraphs
      before upper={\parindent0pt \parskip7pt}, % No indentation and add space between paragraphs
      colback=blue!10,
      colframe=blue!50,
      fonttitle=\bfseries,
      title={Definition \thedefinitioncounter},
      arc=5mm,
      boxrule=0.5mm,
      before skip=10pt, % Adjust vertical space before the box
      after skip=10pt % Adjust vertical space after the box
    ]
  \else
    \tcolorbox[
      breakable,
      parbox=false,
      before upper={\parindent0pt \parskip7pt},
      colback=blue!10,
      colframe=blue!50,
      fonttitle=\bfseries,
      title={Definition \thedefinitioncounter: #1},
      arc=5mm,
      boxrule=0.5mm,
      before skip=10pt,
      after skip=10pt
    ]
  \fi
}{
  \endtcolorbox
}


% Define a new counter for theorems
\newcounter{theoremcounter}
\renewcommand{\thetheoremcounter}{T.\arabic{theoremcounter}}

\newenvironment{theorem}[1][]{%
  \refstepcounter{theoremcounter}%
  \if\relax\detokenize{#1}\relax
    \tcolorbox[
      breakable,
      parbox=false, % Treat content normally regarding paragraphs
      before upper={\parindent0pt \parskip7pt}, % No indentation and add space between paragraphs
      colback=green!10,
      colframe=green!55,
      fonttitle=\bfseries,
      title={Theorem \thetheoremcounter},
      arc=5mm,
      boxrule=0.5mm,
      before skip=10pt, % Adjust vertical space before the box
      after skip=10pt % Adjust vertical space after the box
    ]
  \else
    \tcolorbox[
      breakable,
      parbox=false,
      before upper={\parindent0pt \parskip7pt},
      colback=green!10,
      colframe=green!55,
      fonttitle=\bfseries,
      title={Theorem \thetheoremcounter: #1},
      arc=5mm,
      boxrule=0.5mm,
      before skip=10pt,
      after skip=10pt
    ]
  \fi
}{
  \endtcolorbox
}


% Define a new counter for remarks
\newcounter{remarkcounter}
\renewcommand{\theremarkcounter}{R.\arabic{remarkcounter}}

\newenvironment{remark}[1][]{%
  \refstepcounter{remarkcounter}%
  \if\relax\detokenize{#1}\relax
    \tcolorbox[
      breakable,
      parbox=false, % Treat content normally regarding paragraphs
      before upper={\parindent0pt \parskip7pt}, % No indentation and add space between paragraphs
      colback=green!10,
      colframe=green!55,
      fonttitle=\bfseries,
      title={Remark \theremarkcounter},
      arc=5mm,
      boxrule=0.5mm,
      before skip=10pt, % Adjust vertical space before the box
      after skip=10pt % Adjust vertical space after the box
    ]
  \else
    \tcolorbox[
      breakable,
      parbox=false,
      before upper={\parindent0pt \parskip7pt},
      colback=green!10,
      colframe=green!55,
      fonttitle=\bfseries,
      title={Remark \theremarkcounter: #1},
      arc=5mm,
      boxrule=0.5mm,
      before skip=10pt,
      after skip=10pt
    ]
  \fi
}{
  \endtcolorbox
}

% Define a new counter for lemmas
\newcounter{lemmacounter}
\renewcommand{\thelemmacounter}{L.\arabic{lemmacounter}}

\newenvironment{lemma}[1][]{%
  \refstepcounter{lemmacounter}%
  \if\relax\detokenize{#1}\relax
    \tcolorbox[
      breakable,
      parbox=false, % Treat content normally regarding paragraphs
      before upper={\parindent0pt \parskip7pt}, % No indentation and add space between paragraphs
      colback=green!10,
      colframe=green!55,
      fonttitle=\bfseries,
      title={Lemma \thelemmacounter},
      arc=5mm,
      boxrule=0.5mm,
      before skip=10pt, % Adjust vertical space before the box
      after skip=10pt % Adjust vertical space after the box
    ]
  \else
    \tcolorbox[
      breakable,
      parbox=false,
      before upper={\parindent0pt \parskip7pt},
      colback=green!10,
      colframe=green!55,
      fonttitle=\bfseries,
      title={Lemma \thelemmacounter: #1},
      arc=5mm,
      boxrule=0.5mm,
      before skip=10pt,
      after skip=10pt
    ]
  \fi
}{
  \endtcolorbox
}

% Define a new counter for propositions
\newcounter{propositioncounter}
\renewcommand{\thepropositioncounter}{P.\arabic{propositioncounter}}

\newenvironment{proposition}[1][]{%
  \refstepcounter{propositioncounter}%
  \if\relax\detokenize{#1}\relax
    \tcolorbox[
      breakable,
      parbox=false, % Treat content normally regarding paragraphs
      before upper={\parindent0pt \parskip7pt}, % No indentation and add space between paragraphs
      colback=green!10,
      colframe=green!55,
      fonttitle=\bfseries,
      title={Proposition \thepropositioncounter},
      arc=5mm,
      boxrule=0.5mm,
      before skip=10pt, % Adjust vertical space before the box
      after skip=10pt % Adjust vertical space after the box
    ]
  \else
    \tcolorbox[
      breakable,
      parbox=false,
      before upper={\parindent0pt \parskip7pt},
      colback=green!10,
      colframe=green!55,
      fonttitle=\bfseries,
      title={Proposition \thepropositioncounter: #1},
      arc=5mm,
      boxrule=0.5mm,
      before skip=10pt,
      after skip=10pt
    ]
  \fi
}{
  \endtcolorbox
}

%\newtheorem{proposition}[theorem]{Proposition}  % Propositions share numbering with theorems


% Define a new counter for notes
\newcounter{notescounter}
\renewcommand{\thenotescounter}{D.\arabic{notescounter}}

\newenvironment{notes}[1][]{
  \refstepcounter{notescounter}%
  \if\relax\detokenize{#1}\relax
    % If #1 is empty, set the title to "Notes"
    \tcolorbox[
      breakable,
      parbox=false, % Treat content normally regarding paragraphs
      before upper={\parindent0pt \parskip7pt}, % No indentation and add space between paragraphs
      colback=blue!10,
      colframe=blue!50,
      fonttitle=\bfseries,
      title={Notes},
      arc=5mm,
      boxrule=0.5mm,
      before skip=10pt, % Adjust vertical space before the box
      after skip=10pt % Adjust vertical space after the box
    ]
  \else
    % If #1 is not empty, use it as the title
    \tcolorbox[
      breakable,
      parbox=false,
      before upper={\parindent0pt \parskip7pt},
      colback=blue!10,
      colframe=blue!50,
      fonttitle=\bfseries,
      title={#1}, % Use provided title instead of default
      arc=5mm,
      boxrule=0.5mm,
      before skip=10pt,
      after skip=10pt
    ]
  \fi
}{
  \endtcolorbox
}







% Define a new counter for questions
\newcounter{questionscounter}
\renewcommand{\thequestionscounter}{D.\arabic{questionscounter}}

\newenvironment{questions}[1][]{
  \refstepcounter{questionscounter}%
  \if\relax\detokenize{#1}\relax
    % If #1 is empty, set the title to "Questions"
    \tcolorbox[
      breakable,
      parbox=false, % Treat content normally regarding paragraphs
      before upper={\parindent0pt \parskip7pt}, % No indentation and add space between paragraphs
      colback=red!10,
      colframe=red!50,
      fonttitle=\bfseries,
      title={Questions},
      arc=5mm,
      boxrule=0.5mm,
      before skip=10pt, % Adjust vertical space before the box
      after skip=10pt % Adjust vertical space after the box
    ]
  \else
    % If #1 is not empty, use it as the title
    \tcolorbox[
      breakable,
      parbox=false,
      before upper={\parindent0pt \parskip7pt},
      colback=red!10,
      colframe=red!50,
      fonttitle=\bfseries,
      title={#1}, % Use provided title instead of default
      arc=5mm,
      boxrule=0.5mm,
      before skip=10pt,
      after skip=10pt
    ]
  \fi
}{
  \endtcolorbox
}






% Define a new counter for overview
\newcounter{overviewcounter}
\renewcommand{\theoverviewcounter}{D.\arabic{overviewcounter}}

\newenvironment{overview}[1][]{%
  \refstepcounter{overviewcounter}%
  \if\relax\detokenize{#1}\relax
    \tcolorbox[
      breakable,
      parbox=false, % Treat content normally regarding paragraphs
      before upper={\parindent0pt \parskip7pt}, % No indentation and add space between paragraphs
      colback=green!10,
      colframe=green!55,
      fonttitle=\bfseries,
      title={Overview},
      arc=5mm,
      boxrule=0.5mm,
      before skip=10pt, % Adjust vertical space before the box
      after skip=10pt % Adjust vertical space after the box
    ]
  \else
    \tcolorbox[
      breakable,
      parbox=false,
      before upper={\parindent0pt \parskip7pt},
      colback=green!10,
      colframe=green!55,
      fonttitle=\bfseries,
      title={Overview},
      arc=5mm,
      boxrule=0.5mm,
      before skip=10pt,
      after skip=10pt
    ]
  \fi
}{
  \endtcolorbox
}



% Add a line after paragraph header
\makeatletter
\renewcommand\paragraph{\@startsection{paragraph}{4}{\z@}%
            {-3.25ex \@plus -1ex \@minus -.2ex}%
            {1.5ex \@plus .2ex}%
            {\normalfont\normalsize\bfseries}}
\makeatother


% Taking away line before and after align
\BeforeBeginEnvironment{align}{\vspace{-\parskip}}
\AfterEndEnvironment{align}{\vskip0pt plus 2pt}

\usepackage{changepage}

\title{Monte Paper Notes}

\author{Dylan Baker}
\date{October 2024}

\begin{document}
\maketitle

\tableofcontents

\section{Terms}



\begin{itemize}
    \item $\omega$: Worker
    \item $n$: Location where the worker lives and consumes
    \item $i$: Location where the worker works
    \item $C_{n \omega}$: Final good consumption
        \begin{align}
            C_n=\left[\sum_{i \in N} \int_0^{M_i} c_{n i}(j)^\rho d j\right]^{\frac{1}{\rho}}, \quad \sigma=\frac{1}{1-\rho}>1 \label{eq:good_consumption_index}
        \end{align}
        \begin{itemize}
            \item $\sigma$: The elasticity of substitution between different varieties of goods
        \end{itemize}
    \item $c_{n i}(j)$: Consumption in location $n$ of each variety, $j$, sourced from location $i$
    \item $H_{n \omega}$: Residential land use
    \item $b_{n i \omega}$: Idiosyncratic amenities shock
        \begin{itemize}
            \item This term ``captures the idea that individual workers can have
            idiosyncratic reasons for living and working in different locations.''
            \item Drawn from an independent Fréchet distribution
                \begin{align}
                    G_{n i}(b)=e^{-B_{n i} b^{-\epsilon}}, \quad B_{n i}>0, \epsilon>1
                \end{align}
                \begin{itemize}
                    \item $B_{n i}$: This is the ``scale parameter,'' which ``determines the average amenities from living in location $n$ and working in location $i$.''
                    \item $\epsilon$: This is the ``shape parameter,'' which ``controls the dispersion of amenities.''
                \end{itemize}
        \end{itemize}
    \item $\kappa_{n i}$: Iceberg commuting cost
        \begin{itemize}
            \item $\kappa_{n i} \in[1, \infty)$
        \end{itemize}
    \item $U_{n i \omega}$: Utility
        \begin{align}
            U_{n i \omega}=\frac{b_{n i \omega}}{\kappa_{n i}}\left(\frac{C_{n \omega}}{\alpha}\right)^\alpha\left(\frac{H_{n \omega}}{1-\alpha}\right)^{1-\alpha}
        \end{align}
        \begin{itemize}
            \item $\alpha$: The share of income spent on goods consumption
            \item $(1-\alpha)$: The share of income spent on housing
        \end{itemize}
    \item $X_n$: Aggregate expenditure in location $n$
    \item $P_n$: The price index dual to \eqref{eq:good_consumption_index}
    \item $p_{n i}(j)$: The  ```cost inclusive of freight' price of a variety $j$ produced in location $i$ and consumed in location $n$''
    \item $\bar{v}_n$: The average income of residents of $n$ (some of whom may work outside of $n$)
    \item $R_n$: The measure of residents in location $n$
    \item $Q_n$: Land price in location $n$
    \item $H_n$: The supply of land in location $n$
    \item $x_i(j)$: The number of variety $j$ produced in location $i$
    \item $F$: The fixed cost of producing a variety $j$
    \item $l_i(j)$: The labor input required to produce an amount of variety $j$, $x_i(j)$, in location $i$
    \item $d_{ni}$: The trade cost of shipping from location $i$ to location $n$
    \item $M_i$: The measure of varieties produced in location $i$
    \item $L_i$: The measure of workers in location $i$
    \item $G_{ni}(U)$: The CDF of the indirect utility of living in location $n$ and working in location $i$
        \begin{align}
            G_{n i}(U)=e^{-\Psi_{n i} U^{-\epsilon}}
        \end{align}
        where 
        \begin{align}
            \Psi_{n i}=B_{n i}\left(\kappa_{n i} P_n^\alpha Q_n^{1-\alpha}\right)^{-\epsilon} w_i^\epsilon
        \end{align}
    \item $\lambda_{ni}$: The probability that a worker lives in location $n$ and works in location $i$
        \begin{align}
            \lambda_{n i}=\frac{B_{n i}\left(\kappa_{n i} P_n^\alpha Q_n^{1-\alpha}\right)^{-\epsilon} w_i^\epsilon}{\sum_{r \in N} \sum_{s \in N} B_{r s}\left(\kappa_{r s} P_r^\alpha Q_r^{1-\alpha}\right)^{-\epsilon} w_s^\epsilon} \equiv \frac{\Phi_{n i}}{\Phi}
        \end{align}
    \item $\lambda_n^R$: The probability that a worker lives in location $n$ and works in location $i$
        \begin{align}
            \lambda_n^R=\frac{R_n}{\bar{L}}=\sum_{i \in N} \lambda_{n i}=\sum_{i \in N} \frac{\Phi_{n i}}{\Phi}
        \end{align}
    \item $\lambda_i^L$: The probability that a worker works in location $i$
        \begin{align}
            \lambda_i^L=\frac{L_n}{\bar{L}}=\sum_{n \in N} \lambda_{n i}=\sum_{n \in N} \frac{\Phi_{n i}}{\Phi}
        \end{align}
    \item $\lambda_{ni\mid n}^R$: The probability that a worker living in $n$ commutes to $i$.
        \begin{align}
            \lambda_{n i \mid n}^R \equiv \frac{\lambda_{n i}}{\lambda_n^R}=\frac{B_{n i}\left(w_i / \kappa_{n i}\right)^\epsilon}{\sum_{s \in N} B_{n s}\left(w_s / \kappa_{n s}\right)^\epsilon}
        \end{align}
\end{itemize}


\section{The Model}

\subsection{Preferences and Endowments}

\subsubsection{Preferences}

The preferences of a worker who lives in
location $n$ and works in location $i$ 
is given by the following utility function
of the Cobb-Douglas form:

\begin{align}
    U_{n i \omega}=\frac{b_{n i \omega}}{\kappa_{n i}}\left(\frac{C_{n \omega}}{\alpha}\right)^\alpha\left(\frac{H_{n \omega}}{1-\alpha}\right)^{1-\alpha} \label{eq:utility_exp}
\end{align}

Idiosyncratic amenities 
are drawn from 
an independent Fréchet distribution:

\begin{align}
    G_{n i}(b)=e^{-B_{n i} b^{-\epsilon}}, \quad B_{n i}>0, \epsilon>1
\end{align}

\subsubsection{Good Consumption Index}

The good consumption index is given the form:

\begin{align}
    C_n=\left[\sum_{i \in N} \int_0^{M_i} c_{n i}(j)^\rho d j\right]^{\frac{1}{\rho}}, \quad \sigma=\frac{1}{1-\rho}>1
\end{align}

``The goods consumption index in location $n$ is a constant elasticity of 
substitution (CES) function of consumption of a continuum of tradable 
varieties sourced from each location $i$.''

Utility maximization 
will give that ``the equilibrium consumption 
in location $n$ of each variety sourced from 
location $i$ 
is given by'':

\begin{align}
    c_{n i}(j)=\alpha X_n P_n^{\sigma-1} p_{n i}(j)^{-\sigma} \label{eq:good_nij_consumption}
\end{align}

See \autoref{sec:good_nij_consumption} for derivation.

%%%%%%%%%%%%%%%%%%%%%%%%%%%%%%%%%%%%%%%%%%%%%%%%%%%%%%%%%%%%%%%%%%%%%%%%%%%%%%%%%%%%%%%
\subsubsection{Land and Local Consumption}

``We assume that this land is owned by immobile landlords,
who receive worker expenditure on residential land as income, and consume only
goods where they live.''

From there, we get the expression

\begin{align}
    P_n C_n=\alpha \bar{v}_n R_n+(1-\alpha) \bar{v}_n R_n=\bar{v}_n R_n \label{eq:land_and_local_consumption}
\end{align}

which says that the 
total expenditure on goods in location $n$, $P_n C_n$
is equal to the total labor income of 
residents in location $n$, $\bar{v}_n R_n$.

\begin{questions}
    Should I be saying ``total labor income'' or just ``total income''
    for $\bar{v}_n R_n$?
\end{questions}

The middle term can be read as 
residents' total spending on goods in $n$,
$\alpha \bar{v}_n R_n$, plus 
residents' total spending on land in $n$,
$(1-\alpha) \bar{v}_n R_n$.

We can also get the following expression:

\begin{align}
    Q_n=(1-\alpha) \frac{\bar{v}_n R_n}{H_n} \label{eq:land_market_clearing2}
\end{align}

which says that 
the land price in location $n$, $Q_n$,
is equal to the total spending on land in $n$ 
divided by the supply of land in $n$.

This follows from the land market clearing condition:

\begin{align}
    \underbrace{Q_n \times H_n}_{\text {price } \times \text { quantity of land }}=\underbrace{(1-\alpha) \bar{v}_n R_n}_{\text {total rent paid by residents }} \label{eq:land_market_clearing}
\end{align}

and is useful, because it allows us to express 
rent as a function of the supply of land.

\begin{questions}
    Why do we only say that it's a function of the supply of land?
    It seems to be a function of several things, no?
\end{questions}


%%%%%%%%%%%%%%%%%%%%%%%%%%%%%%%%%%%%%%%%%%%%%%%%%%%%%%%%%%%%%%%%%%%%%%%%%%%%%%%%%%%%%%%
%%%%%%%%%%%%%%%%%%%%%%%%%%%%%%%%%%%%%%%%%%%%%%%%%%%%%%%%%%%%%%%%%%%%%%%%%%%%%%%%%%%%%%%
%%%%%%%%%%%%%%%%%%%%%%%%%%%%%%%%%%%%%%%%%%%%%%%%%%%%%%%%%%%%%%%%%%%%%%%%%%%%%%%%%%%%%%%

\subsection{Production}

\subsubsection{Labor Requirement for Production}

Firms produce tradable varieties under 
monopolistic competition
and increasing returns to scale
using labor as the lone input. 

To produce a variety, firms incur 
a fixed cost, $F$, as well as a 
variable cost
that is determined by the inverse 
of local productivity, $A_i$: $x_i(j)/A_i$.

Thus, the total amount of labor, $l_i(j)$, required to produce
$x_i(j)$ units of variety $j$ in location $i$ is
given by:

\begin{align}
    l_i(j)=F+ \frac{x_i(j)}{A_i} \label{eq:labor_requirement}
\end{align}

%%%%%%%%%%%%%%%%%%%%%%%%%%%%%%%%%%%%%%%%%%%%%%%%%%%%%%%%%%%%%%%%%%%%%%%%%%%%%%%%%%%%%%%
\subsubsection{Price: Constant Markup over Marginal Cost}

Profit maximization  
gives us that 
the equilibrium prices are a 
constant markup over marginal cost:

\begin{align}
    p_{n i}(j)=\left(\frac{\sigma}{\sigma-1}\right) \frac{d_{n i} w_i}{A_i} \label{eq:price_ni_1}
\end{align}

The derivation is given in \autoref{sec:price_ni_1}.

Notice also that $p_{n i}(j)$ is the 
same for all varieties produced in location $i$ and 
consumed in location $n$, i.e.,
there is no $j$ on the RHS. 

%%%%%%%%%%%%%%%%%%%%%%%%%%%%%%%%%%%%%%%%%%%%%%%%%%%%%%%%%%%%%%%%%%%%%%%%%%%%%%%%%%%%%%%
\subsubsection{Equilibrium Output of $j$ in $i$}

If we then additionally note zero profits, we get
an expression for equilibrium output of each variety
in each location:

\begin{align}
    x_i(j)=A_i F(\sigma-1) \label{eq:eq_output_xij}
\end{align}

The derivation is given in \autoref{sec:eq_output_xij}.

%%%%%%%%%%%%%%%%%%%%%%%%%%%%%%%%%%%%%%%%%%%%%%%%%%%%%%%%%%%%%%%%%%%%%%%%%%%%%%%%%%%%%%%
\subsubsection{Measure of Produced Varieties}

\eqref{eq:eq_output_xij} combined with labor market 
clearing gives us that 
the total measure of produced varieties in region $i$, $M_i$,
is proportional to the 
measure of employed workers, $L_i$:

\begin{align}
    M_i = \frac{L_i}{\sigma F} \label{eq:m_i_1}
\end{align}

The derivation is given in \autoref{sec:m_i_1}.


%%%%%%%%%%%%%%%%%%%%%%%%%%%%%%%%%%%%%%%%%%%%%%%%%%%%%%%%%%%%%%%%%%%%%%%%%%%%%%%%%%%%%%%
%%%%%%%%%%%%%%%%%%%%%%%%%%%%%%%%%%%%%%%%%%%%%%%%%%%%%%%%%%%%%%%%%%%%%%%%%%%%%%%%%%%%%%%
%%%%%%%%%%%%%%%%%%%%%%%%%%%%%%%%%%%%%%%%%%%%%%%%%%%%%%%%%%%%%%%%%%%%%%%%%%%%%%%%%%%%%%%

\subsection{Goods Trade}

\subsubsection{Share of $n$'s Expenditure on $i$'s Goods}

``The model implies a gravity equation 
for bilateral trade between locations.''

``Using the CES expenditure function, the 
equilibrium pricing rule, and the measure 
of firms, the 
share of location $n$'s expenditure on 
good's produced in location $i$ is:''

\begin{align}
    \pi_{n i}=\frac{M_i p_{n i}^{1-\sigma}}{\sum_{k \in N} M_k p_{n k}^{1-\sigma}}=\frac{L_i\left(d_{n i} w_i / A_i\right)^{1-\sigma}}{\sum_{k \in N} L_k\left(d_{n k} w_k / A_k\right)^{1-\sigma}} \label{eq:share_ni_1}
\end{align}

The derivation is given in \autoref{sec:share_ni_1}.

Notice that trade between locations $n$ and $i$
depends on both their own trade costs, $d_{n i}$,
as well as the trade costs between $n$ and all other locations.

%%%%%%%%%%%%%%%%%%%%%%%%%%%%%%%%%%%%%%%%%%%%%%%%%%%%%%%%%%%%%%%%%%%%%%%%%%%%%%%%%%%%%%%
\subsubsection{Workplace Income}

``Equating revenue and
expenditure, and using zero profits, workplace income in 
each location equals total expenditure 
on goods produced in that location, namely,''

\begin{align}
    w_i L_i=\sum_{n \in N} \pi_{n i} \bar{v}_n R_n \label{eq:workplace_income}
\end{align}

The derivation is given in \autoref{sec:workplace_income}.

%%%%%%%%%%%%%%%%%%%%%%%%%%%%%%%%%%%%%%%%%%%%%%%%%%%%%%%%%%%%%%%%%%%%%%%%%%%%%%%%%%%%%%%
\subsubsection{Price Index Re-Expression}

Then, using the equilibrium pricing rule
and labor market clearing, we can 
re-express the price index dual to the consumption index as:

\begin{align}
    P_n & =\frac{\sigma}{\sigma-1}\left(\frac{1}{\sigma F}\right)^{\frac{1}{1-\sigma}}\left[\sum_{i \in N} L_i\left(d_{n i} w_i / A_i\right)^{1-\sigma}\right]^{\frac{1}{1-\sigma}} \\
    & =\frac{\sigma}{\sigma-1}\left(\frac{L_n}{\sigma F \pi_{n n}}\right)^{\frac{1}{1-\sigma}} \frac{d_{n n} w_n}{A_n} \label{eq:price_index_p_n_2}
\end{align}

The derivation is given in \autoref{sec:price_ni_2}.

%%%%%%%%%%%%%%%%%%%%%%%%%%%%%%%%%%%%%%%%%%%%%%%%%%%%%%%%%%%%%%%%%%%%%%%%%%%%%%%%%%%%%%%
%%%%%%%%%%%%%%%%%%%%%%%%%%%%%%%%%%%%%%%%%%%%%%%%%%%%%%%%%%%%%%%%%%%%%%%%%%%%%%%%%%%%%%%
%%%%%%%%%%%%%%%%%%%%%%%%%%%%%%%%%%%%%%%%%%%%%%%%%%%%%%%%%%%%%%%%%%%%%%%%%%%%%%%%%%%%%%%

\subsection{Labor Mobility and Commuting}

\subsubsection{Indirect Utility}

Workers are geographically mobile and select 
both the location of their residence and the location of 
their workplace (possibly different) to maximize their 
utility. 

Building from our baseline utility function,
\eqref{eq:utility_exp}, we can 
then get the indirect utility expression:

\begin{align}
    U_{n i \omega}=\frac{b_{n i \omega} w_i}{\kappa_{n i} P_n^\alpha Q_n^{1-\alpha}} \label{eq:indirect_utility}
\end{align}

which is the utility of a worker who lives in location $n$,
works in location $i$, has idiosyncratic amenities $b_{n i \omega}$,
and has made the consumption and housing choices that
maximize their utility.

The derivation is given in \autoref{sec:indirect_utility}.

%%%%%%%%%%%%%%%%%%%%%%%%%%%%%%%%%%%%%%%%%%%%%%%%%%%%%%%%%%%%%%%%%%%%%%%%%%%%%%%%%%%%%%%
\paragraph{Indirect Utility and the Fréchet distribution}

Notice that indirect utility expression follows a Fréchet distribution,
since $b_{n i \omega}$ is drawn from an independent Fréchet distribution
and the other terms are constant in $\omega$:

\begin{align}
    U_{n i \omega}=\underbrace{\frac{w_i}{\kappa_{n i} P_n^\alpha Q_n^{1-\alpha}}}_{\text {constant w.r.t. } \omega} \times b_{n i \omega}
\end{align}

That is, the indirect utility expression is a monotonic 
transformation of $b_{n i \omega}$ with $\frac{w_i}{\kappa_{n i} P_n^\alpha Q_n^{1-\alpha}} > 0$ and also 
follows a Fréchet distribution.

The CDF of $U$ is given by:

\begin{align}
    G_{n i}(U)=e^{-\Psi_{n i} U^{-\epsilon}} \label{eq:frechet_indirect_utility}
\end{align}

where 

\begin{align}
    \Psi_{n i}=B_{n i}\left(\kappa_{n i} P_n^\alpha Q_n^{1-\alpha}\right)^{-\epsilon} w_i^\epsilon
\end{align}

The derivation is given in \autoref{sec:frechet_indirect_utility}.

``Each worker selects the
bilateral commute that offers her the maximum utility, where the maximum of
Fréchet distributed random variables is itself Fréchet distributed.''

%%%%%%%%%%%%%%%%%%%%%%%%%%%%%%%%%%%%%%%%%%%%%%%%%%%%%%%%%%%%%%%%%%%%%%%%%%%%%%%%%%%%%%%
%%%%%%%%%%%%%%%%%%%%%%%%%%%%%%%%%%%%%%%%%%%%%%%%%%%%%%%%%%%%%%%%%%%%%%%%%%%%%%%%%%%%%%%
\subsubsection{The Probability of Living in $n$ and Working in $i$}

Based on the above-established distribution of utility, 
the probability of a worker choosing to live in $n$ 
and work in $i$ is then given by:

\begin{align}
    \lambda_{n i}=\frac{B_{n i}\left(\kappa_{n i} P_n^\alpha Q_n^{1-\alpha}\right)^{-\epsilon} w_i^\epsilon}{\sum_{r \in N} \sum_{s \in N} B_{r s}\left(\kappa_{r s} P_r^\alpha Q_r^{1-\alpha}\right)^{-\epsilon} w_s^\epsilon} \equiv \frac{\Phi_{n i}}{\Phi} \label{eq:lambda_ni_1}
\end{align}

The derivation is given in \autoref{sec:lambda_ni_1}.

Thus, the idiosyncratic shocks imply that 
workers will select different bilateral commutes even 
when faced with the same set of prices, $(P_n, Q_n, w_i)$,
commuting costs, $\kappa_{n i}$, and location characteristics, $B_{n i}$.
However, it also implies that, all else equal, 
workers are more likely to live in location $n$
and work in location $i$ when 
the consumption goods price index ($P_n$) is lower,
the land prices ($Q_n$) are lower, wages ($w_i$) are higher,
the amenities ($B_n$) are nicer, and the commuting 
costs ($\kappa_{n i}$) are lower.

%%%%%%%%%%%%%%%%%%%%%%%%%%%%%%%%%%%%%%%%%%%%%%%%%%%%%%%%%%%%%%%%%%%%%%%%%%%%%%%%%%%%%%%
\paragraph{The Probability of Living in $n$}

Summing over $i$, we get the probability of living in $n$:

\begin{align}
    \lambda_n^R=\frac{R_n}{\bar{L}}=\sum_{i \in N} \lambda_{n i}=\sum_{i \in N} \frac{\Phi_{n i}}{\Phi} \label{eq:lambda_n_R}
\end{align}

where national labor market clearing 
corresponds to $\sum_n \lambda_n^R=\sum_i \lambda_i^L=1$.

%%%%%%%%%%%%%%%%%%%%%%%%%%%%%%%%%%%%%%%%%%%%%%%%%%%%%%%%%%%%%%%%%%%%%%%%%%%%%%%%%%%%%%%

\paragraph{The Probability of Working in $i$}

Summing over $n$, we get the probability of working in $i$:

\begin{align}
    \lambda_i^L=\frac{L_n}{\bar{L}}=\sum_{n \in N} \lambda_{n i}=\sum_{n \in N} \frac{\Phi_{n i}}{\Phi} \label{eq:lambda_i_L}
\end{align}

where national labor market clearing 
corresponds to $\sum_n \lambda_n^R=\sum_i \lambda_i^L=1$.

%%%%%%%%%%%%%%%%%%%%%%%%%%%%%%%%%%%%%%%%%%%%%%%%%%%%%%%%%%%%%%%%%%%%%%%%%%%%%%%%%%%%%%%
%%%%%%%%%%%%%%%%%%%%%%%%%%%%%%%%%%%%%%%%%%%%%%%%%%%%%%%%%%%%%%%%%%%%%%%%%%%%%%%%%%%%%%%

\subsubsection{Worker Income}

\paragraph{Probability that a Worker Living in $n$ Commutes to $i$}

The probability that a worker living in $n$ commutes to $i$ is given by:

\begin{align}
    \lambda_{n i \mid n}^R \equiv \frac{\lambda_{n i}}{\lambda_n^R}=\frac{B_{n i}\left(w_i / \kappa_{n i}\right)^\epsilon}{\sum_{s \in N} B_{n s}\left(w_s / \kappa_{n s}\right)^\epsilon} \label{eq:lambda_ni_n_R}
\end{align}

\eqref{eq:lambda_ni_n_R} implies a 
``commuting gravity equation,''
where the elasticity of commuting flows
with respect to commuting costs 
is $-\epsilon$, since

\begin{align}
    \frac{d \ln \lambda_{n i \mid n}^R}{d \ln \kappa_{n i}}=-\epsilon
\end{align}

since 

\begin{align}
    \ln \left(\lambda_{n i \mid n}^R\right) = &\ln \left(\frac{B_{n i}\left(w_i / \kappa_{n i}\right)^\epsilon}{\sum_{s \in N} B_{n s}\left(w_s / \kappa_{n s}\right)^\epsilon}\right) \\
    = &\ln(B_{ni}) + \epsilon \ln(w_i) - \epsilon \ln(\kappa_{ni}) - \ln\left(\sum_{s \in N} B_{n s}\left(w_s / \kappa_{n s}\right)^\epsilon\right) \\
\end{align}

\begin{questions}
    Not totally sure why we're allowed to neglect the term in the denominator.
\end{questions}

%%%%%%%%%%%%%%%%%%%%%%%%%%%%%%%%%%%%%%%%%%%%%%%%%%%%%%%%%%%%%%%%%%%%%%%%%%%%%%%%%%%%%%%
\paragraph{Labor Market Clearing}

We then re-write the labor market clearing condition
using the $\lambda_{n i \mid n}^R$ term:

\begin{align}
    L_i=\sum_{n \in N} \lambda_{n i \mid n}^R R_n
\end{align}

which says that the 
the labor supply in $i$
must equal the sum across locations $n$ of 
the share of people living in $n$ 
and working in $i$ multiplied by
the measure of residents in $n$.

%%%%%%%%%%%%%%%%%%%%%%%%%%%%%%%%%%%%%%%%%%%%%%%%%%%%%%%%%%%%%%%%%%%%%%%%%%%%%%%%%%%%%%%
\paragraph{Expected Worker Income}

Expected worker income
conditional on living in location $n$ 
can then be written as a weighted 
average of all possible workplace regions,
where the weight is given by the conditional probability 
of working in the region:

\begin{align}
    \bar{v}_n=\sum_{i \in N} \lambda_{n i \mid n}^R w_i \label{eq:expected_worker_income}
\end{align}

Notice that since $\lambda_{n i \mid n}^R$ increases 
as commuting costs to $i$ decrease, expected 
wage is higher in regions with 
low commuting costs to areas with high wages.

%%%%%%%%%%%%%%%%%%%%%%%%%%%%%%%%%%%%%%%%%%%%%%%%%%%%%%%%%%%%%%%%%%%%%%%%%%%%%%%%%%%%%%%
\subsubsection{Expected Utility}

Population mobility implies 
that expected utility 
is the same across 
all work-home location pairs,
which is, naturally, the same as the expected 
utility for the economy as a whole:

\begin{align}
    \bar{U}=\mathbb{E}\left[U_{n i \omega}\right]=\Gamma\left(\frac{\epsilon-1}{\epsilon}\right)\left[\sum_{r \in N} \sum_{s \in N} B_{r s}\left(\kappa_{r s} P_r^\alpha Q_r^{1-\alpha}\right)^{-\epsilon} w_s^\epsilon\right]^{\frac{1}{\epsilon}} \quad \text { all } n, i \in N \label{eq:expected_utility_1}
\end{align}

The derivation is given in \autoref{sec:expected_utility_1}.

Note that even though expected utility 
is constant across locations, other 
details of the locations vary. 
Wages, for example, must vary to accommodate 
preference heterogeneity over location. 

``Workplaces and residences face upward-sloping supply 
functions for workers and residents, respectively.''
In order to attract more workers, workplaces must pay 
higher wages to raise commuters' real income and
induce additional workers
with lower idiosyncratic amenities for that 
workplace to commute there.
Similarly, residences must offer lower land prices 
to raise residents' real income and induce additional
workers to live there who have lower idiosyncratic amenities
for that residence.

Bilateral commutes with attractive characteristics, 
e.g., high workplace wage and low residence cost of living, 
``attract additional commuters with lower 
idiosyncratic amenities, until expected utility 
(taking into account idiosyncratic amenities) 
is the same across all bilateral commutes.''


%%%%%%%%%%%%%%%%%%%%%%%%%%%%%%%%%%%%%%%%%%%%%%%%%%%%%%%%%%%%%%%%%%%%%%%%%%%%%%%%%%%%%%%
%%%%%%%%%%%%%%%%%%%%%%%%%%%%%%%%%%%%%%%%%%%%%%%%%%%%%%%%%%%%%%%%%%%%%%%%%%%%%%%%%%%%%%%
%%%%%%%%%%%%%%%%%%%%%%%%%%%%%%%%%%%%%%%%%%%%%%%%%%%%%%%%%%%%%%%%%%%%%%%%%%%%%%%%%%%%%%%
\subsection{General Equilibrium}

``The general equilibrium 
of the model 
can be referenced 
by the following 
vector of six variables: 

\begin{align}
    \{w_n, \bar{v}_n, Q_n, L_n, R_n, P_n\}_n \\
\end{align}

and the scalar 

\begin{align}
    \bar{U}.
\end{align}

Given this equilibrium vector 
and scalar, all other endogenous 
variables 
of the model can be determined. ''

``This equilibrium vector 
solves the following 6 sets of equations:''

\begin{align}
    &w_i L_i=\sum_{n \in N} \pi_{n i} \bar{v}_n R_n && \parbox[t]{6cm}{\raggedright income equals expenditure, \eqref{eq:workplace_income} above} \\
    &\bar{v}_n=\sum_{i \in N} \lambda_{n i \mid n}^R w_i && \parbox[t]{6cm}{\raggedright average residential income, \eqref{eq:expected_worker_income} above} \\
    &Q_n=(1-\alpha) \frac{\bar{v}_n R_n}{H_n} && \parbox[t]{6cm}{\raggedright land market clearing, \eqref{eq:land_market_clearing2} above} \\
    &\lambda_i^L=\frac{L_n}{\bar{L}}=\sum_{n \in N} \lambda_{n i}=\sum_{n \in N} \frac{\Phi_{n i}}{\Phi} && \parbox[t]{4cm}{\raggedright workplace choice probabilities, \eqref{eq:lambda_i_L} above} \\
    &\lambda_n^R=\frac{R_n}{\bar{L}}=\sum_{i \in N} \lambda_{n i}=\sum_{i \in N} \frac{\Phi_{n i}}{\Phi} && \parbox[t]{4cm}{\raggedright residence choice probabilities, \eqref{eq:lambda_n_R} above} \\
    &P_n =\frac{\sigma}{\sigma-1}\left(\frac{L_n}{\sigma F \pi_{n n}}\right)^{\frac{1}{1-\sigma}} \frac{d_{n n} w_n}{A_n} && \parbox[t]{4cm}{\raggedright price index, \eqref{eq:price_index_p_n_2} above} \\
\end{align}

The final condition 
``needed to determine the scalar $\bar{U}$ is the labor
market clearing condition'':

\begin{align}
    \bar{L}=\sum_{n \in N} R_n=\sum_{n \in N} L_n
\end{align}


%%%%%%%%%%%%%%%%%%%%%%%%%%%%%%%%%%%%%%%%%%%%%%%%%%%%%%%%%%%%%%%%%%%%%%%%%%%%%%%%%%%%%%%
%%%%%%%%%%%%%%%%%%%%%%%%%%%%%%%%%%%%%%%%%%%%%%%%%%%%%%%%%%%%%%%%%%%%%%%%%%%%%%%%%%%%%%%
\subsection{Counterfactuals}


%%%%%%%%%%%%%%%%%%%%%%%%%%%%%%%%%%%%%%%%%%%%%%%%%%%%%%%%%%%%%%%%%%%%%%%%%%%%%%%%%%%%%%%
%%%%%%%%%%%%%%%%%%%%%%%%%%%%%%%%%%%%%%%%%%%%%%%%%%%%%%%%%%%%%%%%%%%%%%%%%%%%%%%%%%%%%%%
%%%%%%%%%%%%%%%%%%%%%%%%%%%%%%%%%%%%%%%%%%%%%%%%%%%%%%%%%%%%%%%%%%%%%%%%%%%%%%%%%%%%%%%
\section{Data and Measurement}

Data used:

\begin{itemize}
    \item Commodity Flow Survey (CFS)
        \begin{itemize}
            \item ``[D]ata on bilateral trade and distances shipped for 123 CFS regions''
        \end{itemize}
    \item American Community Survey (ACS) 2006-2010 \& US Census 1960–2000
        \begin{itemize}
            \item ``[D]ata on bilateral commuting between counties''
        \end{itemize}
    \item Bureau of Economic Analysis (BEA)
        \begin{itemize}
            \item ``[D]ata on employment and wages by workplace''
        \end{itemize}
\end{itemize}

Data on employment and commuting is used to calculate the 
``implied number of residents and their average income by county.''
[More to say here but come back to it.]

%%%%%%%%%%%%%%%%%%%%%%%%%%%%%%%%%%%%%%%%%%%%%%%%%%%%%%%%%%%%%%%%%%%%%%%%%%%%%%%%%%%%%%%
%%%%%%%%%%%%%%%%%%%%%%%%%%%%%%%%%%%%%%%%%%%%%%%%%%%%%%%%%%%%%%%%%%%%%%%%%%%%%%%%%%%%%%%

\subsection{Goods Trade}


``In the Commodity Flow Survey (CFS) data, 
we observe bilateral trade flows and
distances shipped between 123 CFS regions 
and trade deficits for each these CFS
regions''

We don't have a more granular level of data from 
the CFS. Thus, to consider the model at the county level,
``we allocate the deficit for each 
CFS region across the counties within that region according to their shares of CFS
residential income (as measured by $\bar{v}_i R_i$).''


%%%%%%%%%%%%%%%%%%%%%%%%%%%%%%%%%%%%%%%%%%%%%%%%%%%%%%%%%%%%%%%%%%%%%%%%%%%%%%%%%%%%%%%
%%%%%%%%%%%%%%%%%%%%%%%%%%%%%%%%%%%%%%%%%%%%%%%%%%%%%%%%%%%%%%%%%%%%%%%%%%%%%%%%%%%%%%%
%%%%%%%%%%%%%%%%%%%%%%%%%%%%%%%%%%%%%%%%%%%%%%%%%%%%%%%%%%%%%%%%%%%%%%%%%%%%%%%%%%%%%%%
\section{Derivations}

%%%%%%%%%%%%%%%%%%%%%%%%%%%%%%%%%%%%%%%%%%%%%%%%%%%%%%%%%%%%%%%%%%%%%%%%%%%%%%%%%%%%%%%
%%%%%%%%%%%%%%%%%%%%%%%%%%%%%%%%%%%%%%%%%%%%%%%%%%%%%%%%%%%%%%%%%%%%%%%%%%%%%%%%%%%%%%%

\subsection{Derivation of $c_{n i}(j)$ Expression} 
\label{sec:good_nij_consumption}

\subsubsection{Write the Problem}
This is the derivation of \eqref{eq:good_nij_consumption}.

When making consumption decisions surrounding
$c_{n i}(j)$, an individual is solving the following problem:

\begin{align}
    &\underset{\{c_{n i}(j)\}}{\max} C_n \\ 
    \text{s.t. } &\sum_i \int_{0}^{M_i} p_{n i}(j) c_{n i}(j) d j=\alpha X_n
\end{align}

or expanded out:

\begin{align}
    &\underset{\{c_{n i}(j)\}}{\max} \left[\sum_{i \in N} \int_0^{M_i} c_{n i}(j)^\rho d j\right]^{\frac{1}{\rho}} \\
    \text{s.t. } &\sum_i \int_{0}^{M_i} p_{n i}(j) c_{n i}(j) d j=\alpha X_n
\end{align}

where the $\alpha X_n$ term comes from the fact that 
people spend $\alpha$ of their total expenditures, $X_n$,
on goods.

\subsubsection{Lagrangian and FOCs}

The Langragian for this problem is then:

\begin{align}
    \mathcal{L} = \left[\sum_{i \in N} \int_0^{M_i} c_{n i}(j)^\rho d j\right]^{\frac{1}{\rho}} + \lambda \left(\alpha X_n - \sum_i \int_{0}^{M_i} p_{n i}(j) c_{n i}(j) d j\right)
\end{align}

which gives the relevant FOC:

\begin{align}
    \{c_{n i}(j)\} \quad \quad &\frac{\partial}{\partial c_{n i}(j)} \left[\left(\sum_{i \in N} \int_0^{M_i} c_{n i}(j)^\rho d j\right)^{\frac{1}{\rho}}\right] - \lambda p_{n i}(j) = 0 \\
    \Leftrightarrow & \frac{1}{\rho} \left[\sum_{i \in N} \int_0^{M_i} \rho c_{n i}(j)^\rho d j\right]^{\frac{1}{\rho}-1} c_{n i}(j)^{\rho-1} - \lambda p_{n i}(j) = 0 \\
    \Leftrightarrow & C_n^{1-\rho} c_{n i}(j)^{\rho-1} = \lambda p_{n i}(j) \\
    \Leftrightarrow & c_{n i}(j) = \lambda^{\frac{1}{\rho -1}} p_{ni}(j)^{\frac{1}{\rho -1}} C_n \\ 
    \Leftrightarrow & c_{n i}(j) = \lambda^{-\sigma} p_{ni}(j)^{-\sigma} C_n && \text{since $\sigma = \frac{1}{1-\rho}$} \label{eq:inter_good_nij_cons}
\end{align}


\subsubsection{Solve for $\lambda$}

From there, we can define the dual price index 

\begin{align}
    P_n \equiv \left(\sum_{i \in N} \int_0^{M_i} p_{n i}(j)^{1-\sigma} d_j \right)^{\frac{1}{1-\sigma}} \label{eq:price_index_p_n}
\end{align}

and revisit to our budget constraint:

\begin{alignat}{2}
    &\alpha X_n && = \sum_i \int_{0}^{M_i} p_{n i}(j) c_{n i}(j) d j \\
    &&&= \sum_i \int_{0}^{M_i} p_{n i}(j) \lambda^{-\sigma} p_{ni}(j)^{-\sigma} C_n d j \quad \text{by \eqref{eq:inter_good_nij_cons}} \\
    &&&= \lambda^{-\sigma} C_n \sum_i \int_{0}^{M_i} p_{n i}(j)^{1-\sigma} d j \\
    &&& = \lambda^{-\sigma} C_n P_n^{1-\sigma} \\
    \Rightarrow &\lambda^{-\sigma} &&= \frac{\alpha X_n}{C_n} P_n^{\sigma-1} \\
\end{alignat}

\subsubsection{Plug in $\lambda$}

Then, returning to \eqref{eq:inter_good_nij_cons}, we can plug in our expression for $\lambda^{-\sigma}$:

\begin{align}
    c_{n i}(j) &= \lambda^{-\sigma} p_{ni}(j)^{-\sigma} C_n \\
    &= \left(\frac{\alpha X_n}{C_n} P_n^{\sigma-1}\right) p_{ni}(j)^{-\sigma} C_n \\
    &= \alpha X_n P_n^{\sigma-1} p_{ni}(j)^{-\sigma}
\end{align}

which is what we wanted.


%%%%%%%%%%%%%%%%%%%%%%%%%%%%%%%%%%%%%%%%%%%%%%%%%%%%%%%%%%%%%%%%%%%%%%%%%%%%%%%%%%%%%%%
%%%%%%%%%%%%%%%%%%%%%%%%%%%%%%%%%%%%%%%%%%%%%%%%%%%%%%%%%%%%%%%%%%%%%%%%%%%%%%%%%%%%%%%

\subsection{Derivation of $p_{n i}(j)$ Expression}
\label{sec:price_ni_1}
This is the derivation of \eqref{eq:price_ni_1}.

\subsubsection{Marginal Cost}

We will get the expression 
for $p_{n i}(j)$ by equating 
marginal cost and marginal revenue.

First, notice that since 

\begin{align}
    l_i(j)=F+\frac{x_i(j)}{A_i}
\end{align}

cost is given by:

\begin{align}
    \underbrace{\operatorname{Cost}\left(x_i(j)\right)}_{\text {total monetary cost }}=\underbrace{w_i}_{\text {wage }}\left[F+\frac{x_i(j)}{A_i}\right] \label{eq:cost_xij}
\end{align}

However, to account for trade costs,
we must consider the variable cost of 
delivering $q_{ni}(j)$ units of variety $j$
to location $n$ from location $i$:

\begin{align}
    \underbrace{\text{Var Cost}\left(q_{ni}(j)\right)}_{\text {variable monetary cost }}=\underbrace{w_i}_{\text {wage }} \underbrace{d_{ni}}_{\text{trade costs}} \left[\frac{q_{ni}(j)}{A_i}\right] \label{eq:var_cost_qnij}
\end{align}

This is fine for marginal cost, since the fixed cost 
will disappear when we take the derivative.

Also, notice that we've defined $q_{ni}(j)$ such that:

\begin{align}
    x_i(j) = \sum_{n \in N} d_{ni} q_{ni}(j) \label{eq:xij_qnij}
\end{align}

where $d_{ni}$ is the multiplier reflecting 
how many more units of variety $j$ must be made 
to deliver $q_{ni}(j)$ units of variety $j$ to location $n$
from location $i$.

Then, marginal cost is given by:

\begin{align}
    \text{MC} = \frac{d}{d q_{ni}(j)} \text{Var Cost}\left(q_{ni}(j)\right)= \frac{w_i d_{ni}}{A_i}
\end{align}

%%%%%%%%%%%%%%%%%%%%%%%%%%%%%%%%%%%%%%%%%%%%%%%%%%%%%%%%%%%%%%%%%%%%%%%%%%%%%%%%%%%%%%%
\subsubsection{Marginal Revenue}

Under isoelastic (CES) demand, 
we can write the quantity demanded as

\begin{align}
    q_{ni}(j)=A_i p_{n i}(j)^{-\sigma} 
\end{align}

Then, revenue is given by:

\begin{align}
    \text{Rev}(p_{n i}(j)) = p_{n i}(j) q_{ni}(j) = A_i p_{n i}(j)^{1-\sigma}
\end{align}

Then marginal revenue is given by:

\begin{align}
    \text{MR}_{n i}(j) = &\frac{d \text{Rev}(p_{n i}(j))}{dp_{n i}(j)} \frac{d p_{n i}(j)}{d q_{ni}(j)} \label{eq:mr_ni_1}
\end{align}

by the chain rule.

Then notice that

\begin{align}
    \frac{d \text{Rev}(p_{n i}(j))}{dp_{n i}(j)} = A_i (1-\sigma) p_{n i}(j)^{-\sigma} \label{eq:drev_dp}
\end{align}

and 

\begin{align}
    \frac{d q_{ni}(j)}{d p_{n i}(j)} = -\sigma A_i p_{n i}(j)^{-\sigma-1}
\end{align}

Since $q$ and $p$ are inversely related,
we can write:

\begin{align}
    \frac{d p_{n i}(j)}{d q_{ni}(j)} = \left(\frac{d q_{ni}(j)}{d p_{n i}(j)}\right)^{-1} = -\frac{1}{\sigma A_i p_{n i}(j)^{\sigma+1}} \label{eq:dp_dq}
\end{align}

Then, plugging 
\eqref{eq:drev_dp} and \eqref{eq:dp_dq} into \eqref{eq:mr_ni_1}, we get:

\begin{align}
    \text{MR}_{n i}(j) = &\frac{d \text{Rev}(p_{n i}(j))}{dp_{n i}(j)} \frac{d p_{n i}(j)}{d q_{ni}(j)}  \\
    = & A_i (1-\sigma) p_{n i}(j)^{-\sigma} \left(-\frac{1}{\sigma A_i p_{n i}(j)^{\sigma+1}}\right) \\
    = & \left(\frac{\sigma - 1}{\sigma}\right) p_{n i}(j)
\end{align}

%%%%%%%%%%%%%%%%%%%%%%%%%%%%%%%%%%%%%%%%%%%%%%%%%%%%%%%%%%%%%%%%%%%%%%%%%%%%%%%%%%%%%%%

\subsubsection{Equating MR and MC}

Then, equating marginal revenue and marginal cost, we get
the desired expression:

\begin{align}
    \text{MR}_{n i}(j) = \text{MC} \\
    \Rightarrow \left(\frac{\sigma - 1}{\sigma}\right) p_{n i}(j) = \frac{d_{ni} w_i}{A_i} \\
    \Rightarrow p_{n i}(j) = \left(\frac{\sigma}{\sigma-1}\right) \frac{d_{ni} w_i}{A_i}
\end{align}

%%%%%%%%%%%%%%%%%%%%%%%%%%%%%%%%%%%%%%%%%%%%%%%%%%%%%%%%%%%%%%%%%%%%%%%%%%%%%%%%%%%%%%%
%%%%%%%%%%%%%%%%%%%%%%%%%%%%%%%%%%%%%%%%%%%%%%%%%%%%%%%%%%%%%%%%%%%%%%%%%%%%%%%%%%%%%%%

\subsection{Derivation of $x_i(j)$ Expression}
\label{sec:eq_output_xij}

This is the derivation of \eqref{eq:eq_output_xij}.

Zero profit implies that 
revenue minus variable cost equals fixed cost,
which leads to the desired equation:

\begin{align}
    &\text{Revenue} - \text{Variable Cost} = \text{Fixed Cost} \\
    \Rightarrow & \sum_n \left[p_{ni}(j) q_{ni}(j) - w_i d_{ni} \left(\frac{q_{ni}(j)}{A_i}\right)\right] = w_i F && \parbox[t]{4cm}{\raggedright By \eqref{eq:cost_xij} and \eqref{eq:var_cost_qnij}} \\
    \Rightarrow & \sum_n \left[\frac{\sigma}{\sigma-1} \frac{d_{ni} w_i}{A_i} q_{ni}(j) - w_i d_{ni} \left(\frac{q_{ni}(j)}{A_i}\right)\right] = w_i F && \parbox[t]{4cm}{\raggedright By \eqref{eq:price_ni_1}} \\
    \Rightarrow & \sum_n \left[\frac{1}{\sigma-1} \frac{d_{ni}}{A_i} q_{ni}(j)\right] = F \\ 
    \Rightarrow & \sum_n d_{ni} q_{ni}(j) = A_i F (\sigma-1) \\
    \Rightarrow & x_i(j) = A_i F (\sigma-1) && \parbox[t]{4cm}{\raggedright By \eqref{eq:xij_qnij}}
\end{align}

%%%%%%%%%%%%%%%%%%%%%%%%%%%%%%%%%%%%%%%%%%%%%%%%%%%%%%%%%%%%%%%%%%%%%%%%%%%%%%%%%%%%%%%
%%%%%%%%%%%%%%%%%%%%%%%%%%%%%%%%%%%%%%%%%%%%%%%%%%%%%%%%%%%%%%%%%%%%%%%%%%%%%%%%%%%%%%%

\subsection{Derivation of $M_i$ Expression}
\label{sec:m_i_1}

This is a derivation of \eqref{eq:m_i_1}.

\begin{align}
    L_i = &\int_{0}^{M_i} l_i(j) dj && \parbox[t]{4cm}{\raggedright Labor market clearing}\\
    = & \int_{0}^{M_i} \left[F + \frac{x_i(j)}{A_i}\right] dj && \text{By \eqref{eq:labor_requirement}} \\
    = &\int_{0}^{M_i} F + \frac{A_i F (\sigma -1)}{A_i} dj && \text{By \eqref{eq:eq_output_xij}} \\
    = &\int_{0}^{M_i} \sigma F dj \\
    = & \sigma F M_i
\end{align}

By re-arranging, we get the desired expression:

\begin{align}
    M_i = \frac{L_i}{\sigma F}
\end{align}

%%%%%%%%%%%%%%%%%%%%%%%%%%%%%%%%%%%%%%%%%%%%%%%%%%%%%%%%%%%%%%%%%%%%%%%%%%%%%%%%%%%%%%%
%%%%%%%%%%%%%%%%%%%%%%%%%%%%%%%%%%%%%%%%%%%%%%%%%%%%%%%%%%%%%%%%%%%%%%%%%%%%%%%%%%%%%%%
\subsection{Derivation of $\pi_{n i}$ Expression}
\label{sec:share_ni_1}

This is a derivation of \eqref{eq:share_ni_1}.

First, because it will be useful momentarily, note:

\begin{align}
    &P_n=\left[\sum_{k \in N} \int_0^{M_k} p_{n k}(j)^{1-\sigma} d j\right]^{\frac{1}{1-\sigma}} \\
    \Rightarrow &P_n^{1-\sigma} = \sum_{k \in N} \int_0^{M_k} p_{n k}(j)^{1-\sigma} d j \\
    \Rightarrow &P_n^{1-\sigma} = \sum_{k \in N} \int_0^{M_k} p_{nk}^{1-\sigma} d j && \parbox[t]{4cm}{\raggedright By \eqref{eq:price_ni_1}} \\
    \Rightarrow &P_n^{1-\sigma} = \sum_{k \in N} M_k p_{nk}^{1-\sigma} \label{eq:price_ind_interm_share_ni_1}
\end{align}

Now, begin with the baseline expression
for share of expenditure:

\begin{align}
    \pi_{n i}=&\frac{\int_0^{M_i} p_{n i}(j) c_{n i}(j) d j}{\sum_{k \in N} \int_0^{M_k} p_{n k}(j) c_{n k}(j) d j} \\
    = &\frac{\int_0^{M_i} p_{n i}(j) \alpha X_n P_n^{\sigma-1} p_{n i}(j)^{-\sigma} d j}{\sum_{k \in N} \int_0^{M_k} p_{n k}(j) \alpha X_n P_n^{\sigma-1} p_{n k}(j)^{-\sigma} d j} && \parbox[t]{4cm}{\raggedright By \eqref{eq:good_nij_consumption}} \\
    = &\frac{\int_0^{M_i} p_{ni}(j)^{1-\sigma} d j}{\sum_{k \in N} \int_0^{M_k} p_{nk}(j)^{1-\sigma} d j} && \parbox[t]{4cm}{\raggedright Cancel terms} \\
    = &\frac{\int_0^{M_i} p_{ni}(j)^{1-\sigma} d j}{P_n^{1-\sigma}} && \parbox[t]{4cm}{\raggedright By \eqref{eq:price_index_p_n}} \\
    = &\frac{\int_0^{M_i} p_{ni}^{1-\sigma} d j}{\sum_{k \in N} M_k p_{nk}^{1-\sigma}} && \parbox[t]{4cm}{\raggedright By \eqref{eq:price_ni_1} \& \eqref{eq:price_ind_interm_share_ni_1}} \\
    = &\frac{M_i p_{ni}^{1-\sigma}}{\sum_{k \in N} M_k p_{nk}^{1-\sigma}}
\end{align}

Thus, we now have the first desired equality:

\begin{align}
    \pi_{n i}=\frac{M_i p_{n i}^{1-\sigma}}{\sum_{k \in N} M_k p_{n k}^{1-\sigma}} \label{eq:share_ni_inter1}
\end{align}

Now, notice that:

\begin{align}
    M_i p_{ni}^{1-\sigma} = &\left(\frac{L_i}{\sigma F}\right) \left(\frac{\sigma}{\sigma-1}\right)^{1-\sigma} \frac{d_{ni} w_i}{A_i}^{1-\sigma} && \parbox[t]{4cm}{\raggedright By \eqref{eq:price_ni_1} and \eqref{eq:m_i_1}}
\end{align}

Thus, we can re-write \eqref{eq:share_ni_inter1} to be: 

\begin{align}
    \pi_{n i}=&\frac{M_i p_{n i}^{1-\sigma}}{\sum_{k \in N} M_k p_{n k}^{1-\sigma}} \\
    = &\frac{\left(\frac{L_i}{\sigma F}\right) \left(\frac{\sigma}{\sigma-1}\right)^{1-\sigma} \frac{d_{ni} w_i}{A_i}^{1-\sigma}}{\sum_{k \in N} \left(\frac{L_k}{\sigma F}\right) \left(\frac{\sigma}{\sigma-1}\right)^{1-\sigma} \frac{d_{nk} w_k}{A_k}^{1-\sigma}} \\
    = &\frac{L_i \left(\frac{d_{ni} w_i}{A_i}\right)^{1-\sigma}}{\sum_{k \in N} L_k \left(\frac{d_{nk} w_k}{A_k}\right)^{1-\sigma}}
\end{align}

which is what we wanted.

%%%%%%%%%%%%%%%%%%%%%%%%%%%%%%%%%%%%%%%%%%%%%%%%%%%%%%%%%%%%%%%%%%%%%%%%%%%%%%%%%%%%%%%
%%%%%%%%%%%%%%%%%%%%%%%%%%%%%%%%%%%%%%%%%%%%%%%%%%%%%%%%%%%%%%%%%%%%%%%%%%%%%%%%%%%%%%%

\subsection{Derivation of $w_i L_i$ Expression}
\label{sec:workplace_income}

This is a derivation of \eqref{eq:workplace_income}.

There's not really much of a derivation here. 

First, note that total labor income is expressed as:

\begin{align}
    w_i L_i
\end{align}

Total revenue from goods produced in location $i$ is given by:

\begin{align}
    \sum_{n \in N} \pi_{n i} \bar{v}_n R_n
\end{align}

There are no net profits, so 
total revenue minus total cost must be zero:

\begin{align}
    &\sum_{n \in N} \pi_{n i} \bar{v}_n R_n - w_i L_i = 0 \\
    \Rightarrow &\sum_{n \in N} \pi_{n i} \bar{v}_n R_n = w_i L_i
\end{align}

Also, note, since I briefly forgot, that the fixed 
costs of production and the costs associated with 
trade are already accounted for in the $L_i$ term.

%%%%%%%%%%%%%%%%%%%%%%%%%%%%%%%%%%%%%%%%%%%%%%%%%%%%%%%%%%%%%%%%%%%%%%%%%%%%%%%%%%%%%%%
%%%%%%%%%%%%%%%%%%%%%%%%%%%%%%%%%%%%%%%%%%%%%%%%%%%%%%%%%%%%%%%%%%%%%%%%%%%%%%%%%%%%%%%

\subsection{Derivation of $P_n$ Re-Expression}
\label{sec:price_ni_2}

This is a derivation of \eqref{eq:price_index_p_n_2}.

\begin{align}
    P_n = &\left[\sum_{i \in N} \int_0^{M_i} p_{n i}(j)^{1-\sigma} d j\right]^{\frac{1}{1-\sigma}} && \parbox[t]{4cm}{\raggedright By \eqref{eq:price_index_p_n}} \\
    = &\left[\sum_{i \in N} \int_0^{M_i} p_{n i}^{1-\sigma} d j\right]^{\frac{1}{1-\sigma}} && \parbox[t]{4cm}{\raggedright By \eqref{eq:price_ni_1}} \\
    = &\left[\sum_{i \in N} M_i p_{n i}^{1-\sigma}\right]^{\frac{1}{1-\sigma}} \\
    = &\left[\sum_{i \in N} \frac{L_i}{\sigma F} \left(\frac{\sigma}{\sigma-1}\right)^{1-\sigma} \frac{d_{ni} w_i}{A_i}^{1-\sigma}\right]^{\frac{1}{1-\sigma}} && \parbox[t]{4cm}{\raggedright By \eqref{eq:m_i_1} and \eqref{eq:price_ni_1}} \\
    = &\left(\frac{\sigma}{\sigma-1}\right) \left(\frac{1}{\sigma F}\right)^{\frac{1}{1-\sigma}} \left[\sum_{i \in N} L_i \left(\frac{d_{ni} w_i}{A_i}\right)^{1-\sigma}\right]^{\frac{1}{1-\sigma}} \\
\end{align}

which is the first desired equality. 

From there, we can note

\begin{align}
    \pi_{nn} &= \frac{L_n \left(\frac{d_{nn} w_n}{A_n}\right)^{1-\sigma}}{\sum_{i \in N} L_i \left(\frac{d_{ni} w_i}{A_i}\right)^{1-\sigma}} && \parbox[t]{4cm}{\raggedright By \eqref{eq:share_ni_1}} \label{eq:pi_nn}
\end{align}

which let's us pick back up to write:

\begin{align}
    P_n = &\left(\frac{\sigma}{\sigma-1}\right) \left(\frac{1}{\sigma F}\right)^{\frac{1}{1-\sigma}} \left[\sum_{i \in N} L_i \left(\frac{d_{ni} w_i}{A_i}\right)^{1-\sigma}\right]^{\frac{1}{1-\sigma}} \\
    = &\left(\frac{\sigma}{\sigma-1}\right) \left(\frac{1}{\sigma F}\right)^{\frac{1}{1-\sigma}} \left[\frac{L_n \left(\frac{d_{nn} w_n}{A_n}\right)^{1-\sigma}}{\pi_{nn}}\right]^{\frac{1}{1-\sigma}} && \text{By \eqref{eq:pi_nn}} \\
    = &\left(\frac{\sigma}{\sigma-1}\right) \left(\frac{L_n}{\sigma F \pi_{nn}}\right)^{\frac{1}{1-\sigma}} \frac{d_{nn} w_n}{A_n}
\end{align}

which is the second desired expression.

%%%%%%%%%%%%%%%%%%%%%%%%%%%%%%%%%%%%%%%%%%%%%%%%%%%%%%%%%%%%%%%%%%%%%%%%%%%%%%%%%%%%%%%
%%%%%%%%%%%%%%%%%%%%%%%%%%%%%%%%%%%%%%%%%%%%%%%%%%%%%%%%%%%%%%%%%%%%%%%%%%%%%%%%%%%%%%%

\subsection{Derivation of Indirect Utility}
\label{sec:indirect_utility}

This is a derivation of \eqref{eq:indirect_utility}.

Beginning with our utility function:

\begin{align}
    U_{n i \omega}=&\frac{b_{n i \omega}}{\kappa_{n i}}\left(\frac{C_{n \omega}}{\alpha}\right)^\alpha\left(\frac{H_{n \omega}}{1-\alpha}\right)^{1-\alpha}
\end{align}

Optimization gives us that workers (living in region $n$ and working in 
region $i$) spend $\alpha$
of their income, $w_i$, on consumption and $(1-\alpha)$ on housing.
The price of $C_n$ is $P_n$ and the price of $H_n$ is $Q_n$.
Thus we have the expressions:

\begin{align}
    P_n C_n = &\alpha w_i \label{eq:p_n_c_n_opt} \\
    Q_n H_n = &(1-\alpha) w_i \label{eq:q_n_h_n_opt}
\end{align}

Then we can re-write:

\begin{align}
    U_{n i \omega}=&\frac{b_{n i \omega}}{\kappa_{n i}}\left(\frac{C_{n \omega}}{\alpha}\right)^\alpha\left(\frac{H_{n \omega}}{1-\alpha}\right)^{1-\alpha} && \parbox[t]{4cm}{\raggedright by \eqref{eq:utility_exp}} \\
    = &\frac{b_{n i \omega}}{\kappa_{n i}}\left(\frac{\alpha w_i}{\alpha P_n}\right)^\alpha\left(\frac{(1-\alpha) w_i}{(1-\alpha) Q_n}\right)^{1-\alpha} && \parbox[t]{4cm}{\raggedright by \eqref{eq:p_n_c_n_opt} and \eqref{eq:q_n_h_n_opt}} \\
    = &\frac{b_{n i \omega}}{\kappa_{n i}}\left(\frac{w_i}{P_n}\right)^\alpha\left(\frac{w_i}{Q_n}\right)^{1-\alpha} \\
    = &\frac{b_{n i \omega} w_i}{\kappa_{n i} P_n^\alpha Q_n^{1-\alpha}}
\end{align}

which is the desired expression.

%%%%%%%%%%%%%%%%%%%%%%%%%%%%%%%%%%%%%%%%%%%%%%%%%%%%%%%%%%%%%%%%%%%%%%%%%%%%%%%%%%%%%%%
%%%%%%%%%%%%%%%%%%%%%%%%%%%%%%%%%%%%%%%%%%%%%%%%%%%%%%%%%%%%%%%%%%%%%%%%%%%%%%%%%%%%%%%

\subsection{Derivation of Fréchet Distribution for Indirect Utility}
\label{sec:frechet_indirect_utility}

This is a derivation of \eqref{eq:frechet_indirect_utility}.

Consider the following property of 
the Fréchet distribution:

\begin{notes}[Fréchet Distribution Property]
    If $X \sim \text{Fréchet}(B, \epsilon)$, then $F_X(x)=\exp \left[-B x^{-\epsilon}\right]$ for $x>0$.

    Then for a constant $c>0$, $cX \sim \text{Fréchet}\left(c^{-\epsilon} B, \epsilon\right)$.
    
    In other words, 

    \begin{align}
        F_{c X}(u)=\operatorname{Pr}\{c X \leq u\}=\operatorname{Pr}\{X \leq u / c\}=\exp \left[-B(u / c)^{-\epsilon}\right]=\exp \left[-\left(c^{\epsilon} B\right) u^{-\epsilon}\right]
    \end{align}

\end{notes}


Recall that

\begin{align}
    b \sim \text{Fréchet}(B, \epsilon)
\end{align}

and hence

\begin{align}
    G^b_{n i \omega}(b)=\exp \left[-B_{n i} b^{-\epsilon}\right]
\end{align}

Moreover, recall that

\begin{align}
    U_{n i \omega}=\underbrace{\frac{w_i}{\kappa_{n i} P_n^\alpha Q_n^{1-\alpha}}}_{\text {constant w.r.t. } \omega} \times b_{n i \omega}
\end{align}

Thus, employing out property of the Fréchet distribution, we have that

\begin{align}
    G_{n i}(U)=&\exp \left[-\left(\frac{w_i}{\kappa_{n i} P_n^\alpha Q_n^{1-\alpha}}\right)^{\epsilon} B_{n i} U^{-\epsilon}\right] && \parbox[t]{4cm}{\raggedright by the property above} \\
    =&\exp \left[-B_{ni} (\kappa_{n i} P_n^\alpha Q_n^{1-\alpha})^{-\epsilon} w_i^\epsilon U^{-\epsilon}\right] \\
    =&\exp \left[- \Psi_{n i} U^{-\epsilon}\right]
\end{align}

if we define

\begin{align}
    \Psi_{ni} = B_{n i} \left(\kappa_{n i} P_n^\alpha Q_n^{1-\alpha}\right)^{-\epsilon} w_i^\epsilon
\end{align}

which is the desired expression.

Also, notice that this is saying that

\begin{align}
    U_{n i \omega} \sim \text{Fréchet}\left(\Psi_{n i}, \epsilon\right) \label{eq:frechet_indirect_utility_sim}
\end{align}

%%%%%%%%%%%%%%%%%%%%%%%%%%%%%%%%%%%%%%%%%%%%%%%%%%%%%%%%%%%%%%%%%%%%%%%%%%%%%%%%%%%%%%%
%%%%%%%%%%%%%%%%%%%%%%%%%%%%%%%%%%%%%%%%%%%%%%%%%%%%%%%%%%%%%%%%%%%%%%%%%%%%%%%%%%%%%%%

\subsection{Derivation of $\lambda_{n i}$ Expression}
\label{sec:lambda_ni_1}

This is a derivation of \eqref{eq:lambda_ni_1}.

First, note that the pdf of $U_{n i \omega}$ is given by:

\begin{align}
    f_{n i}(u)=\frac{d}{d u} G_{n i}(u)= \frac{d}{d u}\left[\exp \left(-\Psi_{n i} u^{-\epsilon}\right)\right]
    =\epsilon \Psi_{n i} u^{-\epsilon-1} \exp \left[-\Psi_{n i} u^{-\epsilon}\right] \label{eq:pdf_frechet}
\end{align}

\begin{align}
    \lambda_{ni} = &\Pr\left( \underset{(mh)}{\text{argmax }} U_{m h \omega} = (ni) \right) \\
    = &\Pr\left( U_{n i \omega} > U_{m h \omega} \text{ for all } (m h) \neq (n i) \right) \\
    = & \int_0^{\infty} \underbrace{f_{n i}(u)}_{\text {pdf of } U_{n i \omega}} \operatorname{Pr}\left(U_{m h \omega}<u, \forall(m, h) \neq(n, i) \mid U_{n i \omega}=u\right) d u \\
    = & \int_0^{\infty} f_{n i}(u) \operatorname{Pr}\left(U_{m h \omega}<u, \forall(m, h) \neq(n, i) \right) d u && \parbox[t]{4cm}{\raggedright by independence} \\
    = & \int_0^{\infty} f_{n i}(u) \prod_{(m, h) \neq(n, i)} \operatorname{Pr}\left(U_{m h, \omega}<u\right) d u && \parbox[t]{4cm}{\raggedright by independence} \\
    = & \int_0^{\infty} f_{n i}(u) \prod_{(m, h) \neq(n, i)} G_{m h}(u) d u && \parbox[t]{4cm}{\raggedright by definition of CDF} \\
    = & \int_0^{\infty} \epsilon \Psi_{n i} u^{-\epsilon-1} \exp \left[-\Psi_{n i} u^{-\epsilon}\right] \prod_{(m, h) \neq(n, i)} G_{m h}(u) d u && \parbox[t]{4cm}{\raggedright by \eqref{eq:pdf_frechet}} \\
    = & \int_0^{\infty} \epsilon \Psi_{n i} u^{-\epsilon-1} \exp \left[-\Psi_{n i} u^{-\epsilon}\right] \prod_{(m, h) \neq(n, i)} \exp \left[-\Psi_{m h} u^{-\epsilon}\right] d u && \parbox[t]{4cm}{\raggedright by \eqref{eq:frechet_indirect_utility}} \\
    = &\int_0^{\infty} \epsilon \Psi_{n i} u^{-\epsilon-1} \exp [-\underbrace{\left(\Psi_{n i}+\sum_{(m, h) \neq(n, i)} \Psi_{m h}\right)}_{\sum_{r, s} \Psi_{r s}} u^{-\epsilon}] d u \\
    = &\int_0^{\infty} \epsilon \Psi_{n i} u^{-\epsilon-1} \exp \left[-\left(\sum_{r, s} \Psi_{r s}\right) u^{-\epsilon}\right] d u \\
    = &\Psi_{n i} \int_0^{\infty} \epsilon u^{-\epsilon-1} \exp \left[-\left(\sum_{r, s} \Psi_{r s}\right) u^{-\epsilon}\right] d u
\end{align}

Consider the following change of variables:

\begin{align}
    &t = u^{-\epsilon} \\
    &dt = -\epsilon u^{-\epsilon-1} du \\
    \Rightarrow &\epsilon u^{-\epsilon-1} d u=-d t
\end{align}

Notice that as $u$ goes from 0 to $\infty$, $t$ goes from $\infty$ to 0.

Thus, we can re-write:

\begin{align}
    \lambda_{ni} = &\Psi_{n i} \int_0^{\infty} \epsilon u^{-\epsilon-1} \exp \left[-\left(\sum_{r, s} \Psi_{r s}\right) u^{-\epsilon}\right] d u \\
    = &\Psi_{n i} \int_{t=\infty}^{t=0} \exp \left[- \left(\sum_{r, s} \Psi_{r s}\right) t\right](-d t)  \\
    = & \Psi_{n i} \int_{0}^{\infty} \exp \left[- \left(\sum_{r, s} \Psi_{r s}\right) t\right] d t && \parbox[t]{4cm}{\raggedright integral property} \\
    = & \frac{\Psi_{n i}}{(-\sum_{r, s} \Psi_{r s})} \int_{0}^\infty \left(-\sum_{r, s} \Psi_{r s}\right) \left[-\left(\sum_{r, s} \Psi_{r s}\right) t\right] d t && \parbox[t]{4cm}{\raggedright to make the inside of the integral equal $\frac{d}{dx}[e^{cx}]$} \\
    = & \frac{\Psi_{n i}}{\sum_{r, s} \Psi_{r s}} \left[ \lim_{t \to \infty} \exp\left(-\left(\sum_{r, s} \Psi_{r s}\right) t\right) - \exp(0)\right] && \parbox[t]{4cm}{\raggedright by the fundamental theorem of calculus} \\
    = & \frac{\Psi_{n i}}{-(\sum_{r, s} \Psi_{r s})} \left[0 - 1\right] \\
    = & \frac{\Psi_{n i}}{\sum_{r, s} \Psi_{r s}}
\end{align}

which is what we wanted.

%%%%%%%%%%%%%%%%%%%%%%%%%%%%%%%%%%%%%%%%%%%%%%%%%%%%%%%%%%%%%%%%%%%%%%%%%%%%%%%%%%%%%%%
%%%%%%%%%%%%%%%%%%%%%%%%%%%%%%%%%%%%%%%%%%%%%%%%%%%%%%%%%%%%%%%%%%%%%%%%%%%%%%%%%%%%%%%

\subsection{Derivation of $\bar{U}$ Expression}
\label{sec:expected_utility_1}

This is a derivation of \eqref{eq:expected_utility_1}.

We are going to rely on the following properties of 
the Fréchet distribution:

\begin{notes}[Fréchet Distribution Max Stability Property]
    \label{notes:frechet_max_stability}
    If\footnote{
        Note that this is under the parameterization where 
        $X \sim \text{Fréchet}(B, \epsilon)$ corresponds to $F_X(x)=\exp \left[-B x^{-\epsilon}\right]$.
        Sometimes, you will see people parameterize the Fréchet distribution
        as $X \sim \text{Fréchet}(B, \epsilon)$ corresponds to $F_X(x)=\exp \left[-\left(\frac{x}{B}\right)^{-\epsilon}\right]$.
    } \footnote{\color{red} Find a citation for this.} 

    \begin{align}
        X_i \sim \text{Fréchet}(B_i, \epsilon)
    \end{align}

    and 

    \begin{align}
        Y = \max\{X_1, X_2, \ldots, X_n\}
    \end{align}

    Then 

    \begin{align}
        Y \sim \text{Fréchet}\left(\sum_{i=1}^n B_i, \epsilon\right)
    \end{align}

\end{notes}

\begin{notes}[Fréchet Distribution Property]
    \label{notes:frechet_expected_value}
    If\footnote{
        Note that this is under the parameterization where 
        $X \sim \text{Fréchet}(B, \epsilon)$ corresponds to $F_X(x)=\exp \left[-B x^{-\epsilon}\right]$.
        Sometimes, you will see people parameterize the Fréchet distribution
        as $X \sim \text{Fréchet}(B, \epsilon)$ corresponds to $F_X(x)=\exp \left[-\left(\frac{x}{B}\right)^{-\epsilon}\right]$.
    } 
    $X \sim \text{Fréchet}(B, \epsilon)$ and 
    $\epsilon > 1$, then

    \begin{align}
        \mathbb{E}[X] = B^{\frac{1}{\epsilon}} \Gamma\left(1-\frac{1}{\epsilon}\right)
    \end{align}

    
\end{notes}

Recall from \eqref{eq:frechet_indirect_utility_sim} that:

\begin{align}
    U_{n i \omega} \sim \text{Fréchet}\left(\Psi_{n i}, \epsilon\right)
\end{align}

If workers are choosing the location that maximizes utility, 
then their selected utility follows:

\begin{align}
    \max_{n, i} U_{n i \omega} \sim \text{Fréchet}\left(\sum_{n, i} \Psi_{n i}, \epsilon\right)
\end{align}

by the \autoref{notes:frechet_max_stability} property.

From which, as long as $\epsilon > 1$, we get

\begin{align}
    \mathbb{E}[ \max_{n, i} U_{n i \omega}] &= \left(\sum_{n, i} \Psi_{n i}\right)^{\frac{1}{\epsilon}} \Gamma\left(1-\frac{1}{\epsilon}\right) \\
\end{align}

by the \autoref{notes:frechet_expected_value} property.

Additionally, it must be that

\begin{align}
    U_{n i \omega}= \mathbb{E}[ \max_{n, i} U_{n i \omega}]
\end{align}

since if any home-work pair gave strictly higher 
utility, workers would move to that pair until 
it's utility was no longer strictly higher.

Thus, 

\begin{align}
    \bar{U} =  &U_{n i \omega}  \\ 
    = &\left(\sum_{r, s} \Psi_{r s}\right)^{\frac{1}{\epsilon}} \Gamma\left(1-\frac{1}{\epsilon}\right) \\
    = &\Gamma\left(\frac{\epsilon-1}{\epsilon}\right)\left[\sum_{r \in N} \sum_{s \in N} B_{r s}\left(\kappa_{r s} P_r^\alpha Q_r^{1-\alpha}\right)^{-\epsilon} w_s^\epsilon\right]^{\frac{1}{\epsilon}} && \parbox[t]{4cm}{\raggedright by definition of $\Psi_{rs}$} \\
\end{align}

which is the desired expression.

\begin{questions}
    I'm not 100\% sure if my argument here is correct, particularly 
    the part trying to justify

    \begin{align}
        U_{n i \omega}=\mathbb{E}\left[\max _{n, i} U_{n i \omega}\right]
    \end{align}
\end{questions}



\end{document}
