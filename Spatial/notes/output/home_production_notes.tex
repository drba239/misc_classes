\documentclass[10pt]{article}
\usepackage{amsmath}
\usepackage{amsthm}
\usepackage{amsfonts}
\usepackage{amssymb}
\usepackage{amssymb}
\usepackage{booktabs}
\setlength\parindent{0pt}
\usepackage[margin=1.2in]{geometry}
\usepackage{enumitem}
\usepackage{mathtools}
\mathtoolsset{showonlyrefs=true}
\usepackage{pdflscape}
\usepackage{xcolor}
\usepackage{hyperref}
\setcounter{tocdepth}{4}
\setcounter{secnumdepth}{4}
\usepackage[listings,skins,breakable]{tcolorbox} % package for colored boxes
\usepackage{etoolbox}
\usepackage{placeins}
\usepackage{tikz}
\usepackage{color}  % Allows for color customization
\usepackage{subcaption}
\usepackage[utf8]{inputenc}


% Make it so that the bottom page of a 
% book section doesn't have weird spacing
\raggedbottom


% Define custom colors
% You can-redo these later, they're not being used 
% for anything as of 5/12/24
\definecolor{codegreen}{rgb}{0,0.6,0}
\definecolor{codegray}{rgb}{0.5,0.5,0.5}
\definecolor{codepurple}{rgb}{0.58,0,0.82}
\definecolor{backcolour}{rgb}{0.95,0.95,0.92}

% You can-redo the lstlisting style later, it's not being used 
% for anything as of 5/12/24

% Define the lstlisting style
\lstdefinestyle{mystyle}{
    backgroundcolor=\color{backcolour},   
    commentstyle=\color{codegreen},
    keywordstyle=\color{magenta},
    numberstyle=\tiny\color{codegray},
    stringstyle=\color{codepurple},
    basicstyle=\ttfamily\footnotesize,
    breakatwhitespace=false,         
    breaklines=true,                 
    captionpos=b,                    
    keepspaces=true,                 
    numbers=left,                    
    numbersep=5pt,                  
    showspaces=false,                
    showstringspaces=false,
    showtabs=false,                  
    tabsize=2
}
\lstset{style=mystyle}


% Set the length of \parskip to add a line between paragraphs
\setlength{\parskip}{1em}


% Set the second level of itemize to use \circ as the bullet point
\setlist[itemize,2]{label={$\circ$}}

% Define symbols
\DeclareMathSymbol{\Perp}{\mathrel}{symbols}{"3F}
\newcommand\barbelow[1]{\stackunder[1.2pt]{$#1$}{\rule{.8ex}{.075ex}}}
\newcommand{\succprec}{\mathrel{\mathpalette\succ@prec{\succ\prec}}}
\newcommand{\precsucc}{\mathrel{\mathpalette\succ@prec{\prec\succ}}}


\newcounter{example}[section] % Reset example counter at each new section
\renewcommand{\theexample}{\thesection.\arabic{example}} % Format the example number as section.number

\newenvironment{example}
  {% Begin environment
   \refstepcounter{example}% Step counter and allow for labeling
   \noindent\textbf{Example \theexample.} % Display the example number
  }
  {% End environment
   \par\noindent\hfill\textit{End of Example.}\par
  }



% deeper section command
% This will let you go one level deeper than whatever section level you're on.
\makeatletter
\newcommand{\deepersection}[1]{%
  \ifnum\value{subparagraph}>0
    % Already at the deepest standard level (\subparagraph), cannot go deeper
    \subparagraph{#1}
  \else
    \ifnum\value{paragraph}>0
      \subparagraph{#1}
    \else
      \ifnum\value{subsubsection}>0
        \paragraph{#1}
      \else
        \ifnum\value{subsection}>0
          \subsubsection{#1}
        \else
          \ifnum\value{section}>0
            \subsection{#1}
          \else
            \section{#1}
          \fi
        \fi
      \fi
    \fi
  \fi
}
\makeatother


% same section command
% This will let create a section at the same level as whatever section level you're on.
\makeatletter
\newcommand{\samesection}[1]{%
  \ifnum\value{subparagraph}>0
    \subparagraph{#1}
  \else
    \ifnum\value{paragraph}>0
      \paragraph{#1}
    \else
      \ifnum\value{subsubsection}>0
        \subsubsection{#1}
      \else
        \ifnum\value{subsection}>0
          \subsection{#1}
        \else
          \ifnum\value{section}>0
            \section{#1}
          \else
            % Default to section if outside any sectioning
            \section{#1}
          \fi
        \fi
      \fi
    \fi
  \fi
}
\makeatother

\makeatletter
\newcommand{\shallowersection}[1]{%
  \ifnum\value{subparagraph}>0
    \paragraph{#1} % From subparagraph to paragraph
  \else
    \ifnum\value{paragraph}>0
      \subsubsection{#1} % From paragraph to subsubsection
    \else
      \ifnum\value{subsubsection}>0
        \subsection{#1} % From subsubsection to subsection
      \else
        \ifnum\value{subsection}>0
          \section{#1} % From subsection to section
        \else
          \ifnum\value{section}>0
            \chapter{#1} % Assuming a document class with chapters
          \else
            \section{#1} % Default to section if somehow higher than section
          \fi
        \fi
      \fi
    \fi
  \fi
}
\makeatother



\newcounter{problemcounter}
\renewcommand{\theproblemcounter}{Q.\arabic{problemcounter}}

% Define the problem environment
\newenvironment{problem}[1][]{%
  \refstepcounter{problemcounter}%
  \if\relax\detokenize{#1}\relax
    \tcolorbox[breakable, colback=red!10, colframe=red!50, fonttitle=\bfseries, title={Problem \theproblemcounter}, arc=5mm, boxrule=0.5mm]
  \else
    \tcolorbox[breakable, colback=red!10, colframe=red!50, fonttitle=\bfseries, title={Problem \theproblemcounter: #1}, arc=5mm, boxrule=0.5mm]
    \addcontentsline{toc}{subsubsection}{\theproblemcounter: #1}%
  \fi
}{
  \endtcolorbox
}

% Define a new counter for definitions
\newcounter{definitioncounter}
\renewcommand{\thedefinitioncounter}{D.\arabic{definitioncounter}}

\newenvironment{definition}[1][]{%
  \refstepcounter{definitioncounter}%
  \if\relax\detokenize{#1}\relax
    \tcolorbox[
      breakable,
      parbox=false, % Treat content normally regarding paragraphs
      before upper={\parindent0pt \parskip7pt}, % No indentation and add space between paragraphs
      colback=blue!10,
      colframe=blue!50,
      fonttitle=\bfseries,
      title={Definition \thedefinitioncounter},
      arc=5mm,
      boxrule=0.5mm,
      before skip=10pt, % Adjust vertical space before the box
      after skip=10pt % Adjust vertical space after the box
    ]
  \else
    \tcolorbox[
      breakable,
      parbox=false,
      before upper={\parindent0pt \parskip7pt},
      colback=blue!10,
      colframe=blue!50,
      fonttitle=\bfseries,
      title={Definition \thedefinitioncounter: #1},
      arc=5mm,
      boxrule=0.5mm,
      before skip=10pt,
      after skip=10pt
    ]
  \fi
}{
  \endtcolorbox
}


% Define a new counter for theorems
\newcounter{theoremcounter}
\renewcommand{\thetheoremcounter}{T.\arabic{theoremcounter}}

\newenvironment{theorem}[1][]{%
  \refstepcounter{theoremcounter}%
  \if\relax\detokenize{#1}\relax
    \tcolorbox[
      breakable,
      parbox=false, % Treat content normally regarding paragraphs
      before upper={\parindent0pt \parskip7pt}, % No indentation and add space between paragraphs
      colback=green!10,
      colframe=green!55,
      fonttitle=\bfseries,
      title={Theorem \thetheoremcounter},
      arc=5mm,
      boxrule=0.5mm,
      before skip=10pt, % Adjust vertical space before the box
      after skip=10pt % Adjust vertical space after the box
    ]
  \else
    \tcolorbox[
      breakable,
      parbox=false,
      before upper={\parindent0pt \parskip7pt},
      colback=green!10,
      colframe=green!55,
      fonttitle=\bfseries,
      title={Theorem \thetheoremcounter: #1},
      arc=5mm,
      boxrule=0.5mm,
      before skip=10pt,
      after skip=10pt
    ]
  \fi
}{
  \endtcolorbox
}


% Define a new counter for remarks
\newcounter{remarkcounter}
\renewcommand{\theremarkcounter}{R.\arabic{remarkcounter}}

\newenvironment{remark}[1][]{%
  \refstepcounter{remarkcounter}%
  \if\relax\detokenize{#1}\relax
    \tcolorbox[
      breakable,
      parbox=false, % Treat content normally regarding paragraphs
      before upper={\parindent0pt \parskip7pt}, % No indentation and add space between paragraphs
      colback=green!10,
      colframe=green!55,
      fonttitle=\bfseries,
      title={Remark \theremarkcounter},
      arc=5mm,
      boxrule=0.5mm,
      before skip=10pt, % Adjust vertical space before the box
      after skip=10pt % Adjust vertical space after the box
    ]
  \else
    \tcolorbox[
      breakable,
      parbox=false,
      before upper={\parindent0pt \parskip7pt},
      colback=green!10,
      colframe=green!55,
      fonttitle=\bfseries,
      title={Remark \theremarkcounter: #1},
      arc=5mm,
      boxrule=0.5mm,
      before skip=10pt,
      after skip=10pt
    ]
  \fi
}{
  \endtcolorbox
}

% Define a new counter for lemmas
\newcounter{lemmacounter}
\renewcommand{\thelemmacounter}{L.\arabic{lemmacounter}}

\newenvironment{lemma}[1][]{%
  \refstepcounter{lemmacounter}%
  \if\relax\detokenize{#1}\relax
    \tcolorbox[
      breakable,
      parbox=false, % Treat content normally regarding paragraphs
      before upper={\parindent0pt \parskip7pt}, % No indentation and add space between paragraphs
      colback=green!10,
      colframe=green!55,
      fonttitle=\bfseries,
      title={Lemma \thelemmacounter},
      arc=5mm,
      boxrule=0.5mm,
      before skip=10pt, % Adjust vertical space before the box
      after skip=10pt % Adjust vertical space after the box
    ]
  \else
    \tcolorbox[
      breakable,
      parbox=false,
      before upper={\parindent0pt \parskip7pt},
      colback=green!10,
      colframe=green!55,
      fonttitle=\bfseries,
      title={Lemma \thelemmacounter: #1},
      arc=5mm,
      boxrule=0.5mm,
      before skip=10pt,
      after skip=10pt
    ]
  \fi
}{
  \endtcolorbox
}

% Define a new counter for propositions
\newcounter{propositioncounter}
\renewcommand{\thepropositioncounter}{P.\arabic{propositioncounter}}

\newenvironment{proposition}[1][]{%
  \refstepcounter{propositioncounter}%
  \if\relax\detokenize{#1}\relax
    \tcolorbox[
      breakable,
      parbox=false, % Treat content normally regarding paragraphs
      before upper={\parindent0pt \parskip7pt}, % No indentation and add space between paragraphs
      colback=green!10,
      colframe=green!55,
      fonttitle=\bfseries,
      title={Proposition \thepropositioncounter},
      arc=5mm,
      boxrule=0.5mm,
      before skip=10pt, % Adjust vertical space before the box
      after skip=10pt % Adjust vertical space after the box
    ]
  \else
    \tcolorbox[
      breakable,
      parbox=false,
      before upper={\parindent0pt \parskip7pt},
      colback=green!10,
      colframe=green!55,
      fonttitle=\bfseries,
      title={Proposition \thepropositioncounter: #1},
      arc=5mm,
      boxrule=0.5mm,
      before skip=10pt,
      after skip=10pt
    ]
  \fi
}{
  \endtcolorbox
}

%\newtheorem{proposition}[theorem]{Proposition}  % Propositions share numbering with theorems


% Define a new counter for notes
\newcounter{notescounter}
\renewcommand{\thenotescounter}{D.\arabic{notescounter}}

\newenvironment{notes}[1][]{
  \refstepcounter{notescounter}%
  \if\relax\detokenize{#1}\relax
    % If #1 is empty, set the title to "Notes"
    \tcolorbox[
      breakable,
      parbox=false, % Treat content normally regarding paragraphs
      before upper={\parindent0pt \parskip7pt}, % No indentation and add space between paragraphs
      colback=blue!10,
      colframe=blue!50,
      fonttitle=\bfseries,
      title={Notes},
      arc=5mm,
      boxrule=0.5mm,
      before skip=10pt, % Adjust vertical space before the box
      after skip=10pt % Adjust vertical space after the box
    ]
  \else
    % If #1 is not empty, use it as the title
    \tcolorbox[
      breakable,
      parbox=false,
      before upper={\parindent0pt \parskip7pt},
      colback=blue!10,
      colframe=blue!50,
      fonttitle=\bfseries,
      title={#1}, % Use provided title instead of default
      arc=5mm,
      boxrule=0.5mm,
      before skip=10pt,
      after skip=10pt
    ]
  \fi
}{
  \endtcolorbox
}







% Define a new counter for questions
\newcounter{questionscounter}
\renewcommand{\thequestionscounter}{D.\arabic{questionscounter}}

\newenvironment{questions}[1][]{
  \refstepcounter{questionscounter}%
  \if\relax\detokenize{#1}\relax
    % If #1 is empty, set the title to "Questions"
    \tcolorbox[
      breakable,
      parbox=false, % Treat content normally regarding paragraphs
      before upper={\parindent0pt \parskip7pt}, % No indentation and add space between paragraphs
      colback=red!10,
      colframe=red!50,
      fonttitle=\bfseries,
      title={Questions},
      arc=5mm,
      boxrule=0.5mm,
      before skip=10pt, % Adjust vertical space before the box
      after skip=10pt % Adjust vertical space after the box
    ]
  \else
    % If #1 is not empty, use it as the title
    \tcolorbox[
      breakable,
      parbox=false,
      before upper={\parindent0pt \parskip7pt},
      colback=red!10,
      colframe=red!50,
      fonttitle=\bfseries,
      title={#1}, % Use provided title instead of default
      arc=5mm,
      boxrule=0.5mm,
      before skip=10pt,
      after skip=10pt
    ]
  \fi
}{
  \endtcolorbox
}






% Define a new counter for overview
\newcounter{overviewcounter}
\renewcommand{\theoverviewcounter}{D.\arabic{overviewcounter}}

\newenvironment{overview}[1][]{%
  \refstepcounter{overviewcounter}%
  \if\relax\detokenize{#1}\relax
    \tcolorbox[
      breakable,
      parbox=false, % Treat content normally regarding paragraphs
      before upper={\parindent0pt \parskip7pt}, % No indentation and add space between paragraphs
      colback=green!10,
      colframe=green!55,
      fonttitle=\bfseries,
      title={Overview},
      arc=5mm,
      boxrule=0.5mm,
      before skip=10pt, % Adjust vertical space before the box
      after skip=10pt % Adjust vertical space after the box
    ]
  \else
    \tcolorbox[
      breakable,
      parbox=false,
      before upper={\parindent0pt \parskip7pt},
      colback=green!10,
      colframe=green!55,
      fonttitle=\bfseries,
      title={Overview},
      arc=5mm,
      boxrule=0.5mm,
      before skip=10pt,
      after skip=10pt
    ]
  \fi
}{
  \endtcolorbox
}



% Add a line after paragraph header
\makeatletter
\renewcommand\paragraph{\@startsection{paragraph}{4}{\z@}%
            {-3.25ex \@plus -1ex \@minus -.2ex}%
            {1.5ex \@plus .2ex}%
            {\normalfont\normalsize\bfseries}}
\makeatother


% Taking away line before and after align
\BeforeBeginEnvironment{align}{\vspace{-\parskip}}
\AfterEndEnvironment{align}{\vskip0pt plus 2pt}

\usepackage{changepage}

\title{Home Production Notes}

\author{}
\date{October 2024}

\begin{document}
\maketitle

\tableofcontents

\section{Terms}

\input{../input/home_prod_terms.tex}

%%%%%%%%%%%%%%%%%%%%%%%%%%%%%%%%%%%%%%%%%%%%%%%%%%%%%%%%%%%%%%%%%%%%%%%%%%%%%%%%%%%%%%%
%%%%%%%%%%%%%%%%%%%%%%%%%%%%%%%%%%%%%%%%%%%%%%%%%%%%%%%%%%%%%%%%%%%%%%%%%%%%%%%%%%%%%%%
%%%%%%%%%%%%%%%%%%%%%%%%%%%%%%%%%%%%%%%%%%%%%%%%%%%%%%%%%%%%%%%%%%%%%%%%%%%%%%%%%%%%%%%

\section{The Model}

\subsection{Preferences and Endowments}

\subsubsection{Preferences}

The preferences of a worker who lives in location $n$ and works in location $i$ is given by the following utility function of the Cobb-Douglas form:

\begin{align}
    U_{n i \omega}=\frac{b_{n i \omega}}{\kappa_{n i}}\left(\frac{C_{n \omega}}{\alpha}\right)^\alpha\left(\frac{H_{n \omega}}{1-\alpha}\right)^{1-\alpha} s_\omega^\gamma
\end{align}

Idiosyncratic amenities 
are drawn from 
an independent Fréchet distribution:

\begin{align}
    G_{n i}(b)=e^{-B_{n i} b^{-\epsilon}}, \quad B_{n i}>0, \epsilon>1
\end{align}

%%%%%%%%%%%%%%%%%%%%%%%%%%%%%%%%%%%%%%%%%%%%%%%%%%%%%%%%%%%%%%%%%%%%%%%%%%%%%%%%%%%%%%%
\subsubsection{Time Allocation}

Under the given utility function, we find that 
the share of time allocated to home production is given by:

\begin{align}
    s_\omega = \frac{\gamma}{1+\gamma} \label{eq:time_allocation}
\end{align}

and hence the share of time allocated to work is given by:

\begin{align}
    1 - s_\omega = \frac{1}{1+\gamma} \label{eq:time_allocation2}
\end{align}

See \autoref{sec:time_allocation} for derivation.

%%%%%%%%%%%%%%%%%%%%%%%%%%%%%%%%%%%%%%%%%%%%%%%%%%%%%%%%%%%%%%%%%%%%%%%%%%%%%%%%%%%%%%%

\subsubsection{Good Consumption Index}

The good consumption index is given the form:

\begin{align}
    C_n=\left[\sum_{i \in N} \int_0^{M_i} c_{n i}(j)^\rho d j\right]^{\frac{1}{\rho}}, \quad \sigma=\frac{1}{1-\rho}>1
\end{align}

``The goods consumption index in location $n$ is a constant elasticity of 
substitution (CES) function of consumption of a continuum of tradable 
varieties sourced from each location $i$.''

Utility maximization 
will give that ``the equilibrium consumption 
in location $n$ of each variety sourced from 
location $i$ 
is given by'':

\begin{align}
    c_{n i}(j)=\alpha X_n P_n^{\sigma-1} p_{n i}(j)^{-\sigma} \label{eq:good_nij_consumption}
\end{align}

See \autoref{sec:good_nij_consumption} for derivation.

%%%%%%%%%%%%%%%%%%%%%%%%%%%%%%%%%%%%%%%%%%%%%%%%%%%%%%%%%%%%%%%%%%%%%%%%%%%%%%%%%%%%%%%
\subsubsection{Land and Local Consumption}

``We assume that this land is owned by immobile landlords,
who receive worker expenditure on residential land as income, and consume only
goods where they live.''

From there, we get the expression

\begin{align}
    P_n C_n=\alpha \bar{v}_n R_n+(1-\alpha) \bar{v}_n R_n=\bar{v}_n R_n \label{eq:land_and_local_consumption}
\end{align}

which says that the 
total expenditure on goods in location $n$, $P_n C_n$
is equal to the total labor income of 
residents in location $n$, $\bar{v}_n R_n$.

The middle term can be read as 
residents' total spending on goods in $n$,
$\alpha \bar{v}_n R_n$, plus 
residents' total spending on land in $n$,
$(1-\alpha) \bar{v}_n R_n$.

We can also get the following expression:

\begin{align}
    Q_n=(1-\alpha) \frac{\bar{v}_n R_n}{H_n} \label{eq:land_market_clearing2}
\end{align}

which says that 
the land price in location $n$, $Q_n$,
is equal to the total spending on land in $n$ 
divided by the supply of land in $n$.

This follows from the land market clearing condition:

\begin{align}
    \underbrace{Q_n \times H_n}_{\text {price } \times \text { quantity of land }}=\underbrace{(1-\alpha) \bar{v}_n R_n}_{\text {total rent paid by residents }} \label{eq:land_market_clearing}
\end{align}

and is useful, because it allows us to express 
rent as a function of the supply of land.

%%%%%%%%%%%%%%%%%%%%%%%%%%%%%%%%%%%%%%%%%%%%%%%%%%%%%%%%%%%%%%%%%%%%%%%%%%%%%%%%%%%%%%%
%%%%%%%%%%%%%%%%%%%%%%%%%%%%%%%%%%%%%%%%%%%%%%%%%%%%%%%%%%%%%%%%%%%%%%%%%%%%%%%%%%%%%%%
\subsection{Production}

\subsubsection{Labor Requirement for Production}

Firms produce tradable varieties under 
monopolistic competition
and increasing returns to scale
using labor as the lone input. 

To produce a variety, firms incur 
a fixed cost, $F$, as well as a 
variable cost
that is determined by the inverse 
of local productivity, $A_i$: $x_i(j)/A_i$.

Thus, the total amount of labor, $l_i(j)$, required to produce
$x_i(j)$ units of variety $j$ in location $i$ is
given by:

\begin{align}
    l_i(j)=F+ \frac{x_i(j)}{A_i} \label{eq:labor_requirement}
\end{align}

%%%%%%%%%%%%%%%%%%%%%%%%%%%%%%%%%%%%%%%%%%%%%%%%%%%%%%%%%%%%%%%%%%%%%%%%%%%%%%%%%%%%%%%
\subsubsection{Price: Constant Markup over Marginal Cost}

Profit maximization  
gives us that 
the equilibrium prices are a 
constant markup over marginal cost:

\begin{align}
    p_{n i}(j)=\left(\frac{\sigma}{\sigma-1}\right) \frac{d_{n i} w_i}{A_i} \label{eq:price_ni_1}
\end{align}

The derivation is given in \autoref{sec:price_ni_1}.

Notice also that $p_{n i}(j)$ is the 
same for all varieties produced in location $i$ and 
consumed in location $n$, i.e.,
there is no $j$ on the RHS. 


%%%%%%%%%%%%%%%%%%%%%%%%%%%%%%%%%%%%%%%%%%%%%%%%%%%%%%%%%%%%%%%%%%%%%%%%%%%%%%%%%%%%%%%
\subsubsection{Equilibrium Output of $j$ and $i$}

If we then additionally note zero profits, we get
an expression for equilibrium output of each variety
in each location:

\begin{align}
    x_i(j)=A_i F(\sigma-1) \label{eq:eq_output_xij}
\end{align}

The derivation is given in \autoref{sec:eq_output_xij}.

%%%%%%%%%%%%%%%%%%%%%%%%%%%%%%%%%%%%%%%%%%%%%%%%%%%%%%%%%%%%%%%%%%%%%%%%%%%%%%%%%%%%%%%
\subsubsection{Measure of Produced Varieties}

\eqref{eq:time_allocation2} and \eqref{eq:eq_output_xij},
combined with labor market 
clearing, gives us that 
the total measure of produced varieties in region $i$, $M_i$,
is proportional to the 
measure of employed workers, $L_i$:

\begin{align}
    M_i = \left(\frac{L_i}{\sigma F}\right) \left(\frac{1}{1+\gamma}\right) \label{eq:m_i_1}
\end{align}

The derivation is given in \autoref{sec:m_i_1}.

%%%%%%%%%%%%%%%%%%%%%%%%%%%%%%%%%%%%%%%%%%%%%%%%%%%%%%%%%%%%%%%%%%%%%%%%%%%%%%%%%%%%%%%
%%%%%%%%%%%%%%%%%%%%%%%%%%%%%%%%%%%%%%%%%%%%%%%%%%%%%%%%%%%%%%%%%%%%%%%%%%%%%%%%%%%%%%%
\subsection{Goods Trade}

\subsubsection{Share of $n$'s Expenditure on $i$'s Goods}

``The model implies a gravity equation 
for bilateral trade between locations.''

``Using the CES expenditure function, the 
equilibrium pricing rule, and the measure 
of firms, the 
share of location $n$'s expenditure on 
good's produced in location $i$ is:''

\begin{align}
    \pi_{n i}=\frac{M_i p_{n i}^{1-\sigma}}{\sum_{k \in N} M_k p_{n k}^{1-\sigma}}=\frac{L_i\left(d_{n i} w_i / A_i\right)^{1-\sigma}}{\sum_{k \in N} L_k\left(d_{n k} w_k / A_k\right)^{1-\sigma}} \label{eq:share_ni_1}
\end{align}

The derivation is given in \autoref{sec:share_ni_1}.

Notice that trade between locations $n$ and $i$
depends on both their own trade costs, $d_{n i}$,
as well as the trade costs between $n$ and all other locations.

%%%%%%%%%%%%%%%%%%%%%%%%%%%%%%%%%%%%%%%%%%%%%%%%%%%%%%%%%%%%%%%%%%%%%%%%%%%%%%%%%%%%%%%
\subsubsection{Workplace Income}

``Equating revenue and
expenditure, and using zero profits, workplace income in 
each location equals total expenditure 
on goods produced in that location, namely,''

\begin{align}
    w_i L_i=\sum_{n \in N} \pi_{n i} \bar{v}_n R_n \label{eq:workplace_income}
\end{align}

The derivation is given in \autoref{sec:workplace_income}.

%%%%%%%%%%%%%%%%%%%%%%%%%%%%%%%%%%%%%%%%%%%%%%%%%%%%%%%%%%%%%%%%%%%%%%%%%%%%%%%%%%%%%%%
\subsubsection{Price Index Re-Expression}

Then, using the equilibrium pricing rule
and labor market clearing, we can 
re-express the price index dual to the consumption index as:

\begin{align}
    P_n & =\left(\frac{\sigma}{\sigma-1}\right) \left[\left(\frac{1}{1 + \gamma}\right) \left(\frac{1}{\sigma F}\right)\right]^{\frac{1}{1-\sigma}} \left[\sum_{i \in N} L_i \left(\frac{d_{ni} w_i}{A_i}\right)^{1-\sigma}\right]^{\frac{1}{1-\sigma}} \\
    & =\left(\frac{\sigma}{\sigma-1}\right) \left[\left(\frac{1}{1 + \gamma}\right) \left(\frac{L_n}{\sigma F \pi_{nn}}\right)\right]^{\frac{1}{1-\sigma}} \frac{d_{nn} w_n}{A_n} \label{eq:price_index_p_n_2}
\end{align}

The derivation is given in \autoref{sec:price_ni_2}.

%%%%%%%%%%%%%%%%%%%%%%%%%%%%%%%%%%%%%%%%%%%%%%%%%%%%%%%%%%%%%%%%%%%%%%%%%%%%%%%%%%%%%%%
%%%%%%%%%%%%%%%%%%%%%%%%%%%%%%%%%%%%%%%%%%%%%%%%%%%%%%%%%%%%%%%%%%%%%%%%%%%%%%%%%%%%%%%
\subsection{Labor Mobility and Commuting}

\subsubsection{Indirect Utility}




%%%%%%%%%%%%%%%%%%%%%%%%%%%%%%%%%%%%%%%%%%%%%%%%%%%%%%%%%%%%%%%%%%%%%%%%%%%%%%%%%%%%%%%
%%%%%%%%%%%%%%%%%%%%%%%%%%%%%%%%%%%%%%%%%%%%%%%%%%%%%%%%%%%%%%%%%%%%%%%%%%%%%%%%%%%%%%%
%%%%%%%%%%%%%%%%%%%%%%%%%%%%%%%%%%%%%%%%%%%%%%%%%%%%%%%%%%%%%%%%%%%%%%%%%%%%%%%%%%%%%%%
\section{Derivations}

\subsection{Derivation of Time Allocation}
\label{sec:time_allocation}

This is the derivation of \eqref{eq:time_allocation}.

Note that the individual's problem is:

\begin{align}
    \text{max } &\frac{b_{n i \omega}}{\kappa_{n i}}\left(\frac{C_{n \omega}}{\alpha}\right)^\alpha\left(\frac{H_{n \omega}}{1-\alpha}\right)^{1-\alpha} s_\omega^\gamma \\
    \text{s.t. } &s_\omega \in [0,1] \\
    \text{and } & P_n C_{n \omega}+Q_n H_{n \omega}=w_i\left(1-s_\omega\right)
\end{align}

For the earnings that the individual receives, their expenditures 
on consumption and land will follow the standard Cobb-Douglas 
results. That is,

\begin{align}
    P_n C_{n \omega}&=\alpha\left[w_i\left(1-s_\omega\right)\right] \\
    Q_n H_{n \omega}&=(1-\alpha)\left[w_i\left(1-s_\omega\right)\right]
\end{align}

which gives 

\begin{align}
    C_{n \omega}&=\frac{\alpha w_i\left(1-s_\omega\right)}{p} \\
    H_{n \omega}&=\frac{(1-\alpha) w_i\left(1-s_\omega\right)}{Q_n}
\end{align}

Plugging these results back into the worker's problem gives us
the new objective to maximize:

\begin{align}
    \underset{s_\omega}{\text{max }} &\underbrace{\frac{b_{n i \omega}}{\kappa_{n i}} w_i P_n^{-\alpha} Q_n^{-(1-\alpha)}}_{\text {constant wrt } s_\omega} \times\left(1-s_\omega\right) s_\omega^\gamma \\
    \text{s.t. } &s_\omega \in [0,1]
\end{align}

which will yield the same $s_\omega$ as simply solving:

\begin{align}
    \underset{s_\omega}{\text{max }} &\left(1-s_\omega\right) s_\omega^\gamma \\
    \text{s.t. } &s_\omega \in [0,1]
\end{align}

If we ignore the constraint for the moment 
and then later return to verify that it's satisfied, 
we get the FOC and subsequent equalities:

\begin{align}
    &\frac{\partial}{\partial s_\omega} \left(1-s_\omega\right) s_\omega^\gamma = 0 \\
    \Rightarrow &\gamma s_\omega^{\gamma-1}-(\gamma+1)s_\omega^\gamma = 0 \\
    \Rightarrow &\gamma - (\gamma+1)s_\omega = 0 \\
    \Rightarrow &s_\omega = \frac{\gamma}{\gamma+1}
\end{align}

which also gives that time spent working is

\begin{align}
    1 - s_\omega = \frac{1}{\gamma+1}
\end{align}

$s_\omega \in [0,1]$ as long as $\gamma \geq 0$.


%%%%%%%%%%%%%%%%%%%%%%%%%%%%%%%%%%%%%%%%%%%%%%%%%%%%%%%%%%%%%%%%%%%%%%%%%%%%%%%%%%%%%%%
%%%%%%%%%%%%%%%%%%%%%%%%%%%%%%%%%%%%%%%%%%%%%%%%%%%%%%%%%%%%%%%%%%%%%%%%%%%%%%%%%%%%%%%

\subsection{Derivation of $c_{n i}(j)$ Expression} 
\label{sec:good_nij_consumption}

\subsubsection{Write the Problem}
This is the derivation of \eqref{eq:good_nij_consumption}.

When making consumption decisions surrounding
$c_{n i}(j)$, an individual is solving the following problem:

\begin{align}
    &\underset{\{c_{n i}(j)\}}{\max} C_n \\ 
    \text{s.t. } &\sum_i \int_{0}^{M_i} p_{n i}(j) c_{n i}(j) d j=\alpha X_n
\end{align}

or expanded out:

\begin{align}
    &\underset{\{c_{n i}(j)\}}{\max} \left[\sum_{i \in N} \int_0^{M_i} c_{n i}(j)^\rho d j\right]^{\frac{1}{\rho}} \\
    \text{s.t. } &\sum_i \int_{0}^{M_i} p_{n i}(j) c_{n i}(j) d j=\alpha X_n
\end{align}

where the $\alpha X_n$ term comes from the fact that 
people spend $\alpha$ of their total expenditures, $X_n$,
on goods.

\subsubsection{Lagrangian and FOCs}

The Langragian for this problem is then:

\begin{align}
    \mathcal{L} = \left[\sum_{i \in N} \int_0^{M_i} c_{n i}(j)^\rho d j\right]^{\frac{1}{\rho}} + \lambda \left(\alpha X_n - \sum_i \int_{0}^{M_i} p_{n i}(j) c_{n i}(j) d j\right)
\end{align}

which gives the relevant FOC:

\begin{align}
    \{c_{n i}(j)\} \quad \quad &\frac{\partial}{\partial c_{n i}(j)} \left[\left(\sum_{i \in N} \int_0^{M_i} c_{n i}(j)^\rho d j\right)^{\frac{1}{\rho}}\right] - \lambda p_{n i}(j) = 0 \\
    \Leftrightarrow & \frac{1}{\rho} \left[\sum_{i \in N} \int_0^{M_i} \rho c_{n i}(j)^\rho d j\right]^{\frac{1}{\rho}-1} c_{n i}(j)^{\rho-1} - \lambda p_{n i}(j) = 0 \\
    \Leftrightarrow & C_n^{1-\rho} c_{n i}(j)^{\rho-1} = \lambda p_{n i}(j) \\
    \Leftrightarrow & c_{n i}(j) = \lambda^{\frac{1}{\rho -1}} p_{ni}(j)^{\frac{1}{\rho -1}} C_n \\ 
    \Leftrightarrow & c_{n i}(j) = \lambda^{-\sigma} p_{ni}(j)^{-\sigma} C_n && \text{since $\sigma = \frac{1}{1-\rho}$} \label{eq:inter_good_nij_cons}
\end{align}


\subsubsection{Solve for $\lambda$}

From there, we can define the dual price index 

\begin{align}
    P_n \equiv \left(\sum_{i \in N} \int_0^{M_i} p_{n i}(j)^{1-\sigma} d_j \right)^{\frac{1}{1-\sigma}} \label{eq:price_index_p_n}
\end{align}

and revisit to our budget constraint:

\begin{alignat}{2}
    &\alpha X_n && = \sum_i \int_{0}^{M_i} p_{n i}(j) c_{n i}(j) d j \\
    &&&= \sum_i \int_{0}^{M_i} p_{n i}(j) \lambda^{-\sigma} p_{ni}(j)^{-\sigma} C_n d j \quad \text{by \eqref{eq:inter_good_nij_cons}} \\
    &&&= \lambda^{-\sigma} C_n \sum_i \int_{0}^{M_i} p_{n i}(j)^{1-\sigma} d j \\
    &&& = \lambda^{-\sigma} C_n P_n^{1-\sigma} \\
    \Rightarrow &\lambda^{-\sigma} &&= \frac{\alpha X_n}{C_n} P_n^{\sigma-1} \\
\end{alignat}

\subsubsection{Plug in $\lambda$}

Then, returning to \eqref{eq:inter_good_nij_cons}, we can plug in our expression for $\lambda^{-\sigma}$:

\begin{align}
    c_{n i}(j) &= \lambda^{-\sigma} p_{ni}(j)^{-\sigma} C_n \\
    &= \left(\frac{\alpha X_n}{C_n} P_n^{\sigma-1}\right) p_{ni}(j)^{-\sigma} C_n \\
    &= \alpha X_n P_n^{\sigma-1} p_{ni}(j)^{-\sigma}
\end{align}

which is what we wanted.

%%%%%%%%%%%%%%%%%%%%%%%%%%%%%%%%%%%%%%%%%%%%%%%%%%%%%%%%%%%%%%%%%%%%%%%%%%%%%%%%%%%%%%%
%%%%%%%%%%%%%%%%%%%%%%%%%%%%%%%%%%%%%%%%%%%%%%%%%%%%%%%%%%%%%%%%%%%%%%%%%%%%%%%%%%%%%%%
\subsection{Derivation of $p_{n i}(j)$ Expression}
\label{sec:price_ni_1}
This is the derivation of \eqref{eq:price_ni_1}.

\subsubsection{Marginal Cost}

We will get the expression 
for $p_{n i}(j)$ by equating 
marginal cost and marginal revenue.

First, notice that since 

\begin{align}
    l_i(j)=F+\frac{x_i(j)}{A_i}
\end{align}

cost is given by:

\begin{align}
    \underbrace{\operatorname{Cost}\left(x_i(j)\right)}_{\text {total monetary cost }}=\underbrace{w_i}_{\text {wage }}\left[F+\frac{x_i(j)}{A_i}\right] \label{eq:cost_xij}
\end{align}

However, to account for trade costs,
we must consider the variable cost of 
delivering $q_{ni}(j)$ units of variety $j$
to location $n$ from location $i$:

\begin{align}
    \underbrace{\text{Var Cost}\left(q_{ni}(j)\right)}_{\text {variable monetary cost }}=\underbrace{w_i}_{\text {wage }} \underbrace{d_{ni}}_{\text{trade costs}} \left[\frac{q_{ni}(j)}{A_i}\right] \label{eq:var_cost_qnij}
\end{align}

This is fine for marginal cost, since the fixed cost 
will disappear when we take the derivative.

Also, notice that we've defined $q_{ni}(j)$ such that:

\begin{align}
    x_i(j) = \sum_{n \in N} d_{ni} q_{ni}(j) \label{eq:xij_qnij}
\end{align}

where $d_{ni}$ is the multiplier reflecting 
how many more units of variety $j$ must be made 
to deliver $q_{ni}(j)$ units of variety $j$ to location $n$
from location $i$.

Then, marginal cost is given by:

\begin{align}
    \text{MC} = \frac{d}{d q_{ni}(j)} \text{Var Cost}\left(q_{ni}(j)\right)= \frac{w_i d_{ni}}{A_i}
\end{align}

%%%%%%%%%%%%%%%%%%%%%%%%%%%%%%%%%%%%%%%%%%%%%%%%%%%%%%%%%%%%%%%%%%%%%%%%%%%%%%%%%%%%%%%
\subsubsection{Marginal Revenue}

Under isoelastic (CES) demand, 
we can write the quantity demanded as

\begin{align}
    q_{ni}(j)=A_i p_{n i}(j)^{-\sigma} 
\end{align}

Then, revenue is given by:

\begin{align}
    \text{Rev}(p_{n i}(j)) = p_{n i}(j) q_{ni}(j) = A_i p_{n i}(j)^{1-\sigma}
\end{align}

Then marginal revenue is given by:

\begin{align}
    \text{MR}_{n i}(j) = &\frac{d \text{Rev}(p_{n i}(j))}{dp_{n i}(j)} \frac{d p_{n i}(j)}{d q_{ni}(j)} \label{eq:mr_ni_1}
\end{align}

by the chain rule.

Then notice that

\begin{align}
    \frac{d \text{Rev}(p_{n i}(j))}{dp_{n i}(j)} = A_i (1-\sigma) p_{n i}(j)^{-\sigma} \label{eq:drev_dp}
\end{align}

and 

\begin{align}
    \frac{d q_{ni}(j)}{d p_{n i}(j)} = -\sigma A_i p_{n i}(j)^{-\sigma-1}
\end{align}

Since $q$ and $p$ are inversely related,
we can write:

\begin{align}
    \frac{d p_{n i}(j)}{d q_{ni}(j)} = \left(\frac{d q_{ni}(j)}{d p_{n i}(j)}\right)^{-1} = -\frac{1}{\sigma A_i p_{n i}(j)^{\sigma+1}} \label{eq:dp_dq}
\end{align}

Then, plugging 
\eqref{eq:drev_dp} and \eqref{eq:dp_dq} into \eqref{eq:mr_ni_1}, we get:

\begin{align}
    \text{MR}_{n i}(j) = &\frac{d \text{Rev}(p_{n i}(j))}{dp_{n i}(j)} \frac{d p_{n i}(j)}{d q_{ni}(j)}  \\
    = & A_i (1-\sigma) p_{n i}(j)^{-\sigma} \left(-\frac{1}{\sigma A_i p_{n i}(j)^{\sigma+1}}\right) \\
    = & \left(\frac{\sigma - 1}{\sigma}\right) p_{n i}(j)
\end{align}

%%%%%%%%%%%%%%%%%%%%%%%%%%%%%%%%%%%%%%%%%%%%%%%%%%%%%%%%%%%%%%%%%%%%%%%%%%%%%%%%%%%%%%%

\subsubsection{Equating MR and MC}

Then, equating marginal revenue and marginal cost, we get
the desired expression:

\begin{align}
    \text{MR}_{n i}(j) = \text{MC} \\
    \Rightarrow \left(\frac{\sigma - 1}{\sigma}\right) p_{n i}(j) = \frac{d_{ni} w_i}{A_i} \\
    \Rightarrow p_{n i}(j) = \left(\frac{\sigma}{\sigma-1}\right) \frac{d_{ni} w_i}{A_i}
\end{align}

%%%%%%%%%%%%%%%%%%%%%%%%%%%%%%%%%%%%%%%%%%%%%%%%%%%%%%%%%%%%%%%%%%%%%%%%%%%%%%%%%%%%%%%
%%%%%%%%%%%%%%%%%%%%%%%%%%%%%%%%%%%%%%%%%%%%%%%%%%%%%%%%%%%%%%%%%%%%%%%%%%%%%%%%%%%%%%%
\subsection{Derivation of $x_i(j)$ Expression}
\label{sec:eq_output_xij}

This is the derivation of \eqref{eq:eq_output_xij}.

Zero profit implies that 
revenue minus variable cost equals fixed cost,
which leads to the desired equation:

\begin{align}
    &\text{Revenue} - \text{Variable Cost} = \text{Fixed Cost} \\
    \Rightarrow & \sum_n \left[p_{ni}(j) q_{ni}(j) - w_i d_{ni} \left(\frac{q_{ni}(j)}{A_i}\right)\right] = w_i F && \parbox[t]{4cm}{\raggedright By \eqref{eq:cost_xij} and \eqref{eq:var_cost_qnij}} \\
    \Rightarrow & \sum_n \left[\frac{\sigma}{\sigma-1} \frac{d_{ni} w_i}{A_i} q_{ni}(j) - w_i d_{ni} \left(\frac{q_{ni}(j)}{A_i}\right)\right] = w_i F && \parbox[t]{4cm}{\raggedright By \eqref{eq:price_ni_1}} \\
    \Rightarrow & \sum_n \left[\frac{1}{\sigma-1} \frac{d_{ni}}{A_i} q_{ni}(j)\right] = F \\ 
    \Rightarrow & \sum_n d_{ni} q_{ni}(j) = A_i F (\sigma-1) \\
    \Rightarrow & x_i(j) = A_i F (\sigma-1) && \parbox[t]{4cm}{\raggedright By \eqref{eq:xij_qnij}}
\end{align}

%%%%%%%%%%%%%%%%%%%%%%%%%%%%%%%%%%%%%%%%%%%%%%%%%%%%%%%%%%%%%%%%%%%%%%%%%%%%%%%%%%%%%%%
%%%%%%%%%%%%%%%%%%%%%%%%%%%%%%%%%%%%%%%%%%%%%%%%%%%%%%%%%%%%%%%%%%%%%%%%%%%%%%%%%%%%%%%
\subsection{Derivation of $M_i$ Expression}
\label{sec:m_i_1}

This is a derivation of \eqref{eq:m_i_1}.

\begin{align}
    (1-s_\omega) L_i = &\int_{0}^{M_i} l_i(j) dj && \parbox[t]{4cm}{\raggedright Labor market clearing}\\
    = & \int_{0}^{M_i} \left[F + \frac{x_i(j)}{A_i}\right] dj && \text{By \eqref{eq:labor_requirement}} \\
    = &\int_{0}^{M_i} F + \frac{A_i F (\sigma -1)}{A_i} dj && \text{By \eqref{eq:eq_output_xij}} \\
    = &\int_{0}^{M_i} \sigma F dj \\
    = & \sigma F M_i
\end{align}

By re-arranging, we get the expression:

\begin{align}
    M_i = &\frac{(1-s_\omega)L_i}{\sigma F} \\
    = &\left(\frac{L_i}{\sigma F}\right) \left(\frac{1}{1+\gamma}\right) && \text{by \eqref{eq:time_allocation2}}
\end{align}

which is what we wanted. 


%%%%%%%%%%%%%%%%%%%%%%%%%%%%%%%%%%%%%%%%%%%%%%%%%%%%%%%%%%%%%%%%%%%%%%%%%%%%%%%%%%%%%%%
%%%%%%%%%%%%%%%%%%%%%%%%%%%%%%%%%%%%%%%%%%%%%%%%%%%%%%%%%%%%%%%%%%%%%%%%%%%%%%%%%%%%%%%

\subsection{Derivation of $\pi_{n i}$ Expression}
\label{sec:share_ni_1}

This is a derivation of \eqref{eq:share_ni_1}.

First, because it will be useful momentarily, note:

\begin{align}
    &P_n=\left[\sum_{k \in N} \int_0^{M_k} p_{n k}(j)^{1-\sigma} d j\right]^{\frac{1}{1-\sigma}} \\
    \Rightarrow &P_n^{1-\sigma} = \sum_{k \in N} \int_0^{M_k} p_{n k}(j)^{1-\sigma} d j \\
    \Rightarrow &P_n^{1-\sigma} = \sum_{k \in N} \int_0^{M_k} p_{nk}^{1-\sigma} d j && \parbox[t]{4cm}{\raggedright By \eqref{eq:price_ni_1}} \\
    \Rightarrow &P_n^{1-\sigma} = \sum_{k \in N} M_k p_{nk}^{1-\sigma} \label{eq:price_ind_interm_share_ni_1}
\end{align}

Now, begin with the baseline expression
for share of expenditure:

\begin{align}
    \pi_{n i}=&\frac{\int_0^{M_i} p_{n i}(j) c_{n i}(j) d j}{\sum_{k \in N} \int_0^{M_k} p_{n k}(j) c_{n k}(j) d j} \\
    = &\frac{\int_0^{M_i} p_{n i}(j) \alpha X_n P_n^{\sigma-1} p_{n i}(j)^{-\sigma} d j}{\sum_{k \in N} \int_0^{M_k} p_{n k}(j) \alpha X_n P_n^{\sigma-1} p_{n k}(j)^{-\sigma} d j} && \parbox[t]{4cm}{\raggedright By \eqref{eq:good_nij_consumption}} \\
    = &\frac{\int_0^{M_i} p_{ni}(j)^{1-\sigma} d j}{\sum_{k \in N} \int_0^{M_k} p_{nk}(j)^{1-\sigma} d j} && \parbox[t]{4cm}{\raggedright Cancel terms} \\
    = &\frac{\int_0^{M_i} p_{ni}(j)^{1-\sigma} d j}{P_n^{1-\sigma}} && \parbox[t]{4cm}{\raggedright By \eqref{eq:price_index_p_n}} \\
    = &\frac{\int_0^{M_i} p_{ni}^{1-\sigma} d j}{\sum_{k \in N} M_k p_{nk}^{1-\sigma}} && \parbox[t]{4cm}{\raggedright By \eqref{eq:price_ni_1} \& \eqref{eq:price_ind_interm_share_ni_1}} \\
    = &\frac{M_i p_{ni}^{1-\sigma}}{\sum_{k \in N} M_k p_{nk}^{1-\sigma}}
\end{align}

Thus, we now have the first desired equality:

\begin{align}
    \pi_{n i}=\frac{M_i p_{n i}^{1-\sigma}}{\sum_{k \in N} M_k p_{n k}^{1-\sigma}} \label{eq:share_ni_inter1}
\end{align}

Now, notice that:

\begin{align}
    M_i p_{ni}^{1-\sigma} = &\left(\frac{1}{1 + \gamma}\right)\left(\frac{L_i}{\sigma F}\right) \left(\frac{\sigma}{\sigma-1}\right)^{1-\sigma} \left(\frac{d_{ni} w_i}{A_i}\right)^{1-\sigma} && \parbox[t]{4cm}{\raggedright By \eqref{eq:price_ni_1} and \eqref{eq:m_i_1}}
\end{align}

Thus, we can re-write \eqref{eq:share_ni_inter1} to be: 

\begin{align}
    \pi_{n i}=&\frac{M_i p_{n i}^{1-\sigma}}{\sum_{k \in N} M_k p_{n k}^{1-\sigma}} \\
    = &\frac{\left(\frac{1}{1 + \gamma}\right)\left(\frac{L_i}{\sigma F}\right) \left(\frac{\sigma}{\sigma-1}\right)^{1-\sigma} \frac{d_{ni} w_i}{A_i}^{1-\sigma}}{\sum_{k \in N} \left(\frac{1}{1 + \gamma}\right)\left(\frac{L_k}{\sigma F}\right) \left(\frac{\sigma}{\sigma-1}\right)^{1-\sigma} \frac{d_{nk} w_k}{A_k}^{1-\sigma}} \\
    = &\frac{L_i \left(\frac{d_{ni} w_i}{A_i}\right)^{1-\sigma}}{\sum_{k \in N} L_k \left(\frac{d_{nk} w_k}{A_k}\right)^{1-\sigma}}
\end{align}

which is what we wanted.

%%%%%%%%%%%%%%%%%%%%%%%%%%%%%%%%%%%%%%%%%%%%%%%%%%%%%%%%%%%%%%%%%%%%%%%%%%%%%%%%%%%%%%%
%%%%%%%%%%%%%%%%%%%%%%%%%%%%%%%%%%%%%%%%%%%%%%%%%%%%%%%%%%%%%%%%%%%%%%%%%%%%%%%%%%%%%%%
\subsection{Derivation of $w_i L_i$ Expression}
\label{sec:workplace_income}

This is a derivation of \eqref{eq:workplace_income}.

There's not really much of a derivation here. 

First, note that total labor income is expressed as:

\begin{align}
    &w_i L_i (1- s_\omega) \\
    = &w_i L_i \left(\frac{1}{1+\gamma}\right) && \text{By \eqref{eq:time_allocation2}}
\end{align}

Total revenue from goods produced in location $i$ is given by:

\begin{align}
    \sum_{n \in N} \pi_{n i} \left(\frac{1}{1+\gamma}\right) \bar{v}_n R_n
\end{align}

Note that I'm using $\bar{v}_n$ to denote the average 
wage of workers in location $n$, rather than 
average income, with the distinction being that 
income would be wage scaled by share of time spent working.

There are no net profits, so 
total revenue minus total cost must be zero:

\begin{align}
    &\sum_{n \in N} \pi_{n i} \left(\frac{1}{1+\gamma}\right) \bar{v}_n R_n - w_i L_i \left(\frac{1}{1+\gamma}\right) = 0 \\
    \Rightarrow &\sum_{n \in N} \pi_{n i} \bar{v}_n R_n - w_i L_i = 0 \\
    \Rightarrow &\sum_{n \in N} \pi_{n i} \bar{v}_n R_n = w_i L_i
\end{align}

%%%%%%%%%%%%%%%%%%%%%%%%%%%%%%%%%%%%%%%%%%%%%%%%%%%%%%%%%%%%%%%%%%%%%%%%%%%%%%%%%%%%%%%
%%%%%%%%%%%%%%%%%%%%%%%%%%%%%%%%%%%%%%%%%%%%%%%%%%%%%%%%%%%%%%%%%%%%%%%%%%%%%%%%%%%%%%%

\subsection{Derivation of $P_n$ Re-Expression}
\label{sec:price_ni_2}

This is a derivation of \eqref{eq:price_index_p_n_2}.

\begin{align}
    P_n = &\left[\sum_{i \in N} \int_0^{M_i} p_{n i}(j)^{1-\sigma} d j\right]^{\frac{1}{1-\sigma}} && \parbox[t]{4cm}{\raggedright By \eqref{eq:price_index_p_n}} \\
    = &\left[\sum_{i \in N} \int_0^{M_i} p_{n i}^{1-\sigma} d j\right]^{\frac{1}{1-\sigma}} && \parbox[t]{4cm}{\raggedright By \eqref{eq:price_ni_1}} \\
    = &\left[\sum_{i \in N} M_i p_{n i}^{1-\sigma}\right]^{\frac{1}{1-\sigma}} \\
    = &\left[\sum_{i \in N} \left(\frac{1}{1 + \gamma}\right)\frac{L_i}{\sigma F} \left(\frac{\sigma}{\sigma-1}\right)^{1-\sigma} \frac{d_{ni} w_i}{A_i}^{1-\sigma}\right]^{\frac{1}{1-\sigma}} && \parbox[t]{4cm}{\raggedright By \eqref{eq:m_i_1} and \eqref{eq:price_ni_1}} \\
    = &\left(\frac{\sigma}{\sigma-1}\right) \left(\frac{1}{1 + \gamma}\right)^{\frac{1}{1 - \sigma}} \left(\frac{1}{\sigma F}\right)^{\frac{1}{1-\sigma}} \left[\sum_{i \in N} L_i \left(\frac{d_{ni} w_i}{A_i}\right)^{1-\sigma}\right]^{\frac{1}{1-\sigma}} \\
\end{align}

which is the first desired equality. 

From there, we can note

\begin{align}
    \pi_{nn} &= \frac{L_n \left(\frac{d_{nn} w_n}{A_n}\right)^{1-\sigma}}{\sum_{i \in N} L_i \left(\frac{d_{ni} w_i}{A_i}\right)^{1-\sigma}} && \parbox[t]{4cm}{\raggedright By \eqref{eq:share_ni_1}} \label{eq:pi_nn}
\end{align}

which let's us pick back up to write:

\begin{align}
    P_n = &\left(\frac{\sigma}{\sigma-1}\right) \left(\frac{1}{1 + \gamma}\right)^{\frac{1}{1 - \sigma}} \left(\frac{1}{\sigma F}\right)^{\frac{1}{1-\sigma}} \left[\sum_{i \in N} L_i \left(\frac{d_{ni} w_i}{A_i}\right)^{1-\sigma}\right]^{\frac{1}{1-\sigma}} \\
    = &\left(\frac{\sigma}{\sigma-1}\right) \left(\frac{1}{1 + \gamma}\right)^{\frac{1}{1 - \sigma}} \left(\frac{1}{\sigma F}\right)^{\frac{1}{1-\sigma}} \left[\frac{L_n \left(\frac{d_{nn} w_n}{A_n}\right)^{1-\sigma}}{\pi_{nn}}\right]^{\frac{1}{1-\sigma}} && \text{By \eqref{eq:pi_nn}} \\
    = &\left(\frac{\sigma}{\sigma-1}\right) \left[\left(\frac{1}{1 + \gamma}\right) \left(\frac{L_n}{\sigma F \pi_{nn}}\right)\right]^{\frac{1}{1-\sigma}} \frac{d_{nn} w_n}{A_n}
\end{align}

which is the second desired expression.



\end{document}
