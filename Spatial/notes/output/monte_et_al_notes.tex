\documentclass[10pt]{article}
\usepackage{amsmath}
\usepackage{amsthm}
\usepackage{amsfonts}
\usepackage{amssymb}
\usepackage{amssymb}
\usepackage{booktabs}
\setlength\parindent{0pt}
\usepackage[margin=1.2in]{geometry}
\usepackage{enumitem}
\usepackage{mathtools}
\mathtoolsset{showonlyrefs=true}
\usepackage{pdflscape}
\usepackage{xcolor}
\usepackage{hyperref}
\setcounter{tocdepth}{4}
\setcounter{secnumdepth}{4}
\usepackage[listings,skins,breakable]{tcolorbox} % package for colored boxes
\usepackage{etoolbox}
\usepackage{placeins}
\usepackage{tikz}
\usepackage{color}  % Allows for color customization
\usepackage{subcaption}
\usepackage[utf8]{inputenc}


% Make it so that the bottom page of a 
% book section doesn't have weird spacing
\raggedbottom


% Define custom colors
% You can-redo these later, they're not being used 
% for anything as of 5/12/24
\definecolor{codegreen}{rgb}{0,0.6,0}
\definecolor{codegray}{rgb}{0.5,0.5,0.5}
\definecolor{codepurple}{rgb}{0.58,0,0.82}
\definecolor{backcolour}{rgb}{0.95,0.95,0.92}

% You can-redo the lstlisting style later, it's not being used 
% for anything as of 5/12/24

% Define the lstlisting style
\lstdefinestyle{mystyle}{
    backgroundcolor=\color{backcolour},   
    commentstyle=\color{codegreen},
    keywordstyle=\color{magenta},
    numberstyle=\tiny\color{codegray},
    stringstyle=\color{codepurple},
    basicstyle=\ttfamily\footnotesize,
    breakatwhitespace=false,         
    breaklines=true,                 
    captionpos=b,                    
    keepspaces=true,                 
    numbers=left,                    
    numbersep=5pt,                  
    showspaces=false,                
    showstringspaces=false,
    showtabs=false,                  
    tabsize=2
}
\lstset{style=mystyle}


% Set the length of \parskip to add a line between paragraphs
\setlength{\parskip}{1em}


% Set the second level of itemize to use \circ as the bullet point
\setlist[itemize,2]{label={$\circ$}}

% Define symbols
\DeclareMathSymbol{\Perp}{\mathrel}{symbols}{"3F}
\newcommand\barbelow[1]{\stackunder[1.2pt]{$#1$}{\rule{.8ex}{.075ex}}}
\newcommand{\succprec}{\mathrel{\mathpalette\succ@prec{\succ\prec}}}
\newcommand{\precsucc}{\mathrel{\mathpalette\succ@prec{\prec\succ}}}


\newcounter{example}[section] % Reset example counter at each new section
\renewcommand{\theexample}{\thesection.\arabic{example}} % Format the example number as section.number

\newenvironment{example}
  {% Begin environment
   \refstepcounter{example}% Step counter and allow for labeling
   \noindent\textbf{Example \theexample.} % Display the example number
  }
  {% End environment
   \par\noindent\hfill\textit{End of Example.}\par
  }



% deeper section command
% This will let you go one level deeper than whatever section level you're on.
\makeatletter
\newcommand{\deepersection}[1]{%
  \ifnum\value{subparagraph}>0
    % Already at the deepest standard level (\subparagraph), cannot go deeper
    \subparagraph{#1}
  \else
    \ifnum\value{paragraph}>0
      \subparagraph{#1}
    \else
      \ifnum\value{subsubsection}>0
        \paragraph{#1}
      \else
        \ifnum\value{subsection}>0
          \subsubsection{#1}
        \else
          \ifnum\value{section}>0
            \subsection{#1}
          \else
            \section{#1}
          \fi
        \fi
      \fi
    \fi
  \fi
}
\makeatother


% same section command
% This will let create a section at the same level as whatever section level you're on.
\makeatletter
\newcommand{\samesection}[1]{%
  \ifnum\value{subparagraph}>0
    \subparagraph{#1}
  \else
    \ifnum\value{paragraph}>0
      \paragraph{#1}
    \else
      \ifnum\value{subsubsection}>0
        \subsubsection{#1}
      \else
        \ifnum\value{subsection}>0
          \subsection{#1}
        \else
          \ifnum\value{section}>0
            \section{#1}
          \else
            % Default to section if outside any sectioning
            \section{#1}
          \fi
        \fi
      \fi
    \fi
  \fi
}
\makeatother

\makeatletter
\newcommand{\shallowersection}[1]{%
  \ifnum\value{subparagraph}>0
    \paragraph{#1} % From subparagraph to paragraph
  \else
    \ifnum\value{paragraph}>0
      \subsubsection{#1} % From paragraph to subsubsection
    \else
      \ifnum\value{subsubsection}>0
        \subsection{#1} % From subsubsection to subsection
      \else
        \ifnum\value{subsection}>0
          \section{#1} % From subsection to section
        \else
          \ifnum\value{section}>0
            \chapter{#1} % Assuming a document class with chapters
          \else
            \section{#1} % Default to section if somehow higher than section
          \fi
        \fi
      \fi
    \fi
  \fi
}
\makeatother



\newcounter{problemcounter}
\renewcommand{\theproblemcounter}{Q.\arabic{problemcounter}}

% Define the problem environment
\newenvironment{problem}[1][]{%
  \refstepcounter{problemcounter}%
  \if\relax\detokenize{#1}\relax
    \tcolorbox[breakable, colback=red!10, colframe=red!50, fonttitle=\bfseries, title={Problem \theproblemcounter}, arc=5mm, boxrule=0.5mm]
  \else
    \tcolorbox[breakable, colback=red!10, colframe=red!50, fonttitle=\bfseries, title={Problem \theproblemcounter: #1}, arc=5mm, boxrule=0.5mm]
    \addcontentsline{toc}{subsubsection}{\theproblemcounter: #1}%
  \fi
}{
  \endtcolorbox
}

% Define a new counter for definitions
\newcounter{definitioncounter}
\renewcommand{\thedefinitioncounter}{D.\arabic{definitioncounter}}

\newenvironment{definition}[1][]{%
  \refstepcounter{definitioncounter}%
  \if\relax\detokenize{#1}\relax
    \tcolorbox[
      breakable,
      parbox=false, % Treat content normally regarding paragraphs
      before upper={\parindent0pt \parskip7pt}, % No indentation and add space between paragraphs
      colback=blue!10,
      colframe=blue!50,
      fonttitle=\bfseries,
      title={Definition \thedefinitioncounter},
      arc=5mm,
      boxrule=0.5mm,
      before skip=10pt, % Adjust vertical space before the box
      after skip=10pt % Adjust vertical space after the box
    ]
  \else
    \tcolorbox[
      breakable,
      parbox=false,
      before upper={\parindent0pt \parskip7pt},
      colback=blue!10,
      colframe=blue!50,
      fonttitle=\bfseries,
      title={Definition \thedefinitioncounter: #1},
      arc=5mm,
      boxrule=0.5mm,
      before skip=10pt,
      after skip=10pt
    ]
  \fi
}{
  \endtcolorbox
}


% Define a new counter for theorems
\newcounter{theoremcounter}
\renewcommand{\thetheoremcounter}{T.\arabic{theoremcounter}}

\newenvironment{theorem}[1][]{%
  \refstepcounter{theoremcounter}%
  \if\relax\detokenize{#1}\relax
    \tcolorbox[
      breakable,
      parbox=false, % Treat content normally regarding paragraphs
      before upper={\parindent0pt \parskip7pt}, % No indentation and add space between paragraphs
      colback=green!10,
      colframe=green!55,
      fonttitle=\bfseries,
      title={Theorem \thetheoremcounter},
      arc=5mm,
      boxrule=0.5mm,
      before skip=10pt, % Adjust vertical space before the box
      after skip=10pt % Adjust vertical space after the box
    ]
  \else
    \tcolorbox[
      breakable,
      parbox=false,
      before upper={\parindent0pt \parskip7pt},
      colback=green!10,
      colframe=green!55,
      fonttitle=\bfseries,
      title={Theorem \thetheoremcounter: #1},
      arc=5mm,
      boxrule=0.5mm,
      before skip=10pt,
      after skip=10pt
    ]
  \fi
}{
  \endtcolorbox
}


% Define a new counter for remarks
\newcounter{remarkcounter}
\renewcommand{\theremarkcounter}{R.\arabic{remarkcounter}}

\newenvironment{remark}[1][]{%
  \refstepcounter{remarkcounter}%
  \if\relax\detokenize{#1}\relax
    \tcolorbox[
      breakable,
      parbox=false, % Treat content normally regarding paragraphs
      before upper={\parindent0pt \parskip7pt}, % No indentation and add space between paragraphs
      colback=green!10,
      colframe=green!55,
      fonttitle=\bfseries,
      title={Remark \theremarkcounter},
      arc=5mm,
      boxrule=0.5mm,
      before skip=10pt, % Adjust vertical space before the box
      after skip=10pt % Adjust vertical space after the box
    ]
  \else
    \tcolorbox[
      breakable,
      parbox=false,
      before upper={\parindent0pt \parskip7pt},
      colback=green!10,
      colframe=green!55,
      fonttitle=\bfseries,
      title={Remark \theremarkcounter: #1},
      arc=5mm,
      boxrule=0.5mm,
      before skip=10pt,
      after skip=10pt
    ]
  \fi
}{
  \endtcolorbox
}

% Define a new counter for lemmas
\newcounter{lemmacounter}
\renewcommand{\thelemmacounter}{L.\arabic{lemmacounter}}

\newenvironment{lemma}[1][]{%
  \refstepcounter{lemmacounter}%
  \if\relax\detokenize{#1}\relax
    \tcolorbox[
      breakable,
      parbox=false, % Treat content normally regarding paragraphs
      before upper={\parindent0pt \parskip7pt}, % No indentation and add space between paragraphs
      colback=green!10,
      colframe=green!55,
      fonttitle=\bfseries,
      title={Lemma \thelemmacounter},
      arc=5mm,
      boxrule=0.5mm,
      before skip=10pt, % Adjust vertical space before the box
      after skip=10pt % Adjust vertical space after the box
    ]
  \else
    \tcolorbox[
      breakable,
      parbox=false,
      before upper={\parindent0pt \parskip7pt},
      colback=green!10,
      colframe=green!55,
      fonttitle=\bfseries,
      title={Lemma \thelemmacounter: #1},
      arc=5mm,
      boxrule=0.5mm,
      before skip=10pt,
      after skip=10pt
    ]
  \fi
}{
  \endtcolorbox
}

% Define a new counter for propositions
\newcounter{propositioncounter}
\renewcommand{\thepropositioncounter}{P.\arabic{propositioncounter}}

\newenvironment{proposition}[1][]{%
  \refstepcounter{propositioncounter}%
  \if\relax\detokenize{#1}\relax
    \tcolorbox[
      breakable,
      parbox=false, % Treat content normally regarding paragraphs
      before upper={\parindent0pt \parskip7pt}, % No indentation and add space between paragraphs
      colback=green!10,
      colframe=green!55,
      fonttitle=\bfseries,
      title={Proposition \thepropositioncounter},
      arc=5mm,
      boxrule=0.5mm,
      before skip=10pt, % Adjust vertical space before the box
      after skip=10pt % Adjust vertical space after the box
    ]
  \else
    \tcolorbox[
      breakable,
      parbox=false,
      before upper={\parindent0pt \parskip7pt},
      colback=green!10,
      colframe=green!55,
      fonttitle=\bfseries,
      title={Proposition \thepropositioncounter: #1},
      arc=5mm,
      boxrule=0.5mm,
      before skip=10pt,
      after skip=10pt
    ]
  \fi
}{
  \endtcolorbox
}

%\newtheorem{proposition}[theorem]{Proposition}  % Propositions share numbering with theorems


% Define a new counter for notes
\newcounter{notescounter}
\renewcommand{\thenotescounter}{D.\arabic{notescounter}}

\newenvironment{notes}[1][]{
  \refstepcounter{notescounter}%
  \if\relax\detokenize{#1}\relax
    % If #1 is empty, set the title to "Notes"
    \tcolorbox[
      breakable,
      parbox=false, % Treat content normally regarding paragraphs
      before upper={\parindent0pt \parskip7pt}, % No indentation and add space between paragraphs
      colback=blue!10,
      colframe=blue!50,
      fonttitle=\bfseries,
      title={Notes},
      arc=5mm,
      boxrule=0.5mm,
      before skip=10pt, % Adjust vertical space before the box
      after skip=10pt % Adjust vertical space after the box
    ]
  \else
    % If #1 is not empty, use it as the title
    \tcolorbox[
      breakable,
      parbox=false,
      before upper={\parindent0pt \parskip7pt},
      colback=blue!10,
      colframe=blue!50,
      fonttitle=\bfseries,
      title={#1}, % Use provided title instead of default
      arc=5mm,
      boxrule=0.5mm,
      before skip=10pt,
      after skip=10pt
    ]
  \fi
}{
  \endtcolorbox
}







% Define a new counter for questions
\newcounter{questionscounter}
\renewcommand{\thequestionscounter}{D.\arabic{questionscounter}}

\newenvironment{questions}[1][]{
  \refstepcounter{questionscounter}%
  \if\relax\detokenize{#1}\relax
    % If #1 is empty, set the title to "Questions"
    \tcolorbox[
      breakable,
      parbox=false, % Treat content normally regarding paragraphs
      before upper={\parindent0pt \parskip7pt}, % No indentation and add space between paragraphs
      colback=red!10,
      colframe=red!50,
      fonttitle=\bfseries,
      title={Questions},
      arc=5mm,
      boxrule=0.5mm,
      before skip=10pt, % Adjust vertical space before the box
      after skip=10pt % Adjust vertical space after the box
    ]
  \else
    % If #1 is not empty, use it as the title
    \tcolorbox[
      breakable,
      parbox=false,
      before upper={\parindent0pt \parskip7pt},
      colback=red!10,
      colframe=red!50,
      fonttitle=\bfseries,
      title={#1}, % Use provided title instead of default
      arc=5mm,
      boxrule=0.5mm,
      before skip=10pt,
      after skip=10pt
    ]
  \fi
}{
  \endtcolorbox
}






% Define a new counter for overview
\newcounter{overviewcounter}
\renewcommand{\theoverviewcounter}{D.\arabic{overviewcounter}}

\newenvironment{overview}[1][]{%
  \refstepcounter{overviewcounter}%
  \if\relax\detokenize{#1}\relax
    \tcolorbox[
      breakable,
      parbox=false, % Treat content normally regarding paragraphs
      before upper={\parindent0pt \parskip7pt}, % No indentation and add space between paragraphs
      colback=green!10,
      colframe=green!55,
      fonttitle=\bfseries,
      title={Overview},
      arc=5mm,
      boxrule=0.5mm,
      before skip=10pt, % Adjust vertical space before the box
      after skip=10pt % Adjust vertical space after the box
    ]
  \else
    \tcolorbox[
      breakable,
      parbox=false,
      before upper={\parindent0pt \parskip7pt},
      colback=green!10,
      colframe=green!55,
      fonttitle=\bfseries,
      title={Overview},
      arc=5mm,
      boxrule=0.5mm,
      before skip=10pt,
      after skip=10pt
    ]
  \fi
}{
  \endtcolorbox
}



% Add a line after paragraph header
\makeatletter
\renewcommand\paragraph{\@startsection{paragraph}{4}{\z@}%
            {-3.25ex \@plus -1ex \@minus -.2ex}%
            {1.5ex \@plus .2ex}%
            {\normalfont\normalsize\bfseries}}
\makeatother


% Taking away line before and after align
\BeforeBeginEnvironment{align}{\vspace{-\parskip}}
\AfterEndEnvironment{align}{\vskip0pt plus 2pt}

\usepackage{changepage}

\title{Ch. 5 Optimal Experiment Design Notes}

\author{Dylan Baker}
\date{October 2024}

\begin{document}
\maketitle

\section{Terms}



\begin{itemize}
    \item $\omega$: Worker
    \item $n$: Location where the worker lives and consumes
    \item $i$: Location where the worker works
    \item $C_{n \omega}$: Final good consumption
        \begin{align}
            C_n=\left[\sum_{i \in N} \int_0^{M_i} c_{n i}(j)^\rho d j\right]^{\frac{1}{\rho}}, \quad \sigma=\frac{1}{1-\rho}>1 \label{eq:good_consumption_index}
        \end{align}
        \begin{itemize}
            \item $\sigma$: The elasticity of substitution between different varieties of goods
        \end{itemize}
    \item $c_{n i}(j)$: Consumption in location $n$ of each variety, $j$, sourced from location $i$
    \item $H_{n \omega}$: Residential land use
    \item $b_{n i \omega}$: Idiosyncratic amenities shock
        \begin{itemize}
            \item This term ``captures the idea that individual workers can have
            idiosyncratic reasons for living and working in different locations.''
            \item Drawn from an independent Fréchet distribution
                \begin{align}
                    G_{n i}(b)=e^{-B_{n i} b^{-\epsilon}}, \quad B_{n i}>0, \epsilon>1
                \end{align}
                \begin{itemize}
                    \item $B_{n i}$: This is the ``scale parameter,'' which ``determines the average amenities from living in location $n$ and working in location $i$.''
                    \item $\epsilon$: This is the ``shape parameter,'' which ``controls the dispersion of amenities.''
                \end{itemize}
        \end{itemize}
    \item $\kappa_{n i}$: Iceberg commuting cost
        \begin{itemize}
            \item $\kappa_{n i} \in[1, \infty)$
        \end{itemize}
    \item $U_{n i \omega}$: Utility
        \begin{align}
            U_{n i \omega}=\frac{b_{n i \omega}}{\kappa_{n i}}\left(\frac{C_{n \omega}}{\alpha}\right)^\alpha\left(\frac{H_{n \omega}}{1-\alpha}\right)^{1-\alpha}
        \end{align}
        \begin{itemize}
            \item $\alpha$: The share of income spent on goods consumption
            \item $(1-\alpha)$: The share of income spent on housing
        \end{itemize}
    \item $X_n$: Aggregate expenditure in location $n$
    \item $P_n$: The price index dual to \eqref{eq:good_consumption_index}
    \item $p_{n i}(j)$: The  ```cost inclusive of freight' price of a variety $j$ produced in location $i$ and consumed in location $n$''
    \item $\bar{v}_n$: The average income of residents of $n$ (some of whom may work outside of $n$)
    \item $R_n$: The measure of residents in location $n$
    \item $Q_n$: Land price in location $n$
    \item $H_n$: The supply of land in location $n$
    \item $x_i(j)$: The number of variety $j$ produced in location $i$
    \item $F$: The fixed cost of producing a variety $j$
    \item $l_i(j)$: The labor input required to produce an amount of variety $j$, $x_i(j)$, in location $i$
    \item $d_{ni}$: The trade cost of shipping from location $i$ to location $n$
    \item $M_i$: The measure of varieties produced in location $i$
    \item $L_i$: The measure of workers in location $i$
    \item $G_{ni}(U)$: The CDF of the indirect utility of living in location $n$ and working in location $i$
        \begin{align}
            G_{n i}(U)=e^{-\Psi_{n i} U^{-\epsilon}}
        \end{align}
        where 
        \begin{align}
            \Psi_{n i}=B_{n i}\left(\kappa_{n i} P_n^\alpha Q_n^{1-\alpha}\right)^{-\epsilon} w_i^\epsilon
        \end{align}
    \item $\lambda_{ni}$: The probability that a worker lives in location $n$ and works in location $i$
        \begin{align}
            \lambda_{n i}=\frac{B_{n i}\left(\kappa_{n i} P_n^\alpha Q_n^{1-\alpha}\right)^{-\epsilon} w_i^\epsilon}{\sum_{r \in N} \sum_{s \in N} B_{r s}\left(\kappa_{r s} P_r^\alpha Q_r^{1-\alpha}\right)^{-\epsilon} w_s^\epsilon} \equiv \frac{\Phi_{n i}}{\Phi}
        \end{align}
    \item $\lambda_n^R$: The probability that a worker lives in location $n$ and works in location $i$
        \begin{align}
            \lambda_n^R=\frac{R_n}{\bar{L}}=\sum_{i \in N} \lambda_{n i}=\sum_{i \in N} \frac{\Phi_{n i}}{\Phi}
        \end{align}
    \item $\lambda_i^L$: The probability that a worker works in location $i$
        \begin{align}
            \lambda_i^L=\frac{L_n}{\bar{L}}=\sum_{n \in N} \lambda_{n i}=\sum_{n \in N} \frac{\Phi_{n i}}{\Phi}
        \end{align}
    \item $\lambda_{ni\mid n}^R$: The probability that a worker living in $n$ commutes to $i$.
        \begin{align}
            \lambda_{n i \mid n}^R \equiv \frac{\lambda_{n i}}{\lambda_n^R}=\frac{B_{n i}\left(w_i / \kappa_{n i}\right)^\epsilon}{\sum_{s \in N} B_{n s}\left(w_s / \kappa_{n s}\right)^\epsilon}
        \end{align}
\end{itemize}


\section{The Model}

\subsection{Preferences and Endowments}

\subsubsection{Preferences}

The preferences of a worker who lives in
location $n$ and works in location $i$ 
is given by the following utility function
of the Cobb-Douglas form:

\begin{align}
    U_{n i \omega}=\frac{b_{n i \omega}}{\kappa_{n i}}\left(\frac{C_{n \omega}}{\alpha}\right)^\alpha\left(\frac{H_{n \omega}}{1-\alpha}\right)^{1-\alpha}
\end{align}

Idiosyncratic amenities 
are drawn from 
an independent Fréchet distribution:

\begin{align}
    G_{n i}(b)=e^{-B_{n i} b^{-\epsilon}}, \quad B_{n i}>0, \epsilon>1
\end{align}

\subsubsection{Good Consumption Index}

The good consumption index is given the form:

\begin{align}
    C_n=\left[\sum_{i \in N} \int_0^{M_i} c_{n i}(j)^\rho d j\right]^{\frac{1}{\rho}}, \quad \sigma=\frac{1}{1-\rho}>1
\end{align}

``The goods consumption index in location $n$ is a constant elasticity of 
substitution (CES) function of consumption of a continuum of tradable 
varieties sourced from each location $i$.''

Utility maximization 
will give that ``the equilibrium consumption 
in location $n$ of each variety sourced from 
location $i$ 
is given by'':

\begin{align}
    c_{n i}(j)=\alpha X_n P_n^{\sigma-1} p_{n i}(j)^{-\sigma} \label{eq:good_nij_consumption}
\end{align}

See \autoref{sec:good_nij_consumption} for derivation.

\subsubsection{Land and Local Consumption}

``We assume that this land is owned by immobile landlords,
who receive worker expenditure on residential land as income, and consume only
goods where they live.''

From there, we get the expression

\begin{align}
    P_n C_n=\alpha \bar{v}_n R_n+(1-\alpha) \bar{v}_n R_n=\bar{v}_n R_n
\end{align}

which says that the 
total expenditure on goods in location $n$, $P_n C_n$
is equal to the total labor income of 
residents in location $n$, $\bar{v}_n R_n$.

\begin{questions}
    Should I be saying ``total labor income'' or just ``total income''
    for $\bar{v}_n R_n$?
\end{questions}

The middle term can be read as 
residents total spending on goods in $n$,
$\alpha \bar{v}_n R_n$, plus 
residents total spending on land in $n$,
$(1-\alpha) \bar{v}_n R_n$.

We can also get the following expression:

\begin{align}
    Q_n=(1-\alpha) \frac{\bar{v}_n R_n}{H_n}
\end{align}

which says that 
the land price in location $n$, $Q_n$,
is equal to the total spending on land in $n$ 
divided by the supply of land in $n$.

This follows from the land market clearing condition:

\begin{align}
    \underbrace{Q_n \times H_n}_{\text {price } \times \text { quantity of land }}=\underbrace{(1-\alpha) \bar{v}_n R_n}_{\text {total rent paid by residents }}
\end{align}

and is useful, because it allows us to express 
rent as a function of the supply of land.

\begin{questions}
    Why do we only say that it's a function of the supply of land?
    It seems to be a function of several things, no?
\end{questions}


%%%%%%%%%%%%%%%%%%%%%%%%%%%%%%%%%%%%%%%%%%%%%%%%%%%%%%%%%%%%%%%%%%%%%%%%%%%%%%%%%%%%%%%
%%%%%%%%%%%%%%%%%%%%%%%%%%%%%%%%%%%%%%%%%%%%%%%%%%%%%%%%%%%%%%%%%%%%%%%%%%%%%%%%%%%%%%%
%%%%%%%%%%%%%%%%%%%%%%%%%%%%%%%%%%%%%%%%%%%%%%%%%%%%%%%%%%%%%%%%%%%%%%%%%%%%%%%%%%%%%%%

\section{Production}

Firms produce tradable varieties under 
monopolistic competition
and increasing returns to scale
using labor as the lone input. 

To produce a variety, firms incur 
a fixed cost, $F$, as well as a 
variable cost
that is determined by the inverse 
of local productivity, $A_i(j)$: $x_i(j)/A_i(j)$.

Thus, the total amount of labor, $l_i(j)$, required to produce
$x_i(j)$ units of variety $j$ in location $i$ is
given by:

\begin{align}
    l_i(j)=F+ \frac{x_i(j)}{A_i(j)}
\end{align}


%%%%%%%%%%%%%%%%%%%%%%%%%%%%%%%%%%%%%%%%%%%%%%%%%%%%%%%%%%%%%%%%%%%%%%%%%%%%%%%%%%%%%%%
%%%%%%%%%%%%%%%%%%%%%%%%%%%%%%%%%%%%%%%%%%%%%%%%%%%%%%%%%%%%%%%%%%%%%%%%%%%%%%%%%%%%%%%
%%%%%%%%%%%%%%%%%%%%%%%%%%%%%%%%%%%%%%%%%%%%%%%%%%%%%%%%%%%%%%%%%%%%%%%%%%%%%%%%%%%%%%%
\section{Derivations}

\subsection{Derivation of $c_{n i}(j)$ Expression} 
\label{sec:good_nij_consumption}

\subsubsection{Write the Problem}
This is the derivation of \eqref{eq:good_nij_consumption}.

When making consumption decisions surrounding
$c_{n i}(j)$, an individual is solving the following problem:

\begin{align}
    &\underset{\{c_{n i}(j)\}}{\max} C_n \\ 
    \text{s.t. } &\sum_i \int_{0}^{M_i} p_{n i}(j) c_{n i}(j) d j=\alpha X_n
\end{align}

or expanded out:

\begin{align}
    &\underset{\{c_{n i}(j)\}}{\max} \left[\sum_{i \in N} \int_0^{M_i} c_{n i}(j)^\rho d j\right]^{\frac{1}{\rho}} \\
    \text{s.t. } &\sum_i \int_{0}^{M_i} p_{n i}(j) c_{n i}(j) d j=\alpha X_n
\end{align}

where the $\alpha X_n$ term comes from the fact that 
people spend $\alpha$ of their total expenditures, $X_n$,
on goods.

\subsubsection{Lagrangian and FOCs}

The Langragian for this problem is then:

\begin{align}
    \mathcal{L} = \left[\sum_{i \in N} \int_0^{M_i} c_{n i}(j)^\rho d j\right]^{\frac{1}{\rho}} + \lambda \left(\alpha X_n - \sum_i \int_{0}^{M_i} p_{n i}(j) c_{n i}(j) d j\right)
\end{align}

which gives the relevant FOC:

\begin{align}
    \{c_{n i}(j)\} \quad \quad &\frac{\partial}{\partial c_{n i}(j)} \left[\left(\sum_{i \in N} \int_0^{M_i} c_{n i}(j)^\rho d j\right)^{\frac{1}{\rho}}\right] - \lambda p_{n i}(j) = 0 \\
    \Leftrightarrow & \frac{1}{\rho} \left[\sum_{i \in N} \int_0^{M_i} \rho c_{n i}(j)^\rho d j\right]^{\frac{1}{\rho}-1} c_{n i}(j)^{\rho-1} - \lambda p_{n i}(j) = 0 \\
    \Leftrightarrow & C_n^{1-\rho} c_{n i}(j)^{\rho-1} = \lambda p_{n i}(j) \\
    \Leftrightarrow & c_{n i}(j) = \lambda^{\frac{1}{\rho -1}} p_{ni}(j)^{\frac{1}{\rho -1}} C_n \\ 
    \Leftrightarrow & c_{n i}(j) = \lambda^{-\sigma} p_{ni}(j)^{-\sigma} C_n && \text{since $\sigma = \frac{1}{1-\rho}$} \label{eq:inter_good_nij_cons}
\end{align}


\subsubsection{Solve for $\lambda$}

From there, we can define the dual price index 

\begin{align}
    P_n \equiv \left(\sum_{i \in N} \int_0^{M_i} p_{n i}(j)^{1-\sigma} d_j \right)^{\frac{1}{1-\sigma}}
\end{align}

and revisit to our budget constraint:

\begin{alignat}{2}
    &\alpha X_n && = \sum_i \int_{0}^{M_i} p_{n i}(j) c_{n i}(j) d j \\
    &&&= \sum_i \int_{0}^{M_i} p_{n i}(j) \lambda^{-\sigma} p_{ni}(j)^{-\sigma} C_n d j \quad \text{by \eqref{eq:inter_good_nij_cons}} \\
    &&&= \lambda^{-\sigma} C_n \sum_i \int_{0}^{M_i} p_{n i}(j)^{1-\sigma} d j \\
    &&& = \lambda^{-\sigma} C_n P_n^{1-\sigma} \\
    \Rightarrow &\lambda^{-\sigma} &&= \frac{\alpha X_n}{C_n} P_n^{\sigma-1} \\
\end{alignat}

\subsubsection{Plug in $\lambda$}

Then, returning to \eqref{eq:inter_good_nij_cons}, we can plug in our expression for $\lambda^{-\sigma}$:

\begin{align}
    c_{n i}(j) &= \lambda^{-\sigma} p_{ni}(j)^{-\sigma} C_n \\
    &= \left(\frac{\alpha X_n}{C_n} P_n^{\sigma-1}\right) p_{ni}(j)^{-\sigma} C_n \\
    &= \alpha X_n P_n^{\sigma-1} p_{ni}(j)^{-\sigma}
\end{align}

which is what we wanted.

\end{document}
