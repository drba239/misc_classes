\section{Chapter 1: Introduction}

\subsection{The Actors in the Labor Market}

This book will consider three 
actors within the labor market: 
(1) workers, 
(2) firms, and
(3) government.
The decisions of workers will be based on a desire 
to optimize what Borjas calls their well-being.
These decisions across workers generate the labor supply curve.
Firms have the goal of maximizing profits.
The firm's demand for labor is a ``derived demand''
in the sense that it is derived from consumer's 
demand for the firm's output.
Equilibrium is attained when supply equals demand in a 
free market economy.
The government's motives are left more opaque.

%%%%%%%%%%%%%%%%%%%%%%%%%%%%%%%%%%%%%%%%%%%%%%%%%%%%%%%%%%%%%%%%%%%%%%%%%%%%%%%%%%%%%%%
%%%%%%%%%%%%%%%%%%%%%%%%%%%%%%%%%%%%%%%%%%%%%%%%%%%%%%%%%%%%%%%%%%%%%%%%%%%%%%%%%%%%%%%
\subsection{Why Do We Need a Theory?}

Writing out supply and demand curves reflects the 
construction of a model, which makes predictions
about what will transpire if certain conditions change.
The model is simple but is useful for organizing our thoughts 
about the labor market and provides a solid foundation upon 
which to build more complex infrastructure.

The predictions of the supply and demand model
is an example of positive economics. 
Positive economics is concerned with
``What is?'' questions, i.e., 
questions about how the world actually works.
This is in contrast to normative economics,
which is concerned with
questions of ``What ought to be?''

%%%%%%%%%%%%%%%%%%%%%%%%%%%%%%%%%%%%%%%%%%%%%%%%%%%%%%%%%%%%%%%%%%%%%%%%%%%%%%%%%%%%%%%
%%%%%%%%%%%%%%%%%%%%%%%%%%%%%%%%%%%%%%%%%%%%%%%%%%%%%%%%%%%%%%%%%%%%%%%%%%%%%%%%%%%%%%%
%%%%%%%%%%%%%%%%%%%%%%%%%%%%%%%%%%%%%%%%%%%%%%%%%%%%%%%%%%%%%%%%%%%%%%%%%%%%%%%%%%%%%%%
