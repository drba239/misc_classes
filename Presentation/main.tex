% -------------------------------------
% Setup 
% -------------------------------------
\documentclass[12pt, aspectratio=169]{beamer}

\usepackage{amsmath}
\usepackage{amssymb}
\usepackage[backend=biber,style=authoryear]{biblatex}
\usepackage{booktabs}
\usepackage{epsfig}
\usepackage{fontawesome}
\usepackage{graphicx}
\usepackage{hyperref}
\usepackage[utf8x]{inputenc}
\usepackage{mathtools}
\usepackage{pgfplots}
\usepackage{tikz}

\usepackage{tcolorbox}


\mathtoolsset{showonlyrefs=true}

\DeclareCiteCommand{\cite}
  {\usebibmacro{prenote}}
  {\usebibmacro{citeindex}%
   \printnames{labelname}%
   \space(\printfield{year})}
  {\multicitedelim}
  {\usebibmacro{postnote}}




% Spacing
\usepackage{parskip}
\usepackage{etoolbox}
\usepackage{setspace}
\setlength{\itemsep}{10em}
\usepackage{adjustbox}






\usepackage{physics}
% \DeclareMathOperator*{\argmax}{arg\,max}
% \DeclareMathOperator*{\argmin}{arg\,min}
\usepackage{listings}
\usepackage{booktabs}
\usepackage{caption}
\hypersetup{
    %colorlinks=true,
    %linkcolor=blue,
    %filecolor=magenta,      
    %urlcolor=cyan  
    }
\usepackage{threeparttable}
\usepackage{pdflscape}
\usepackage{float}
%\usepackage{natbib}
\usepackage{multirow}


%%%%%%%%%%%%%%%%%%%%%%%%%%%%%%%%%%%%%%%%%%%%%%%%%%%%%%%%%%%%%%%%%%%%%%%%%%%%%%%%%%%%%%%
% Shortcuts 
%%%%%%%%%%%%%%%%%%%%%%%%%%%%%%%%%%%%%%%%%%%%%%%%%%%%%%%%%%%%%%%%%%%%%%%%%%%%%%%%%%%%%%%


% % Shortcut greek
% \def\a{\alpha}
% \def\b{\beta}
% \def\g{\gamma}
% \def\D{\Delta}
% \def\d{\delta}
% \def\z{\zeta}
% \def\k{\kappa}
% \def\l{\lambda}
% \def\n{\nu}
% \def\r{\rho}
% \def\s{\sigma}
% \def\t{\tau}
% \def\x{\xi}
% \def\w{\omega}
% \def\W{\Omega}
% \def\th{\theta}
% \newcommand{\ep}{\varepsilon}

% Probability operators
% \newcommand{\V}{\mathbb{V}}
% \renewcommand{\P}{\mathbb{P}}
% \newcommand{\E}{\mathbb{E}}
% \renewcommand{\var}{\operatorname{var}}
% \newcommand{\cov}{\operatorname{cov}}
% \newcommand{\1}{\mathbbm{1}}
% \newcommand{\R}{\mathbb{R}}
% \newcommand{\supp}{\operatorname{supp}}
% \newcommand{\pcon}{\overset{p}{\to}}
% \newcommand{\dcon}{\overset{d}{\to}}
% \newcommand{\ascon}{\overset{a.s.}{\to}}
% \newcommand{\rcon}{\overset{r}{\to}}


% % Question format
% \newcommand{\question}[1]{ \begin{center} \noindent\colorbox{gray!10}{
% \parbox{0.8\textwidth}{\vspace{0.125in} #1 \vspace{0.125in} } } \end{center} }
% \newcommand{\subq}[1]{ \begin{center} \noindent\colorbox{gray!10}{
% \parbox{0.8\textwidth}{\vspace{0.125in} #1 \vspace{0.125in} } } \end{center} }


% % Other formatting shortcuts
% \newcommand{\notimplies}{\;\not\!\!\!\implies}
% %\newcommand{\ul}{\underline}
% \renewcommand{\bf}{\textbf}
% \newcommand{\fan}{\mathcal}
% \newcommand{\red}{\textcolor{red}}
% \newcommand{\green}{\textcolor{green}}
% \newcommand{\blue}{\textcolor{blue}}
% \newcommand{\gray}{\textcolor{gray}}




% ------------------------------------------------------------------------------
% Use the metropolis beamer template
% ------------------------------------------------------------------------------
\usepackage[T1]{fontenc}
\definecolor{maroon}{rgb}{0.5, 0.0, 0.0}
\mode<presentation>
{
  \usetheme[progressbar=foot,numbering=fraction,background=light]{metropolis} 
  \usecolortheme{beaver} % or try albatross, beaver, crane, ...
  \setbeamercolor{title}{fg=maroon}
  \setbeamercolor{background canvas}{bg=white}
  \setbeamercolor{subsection title}{fg=maroon}
  \usefonttheme{professionalfonts}  % or try serif, structurebold, ...
  \setbeamertemplate{navigation symbols}{}
  \setbeamertemplate{caption}[numbered]
  \setbeamercolor{section title}{fg=white, bg=maroon}
}

% \setbeamertemplate{section page}{
%   \vfill
%   \centering
%   \begin{beamercolorbox}[sep=8pt,center,shadow=false,rounded=false]{section title}
%     \usebeamerfont{title}\usebeamercolor{title}\insertsectionhead\par%
%   \end{beamercolorbox}
%   \vfill
% }

% bibliography stuff
% \setbeamertemplate{frametitle continuation}{}
% \setbeamertemplate{bibliography entry title}{}
% \setbeamertemplate{bibliography entry location}{}
% \setbeamertemplate{bibliography entry note}{}
% \renewcommand{\bibsection}{} %natbib

% \newenvironment{transitionframe}{
%   \setbeamercolor{background canvas}{bg=maroon,fg=white}
%   \color{white}
%   \centering
%   \LARGE
%   \begin{center}
%   \begin{frame}[plain, noframenumbering]}{
%   \end{frame}
%   \end{center}
% }

\setbeamertemplate{note page}[plain]
\usepackage{appendixnumberbeamer}



% Counter for notes
\newcounter{notescounter}

% Define the notes environment
\newenvironment{notes}[1][]{
  \refstepcounter{notescounter}%
  \if\relax\detokenize{#1}\relax
    % If #1 is empty, set the title to "Notes"
    \begin{tcolorbox}[
      colback=blue!10,
      colframe=blue!50,
      fonttitle=\bfseries,
      title={Notes},
      arc=5mm,
      boxrule=0.5mm,
      width=\textwidth,
      before skip=9pt,
      after skip=9pt
    ]
  \else
    % If #1 is not empty, use it as the title
    \begin{tcolorbox}[
      colback=blue!10,
      colframe=blue!50,
      fonttitle=\bfseries,
      title={#1},
      arc=5mm,
      boxrule=0.5mm,
      width=\textwidth,
      before skip=9pt,
      after skip=9pt
    ]
  \fi
}{
  \end{tcolorbox}
}

\usepackage{soul}
\usepackage{tikz}
\usetikzlibrary{tikzmark}

% SET VARIABLES 
\newcommand\nn{622} 
\newcommand\paneltot{7,344}
\newcommand\panelind{604}
\newcommand\boots{200}

% OUTCOMES
\newcommand\yrform{71}
\newcommand\yrinf{93}
\newcommand\yrmin{5}
\newcommand\yrmax{18}
\newcommand\suppref{28.6}
\newcommand\formpref{37.4}


% -------------------------------------
% Begin Document 
% -------------------------------------
\begin{document}
\title{Some Stuff}
\author{Dylan Baker}
\date{November 22, 2024}

\begin{frame}[plain, noframenumbering] 

\maketitle

\end{frame}

\begin{frame}{The Effects of Media Coverage of Immigration}

    \begin{itemize}
        \item Basically, I'm about to talk about the effect of media coverage of immigration on 
            views about immigration/immigrants and support for anti-immigration parties/policies.
        \item My concern is that when we have ``the effects of immigration''
            studies, we're essentially looking at environments where everyone is
            being ``treated'' by a key mechanism (media coverage) and there's some variation around a high-mean dosage.
        \item Example: Why are people in Wyoming mad about immigration while also not experiencing inflows of immigrants?
    \end{itemize}
    
    
\end{frame}


\begin{frame}{Studies on the Effects of Immigration}

\begin{itemize}
\item There are a number of studies that look at the effects of immigration on 
    political behavior, e.g.,:
    \begin{itemize}
        \item Halla, M., Wagner, A. F., \& Zweimüller, J. (2017). Immigration and Voting for the Far Right. \emph{Journal of the European Economic Association, 15}(6), 1341–1385. \href{https://doi.org/10.1093/jeea/jvx003}{https://doi.org/10.1093/jeea/jvx003}
        \item Mayda, A. M., Peri, G., \& Steingress, W. (2022). The Political Impact of Immigration: Evidence from the United States. \emph{American Economic Journal: Applied Economics, 14}(1), 358-389. \href{https://doi.org/10.1257/app.20190081}{https://doi.org/10.1257/app.20190081}
    \end{itemize}    
\end{itemize}

\end{frame}

\begin{frame}{Studies on Immigration Coverage}

    \begin{itemize}
        \item There have been fewer, but still some, papers on immigration media coverage, e.g.,:
            \begin{itemize}
                \item Benesch, C., Loretz, S., Stadelmann, D., \& Thomas, T. (2019). Media coverage and immigration worries: Econometric evidence. \emph{Journal of Economic Behavior \& Organization, 160}, 52–67. \href{https://doi.org/10.1016/j.jebo.2019.02.011}{https://doi.org/10.1016/j.jebo.2019.02.011}
                \item Keita, S., Renault, T., \& Valette, J. (2024). The Usual Suspects: Offender Origin, Media Reporting and Natives’ Attitudes Towards Immigration. \emph{The Economic Journal, 134}(657), 322–362. \href{https://doi.org/10.109}{https://doi.org/10.109}
                \item Djourelova, M. (2023). Persuasion through Slanted Language: Evidence from the Media Coverage of Immigration. \emph{American Economic Review, 113}(3), 800–835. \href{https://doi.org/10.1257/aer.20211537}{https://doi.org/10.1257/aer.20211537}
            \end{itemize}
    \end{itemize}
    
\end{frame}

\begin{frame}{Main Results and a Concern}
    \begin{itemize}
        \item In general, studies are showing that there's immigration into an area 
            increases anti-immigrant sentiment and support for anti-immigration parties. (I am skipping nuance here.)
        \item Moreover, in general, immigration media coverage is shown to 
            increase concern about migration.
        \item I'm concerned that a lot of the effect that we're interested in is 
        hiding in the intercept or time fixed effects of these models.
    \end{itemize}
\end{frame}

\begin{frame}{Examples}

    \begin{itemize}
        \item Motivating Example: 
        \begin{itemize}
            \item The Austrian Anti-Immigrant Far Right Party went from 
                15.29\% to 30.36\% in its vote share from 2014 to 2015. Steinmayr (2021)
                finds that exposure to passing migrants increases support for the 
                Far Right party by 1.5-2.3 percentage points.
            \item Benesch et al. (2019) look at plausibly exogenous changes in 
                media coverage of immigration in Germany via the timing 
                of Swiss immigration-related referenda. They find that an 
                additional news item is associated with a 0.008 point increase 
                in immigration concerns (on a 3 point scale) and
                a 10 percentage point increase in the percent of news items 
                that are immigration related is associated with a 0.1314 point increase.
        \end{itemize}
    \end{itemize}

\end{frame}



\begin{frame}{The Concern Revisited}
    \begin{itemize}
        \item We may be looking at environments where everyone is 
            being ``treated'' by high levels of immigration coverage, so we're getting 
            the effect of marginal changes in coverage or inflows (which may 
            adjust coverage) around a high baseline, while 
            being unable to think about the effects of moving from low 
            to high coverage because we lack a good control group.
    \end{itemize}
\end{frame}

\begin{frame}{Thoughts on Addressing This}
    
    \begin{itemize}
        \item I need to identify settings where there is greater variation in 
            the broader
            media environment.
        \item Possible Avenues: Looking at earlier American history
            \begin{itemize}
                \item Something formulation of the rise of telegraph lines, 
                    since that gave smaller towns access to Associated Press news wires,
                    which would cover immigration news that the local newspaper 
                    may not have otherwise covered.
                    \begin{itemize}
                        \item Big endogeneity issues + how to isolate immigration coverage
                    \end{itemize}
                \item Inflows of immigrants into the headquarters of firms that own 
                    newspaper groups in many areas.
            \end{itemize}
    \end{itemize}

\end{frame}





\end{document}